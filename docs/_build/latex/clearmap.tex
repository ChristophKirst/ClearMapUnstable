% Generated by Sphinx.
\def\sphinxdocclass{report}
\documentclass[letterpaper,10pt,english]{sphinxmanual}
\usepackage[utf8]{inputenc}
\DeclareUnicodeCharacter{00A0}{\nobreakspace}
\usepackage{cmap}
\usepackage[T1]{fontenc}
\usepackage{babel}
\usepackage{times}
\usepackage[Bjarne]{fncychap}
\usepackage{longtable}
\usepackage{sphinx}
\usepackage{multirow}


\addto\captionsenglish{\renewcommand{\figurename}{Fig. }}
\addto\captionsenglish{\renewcommand{\tablename}{Table }}
\floatname{literal-block}{Listing }



\title{ClearMap Documentation}
\date{December 19, 2015}
\release{0.9.2}
\author{Christoph Kirst}
\newcommand{\sphinxlogo}{}
\renewcommand{\releasename}{Release}
\makeindex

\makeatletter
\def\PYG@reset{\let\PYG@it=\relax \let\PYG@bf=\relax%
    \let\PYG@ul=\relax \let\PYG@tc=\relax%
    \let\PYG@bc=\relax \let\PYG@ff=\relax}
\def\PYG@tok#1{\csname PYG@tok@#1\endcsname}
\def\PYG@toks#1+{\ifx\relax#1\empty\else%
    \PYG@tok{#1}\expandafter\PYG@toks\fi}
\def\PYG@do#1{\PYG@bc{\PYG@tc{\PYG@ul{%
    \PYG@it{\PYG@bf{\PYG@ff{#1}}}}}}}
\def\PYG#1#2{\PYG@reset\PYG@toks#1+\relax+\PYG@do{#2}}

\expandafter\def\csname PYG@tok@gd\endcsname{\def\PYG@tc##1{\textcolor[rgb]{0.63,0.00,0.00}{##1}}}
\expandafter\def\csname PYG@tok@gu\endcsname{\let\PYG@bf=\textbf\def\PYG@tc##1{\textcolor[rgb]{0.50,0.00,0.50}{##1}}}
\expandafter\def\csname PYG@tok@gt\endcsname{\def\PYG@tc##1{\textcolor[rgb]{0.00,0.27,0.87}{##1}}}
\expandafter\def\csname PYG@tok@gs\endcsname{\let\PYG@bf=\textbf}
\expandafter\def\csname PYG@tok@gr\endcsname{\def\PYG@tc##1{\textcolor[rgb]{1.00,0.00,0.00}{##1}}}
\expandafter\def\csname PYG@tok@cm\endcsname{\let\PYG@it=\textit\def\PYG@tc##1{\textcolor[rgb]{0.25,0.50,0.56}{##1}}}
\expandafter\def\csname PYG@tok@vg\endcsname{\def\PYG@tc##1{\textcolor[rgb]{0.73,0.38,0.84}{##1}}}
\expandafter\def\csname PYG@tok@m\endcsname{\def\PYG@tc##1{\textcolor[rgb]{0.13,0.50,0.31}{##1}}}
\expandafter\def\csname PYG@tok@mh\endcsname{\def\PYG@tc##1{\textcolor[rgb]{0.13,0.50,0.31}{##1}}}
\expandafter\def\csname PYG@tok@cs\endcsname{\def\PYG@tc##1{\textcolor[rgb]{0.25,0.50,0.56}{##1}}\def\PYG@bc##1{\setlength{\fboxsep}{0pt}\colorbox[rgb]{1.00,0.94,0.94}{\strut ##1}}}
\expandafter\def\csname PYG@tok@ge\endcsname{\let\PYG@it=\textit}
\expandafter\def\csname PYG@tok@vc\endcsname{\def\PYG@tc##1{\textcolor[rgb]{0.73,0.38,0.84}{##1}}}
\expandafter\def\csname PYG@tok@il\endcsname{\def\PYG@tc##1{\textcolor[rgb]{0.13,0.50,0.31}{##1}}}
\expandafter\def\csname PYG@tok@go\endcsname{\def\PYG@tc##1{\textcolor[rgb]{0.20,0.20,0.20}{##1}}}
\expandafter\def\csname PYG@tok@cp\endcsname{\def\PYG@tc##1{\textcolor[rgb]{0.00,0.44,0.13}{##1}}}
\expandafter\def\csname PYG@tok@gi\endcsname{\def\PYG@tc##1{\textcolor[rgb]{0.00,0.63,0.00}{##1}}}
\expandafter\def\csname PYG@tok@gh\endcsname{\let\PYG@bf=\textbf\def\PYG@tc##1{\textcolor[rgb]{0.00,0.00,0.50}{##1}}}
\expandafter\def\csname PYG@tok@ni\endcsname{\let\PYG@bf=\textbf\def\PYG@tc##1{\textcolor[rgb]{0.84,0.33,0.22}{##1}}}
\expandafter\def\csname PYG@tok@nl\endcsname{\let\PYG@bf=\textbf\def\PYG@tc##1{\textcolor[rgb]{0.00,0.13,0.44}{##1}}}
\expandafter\def\csname PYG@tok@nn\endcsname{\let\PYG@bf=\textbf\def\PYG@tc##1{\textcolor[rgb]{0.05,0.52,0.71}{##1}}}
\expandafter\def\csname PYG@tok@no\endcsname{\def\PYG@tc##1{\textcolor[rgb]{0.38,0.68,0.84}{##1}}}
\expandafter\def\csname PYG@tok@na\endcsname{\def\PYG@tc##1{\textcolor[rgb]{0.25,0.44,0.63}{##1}}}
\expandafter\def\csname PYG@tok@nb\endcsname{\def\PYG@tc##1{\textcolor[rgb]{0.00,0.44,0.13}{##1}}}
\expandafter\def\csname PYG@tok@nc\endcsname{\let\PYG@bf=\textbf\def\PYG@tc##1{\textcolor[rgb]{0.05,0.52,0.71}{##1}}}
\expandafter\def\csname PYG@tok@nd\endcsname{\let\PYG@bf=\textbf\def\PYG@tc##1{\textcolor[rgb]{0.33,0.33,0.33}{##1}}}
\expandafter\def\csname PYG@tok@ne\endcsname{\def\PYG@tc##1{\textcolor[rgb]{0.00,0.44,0.13}{##1}}}
\expandafter\def\csname PYG@tok@nf\endcsname{\def\PYG@tc##1{\textcolor[rgb]{0.02,0.16,0.49}{##1}}}
\expandafter\def\csname PYG@tok@si\endcsname{\let\PYG@it=\textit\def\PYG@tc##1{\textcolor[rgb]{0.44,0.63,0.82}{##1}}}
\expandafter\def\csname PYG@tok@s2\endcsname{\def\PYG@tc##1{\textcolor[rgb]{0.25,0.44,0.63}{##1}}}
\expandafter\def\csname PYG@tok@vi\endcsname{\def\PYG@tc##1{\textcolor[rgb]{0.73,0.38,0.84}{##1}}}
\expandafter\def\csname PYG@tok@nt\endcsname{\let\PYG@bf=\textbf\def\PYG@tc##1{\textcolor[rgb]{0.02,0.16,0.45}{##1}}}
\expandafter\def\csname PYG@tok@nv\endcsname{\def\PYG@tc##1{\textcolor[rgb]{0.73,0.38,0.84}{##1}}}
\expandafter\def\csname PYG@tok@s1\endcsname{\def\PYG@tc##1{\textcolor[rgb]{0.25,0.44,0.63}{##1}}}
\expandafter\def\csname PYG@tok@gp\endcsname{\let\PYG@bf=\textbf\def\PYG@tc##1{\textcolor[rgb]{0.78,0.36,0.04}{##1}}}
\expandafter\def\csname PYG@tok@sh\endcsname{\def\PYG@tc##1{\textcolor[rgb]{0.25,0.44,0.63}{##1}}}
\expandafter\def\csname PYG@tok@ow\endcsname{\let\PYG@bf=\textbf\def\PYG@tc##1{\textcolor[rgb]{0.00,0.44,0.13}{##1}}}
\expandafter\def\csname PYG@tok@sx\endcsname{\def\PYG@tc##1{\textcolor[rgb]{0.78,0.36,0.04}{##1}}}
\expandafter\def\csname PYG@tok@bp\endcsname{\def\PYG@tc##1{\textcolor[rgb]{0.00,0.44,0.13}{##1}}}
\expandafter\def\csname PYG@tok@c1\endcsname{\let\PYG@it=\textit\def\PYG@tc##1{\textcolor[rgb]{0.25,0.50,0.56}{##1}}}
\expandafter\def\csname PYG@tok@kc\endcsname{\let\PYG@bf=\textbf\def\PYG@tc##1{\textcolor[rgb]{0.00,0.44,0.13}{##1}}}
\expandafter\def\csname PYG@tok@c\endcsname{\let\PYG@it=\textit\def\PYG@tc##1{\textcolor[rgb]{0.25,0.50,0.56}{##1}}}
\expandafter\def\csname PYG@tok@mf\endcsname{\def\PYG@tc##1{\textcolor[rgb]{0.13,0.50,0.31}{##1}}}
\expandafter\def\csname PYG@tok@err\endcsname{\def\PYG@bc##1{\setlength{\fboxsep}{0pt}\fcolorbox[rgb]{1.00,0.00,0.00}{1,1,1}{\strut ##1}}}
\expandafter\def\csname PYG@tok@mb\endcsname{\def\PYG@tc##1{\textcolor[rgb]{0.13,0.50,0.31}{##1}}}
\expandafter\def\csname PYG@tok@ss\endcsname{\def\PYG@tc##1{\textcolor[rgb]{0.32,0.47,0.09}{##1}}}
\expandafter\def\csname PYG@tok@sr\endcsname{\def\PYG@tc##1{\textcolor[rgb]{0.14,0.33,0.53}{##1}}}
\expandafter\def\csname PYG@tok@mo\endcsname{\def\PYG@tc##1{\textcolor[rgb]{0.13,0.50,0.31}{##1}}}
\expandafter\def\csname PYG@tok@kd\endcsname{\let\PYG@bf=\textbf\def\PYG@tc##1{\textcolor[rgb]{0.00,0.44,0.13}{##1}}}
\expandafter\def\csname PYG@tok@mi\endcsname{\def\PYG@tc##1{\textcolor[rgb]{0.13,0.50,0.31}{##1}}}
\expandafter\def\csname PYG@tok@kn\endcsname{\let\PYG@bf=\textbf\def\PYG@tc##1{\textcolor[rgb]{0.00,0.44,0.13}{##1}}}
\expandafter\def\csname PYG@tok@o\endcsname{\def\PYG@tc##1{\textcolor[rgb]{0.40,0.40,0.40}{##1}}}
\expandafter\def\csname PYG@tok@kr\endcsname{\let\PYG@bf=\textbf\def\PYG@tc##1{\textcolor[rgb]{0.00,0.44,0.13}{##1}}}
\expandafter\def\csname PYG@tok@s\endcsname{\def\PYG@tc##1{\textcolor[rgb]{0.25,0.44,0.63}{##1}}}
\expandafter\def\csname PYG@tok@kp\endcsname{\def\PYG@tc##1{\textcolor[rgb]{0.00,0.44,0.13}{##1}}}
\expandafter\def\csname PYG@tok@w\endcsname{\def\PYG@tc##1{\textcolor[rgb]{0.73,0.73,0.73}{##1}}}
\expandafter\def\csname PYG@tok@kt\endcsname{\def\PYG@tc##1{\textcolor[rgb]{0.56,0.13,0.00}{##1}}}
\expandafter\def\csname PYG@tok@sc\endcsname{\def\PYG@tc##1{\textcolor[rgb]{0.25,0.44,0.63}{##1}}}
\expandafter\def\csname PYG@tok@sb\endcsname{\def\PYG@tc##1{\textcolor[rgb]{0.25,0.44,0.63}{##1}}}
\expandafter\def\csname PYG@tok@k\endcsname{\let\PYG@bf=\textbf\def\PYG@tc##1{\textcolor[rgb]{0.00,0.44,0.13}{##1}}}
\expandafter\def\csname PYG@tok@se\endcsname{\let\PYG@bf=\textbf\def\PYG@tc##1{\textcolor[rgb]{0.25,0.44,0.63}{##1}}}
\expandafter\def\csname PYG@tok@sd\endcsname{\let\PYG@it=\textit\def\PYG@tc##1{\textcolor[rgb]{0.25,0.44,0.63}{##1}}}

\def\PYGZbs{\char`\\}
\def\PYGZus{\char`\_}
\def\PYGZob{\char`\{}
\def\PYGZcb{\char`\}}
\def\PYGZca{\char`\^}
\def\PYGZam{\char`\&}
\def\PYGZlt{\char`\<}
\def\PYGZgt{\char`\>}
\def\PYGZsh{\char`\#}
\def\PYGZpc{\char`\%}
\def\PYGZdl{\char`\$}
\def\PYGZhy{\char`\-}
\def\PYGZsq{\char`\'}
\def\PYGZdq{\char`\"}
\def\PYGZti{\char`\~}
% for compatibility with earlier versions
\def\PYGZat{@}
\def\PYGZlb{[}
\def\PYGZrb{]}
\makeatother

\renewcommand\PYGZsq{\textquotesingle}

\begin{document}

\maketitle
\tableofcontents
\phantomsection\label{index::doc}


\emph{ClearMap} is a toolbox for the analysis and registration of volumetric data
from cleared tissues.

\emph{ClearMap} has been designed to analyse large 3D image stack datasets obtained with Light Sheet Microscopy
of iDISCO+ cleared mouse brains samples immunolabeled for nuclear proteins. ClearMap performs image registration to the Allen Institute Brain Atlas, volumetric image processing and object detection and statistical analysis. In particular, the tools in \emph{ClearMap} have been written with the mapping of Immediate Early Genes in the brain as the primary application.

However, the tools included with \emph{ClearMap} should also be more broadly useful for data obtained with other types of microscopes, other types of markers, and other clearing techniques. Moreover, the registration and region segmentation capabilities of ClearMap are not depending on the Atlases and annotations we used in our study. Users are free to import their own reference files and annotation files, so the use of \emph{ClearMap} can be easily expanded to other species, other organs or samples.

\emph{ClearMap} is written in Python 2.7, and is designed to take advantage of parallel processing capabilities of modern workstations. We hope the open structure of the code will enable in the future many new modules to be added to ClearMap to broaden the range of applications to different types of biological objects or structures.


\chapter{Using ClearMap}
\label{index:using-clearmap}\label{index:clearmap}

\section{Overview of ClearMap}
\label{introduction::doc}\label{introduction:overview-of-clearmap}
\emph{ClearMap} is a toolbox to analyze and register microscopy images of cleared
tissue. It is targeted towards cleared brain tissue using the {\hyperref[introduction:idisco-clearing-method]{\emph{iDISCO+ Clearing Method}}}
but can be used with any volumetric imaging data. ClearMap contains a large number of functions dedicated to many aspects of 3D image manipulation and object detection, which could open a lot of possibilities for advanced users. For most users however, all relevant functions are explained in the tutorial in the next section, which contains a classic application case for ClearMap.

The ClearMap code package is structured into four main modules:
\begin{itemize}
\item {} 
{\hyperref[introduction:io]{\emph{IO}}} for reading and writing images and data

\item {} 
{\hyperref[introduction:alignment]{\emph{Alignment}}} for resampling, reorientation and registration of images onto references

\item {} 
{\hyperref[introduction:image-processing]{\emph{Image Processing}}} for correcting and quantifying the image data

\item {} 
{\hyperref[introduction:analysis]{\emph{Analysis}}} for the statistical analysis of the data

\end{itemize}


\subsection{IO}
\label{introduction:io}
ClearMap supports a wide range of image formats with a particular focus on volumetric data packaged as stacks or individual files:

\begin{tabulary}{\linewidth}{|L|L|}
\hline
\textsf{\relax 
Format
} & \textsf{\relax 
Description
}\\
\hline
TIF
 & 
tif images and stacks
\\
\hline
RAW / MHD
 & 
raw image files with optional mhd header file
\\
\hline
NRRD
 & 
nearly raw raster data files
\\
\hline
IMS
 & 
Imaris image files
\\
\hline
pattern
 & 
folder, file list or file pattern of a stack of 2d images
\\
\hline\end{tabulary}


We recommend using when possible the mhd format, which is more broadly compatible than tif or nrrd.

\begin{notice}{note}{Note:}
ClearMap can read the image data from a Bitplane’s Imaris, but can’t export image data as an Imaris file.
\end{notice}

Images are represented internally as numpy arrays. ClearMap assumes images
in arrays are arranged as {[}x,y{]}, {[}x,y,z{]} or {[}x,y,z,c{]} where x,y,z correspond to
the x,y,z coordinates as when viewed in an image viewer such as \phantomsection\label{introduction:id1}{\hyperref[introduction:imagej]{\emph{{[}ImageJ{]}}}} and
c to a possible color channel.

ClearMap also supports several data formats for storing data points, such as
cell center coordinates or intensities:

\begin{tabulary}{\linewidth}{|L|L|}
\hline
\textsf{\relax 
Format
} & \textsf{\relax 
Description
}\\
\hline
CSV
 & 
comma separated values in text file, for exporting to other programs
\\
\hline
NPY
 & 
numpy binary file, faster and more compact format for the point data
\\
\hline
VTK
 & 
vtk point data file, for exporting to some programs
\\
\hline
IMS
 & 
Imaris data file, for writing points onto an existing Imaris file
\\
\hline\end{tabulary}


\emph{points} files simply contain all point coordinates arranged as an array of {[}x,y,z{]} coordinates where each line is a detected cell center. \emph{intensities} files are companion to point files (only for csv and npy formats), where each line contains informations about intensity and detected size for the corresponding center in the point file. Each line in the array of the intensities file has 4 rows organized as follows:

\begin{tabulary}{\linewidth}{|L|L|}
\hline
\textsf{\relax 
Row
} & \textsf{\relax 
Description
}\\
\hline
0
 & 
Max intensity of the cell center in the raw data
\\
\hline
1
 & 
Max intensity of the cell center after the DoG filtering.
\\
\hline
2
 & 
Max intensity of the cell center after the background subtraction
\\
\hline
3
 & 
Cell size in voxels after the watershed detection
\\
\hline\end{tabulary}



\subsection{Alignment}
\label{introduction:alignment}
The Alignment module provides tools to resample, reorient and register
volumetric images in a fast parallel way.

Image registration is done by interfacing to the \phantomsection\label{introduction:id2}{\hyperref[introduction:elastix]{\emph{{[}Elastix{]}}}} software package. This package allows to align cleared mouse brains onto the Allan brain atlas \phantomsection\label{introduction:id3}{\hyperref[introduction:aba]{\emph{{[}ABA{]}}}}.


\subsection{Image Processing}
\label{introduction:image-processing}
ClearMap provides a number of image processing tools with a focus on the
processing of large 3D volumetric images in parallel. For the detection of objects in 3D, ClearMap has a modular architecture. For the user, most of this is hidden and called by the \code{SpotDetection} function (see the example script).

The main processing modules include:

\begin{tabulary}{\linewidth}{|L|L|}
\hline
\textsf{\relax 
Module
} & \textsf{\relax 
Description
}\\
\hline
\code{BackgroundRemoval}
 & 
Background estimation and removal via morphological opening
\\
\hline
\code{IlluminationCorrection}
 & 
Correction of vignetting and other illumination errors
\\
\hline
{\hyperref[api/ClearMap.ImageProcessing.Filter:module-ClearMap.ImageProcessing.Filter]{\emph{\code{Filter}}}}
 & 
Filtering of the image via large set of filter kernels
\\
\hline
\code{GreyReconstruction}
 & 
Reconstruction of images
\\
\hline
\code{SpotDetection}
 & 
Detection of local peaks
\\
\hline
\code{CellDetection}
 & 
Detection of cell centers
\\
\hline
\code{CellSizeDetection}
 & 
Detection of cell shapes via watershed
\\
\hline
\code{IlastikClassification}
 & 
Classification of voxels via interface to \phantomsection\label{introduction:id4}{\hyperref[introduction:ilastik]{\emph{{[}Ilastik{]}}}}
\\
\hline\end{tabulary}


The modular structure of this sub-packages allows for fast and flexible integration of
additional modules for future developments.


\subsection{Analysis}
\label{introduction:analysis}
This part of ClearMap provides a toolbox for the statistical analysis and
visualization of detected cells or structures and region specific analysis
of annotated data.

For cleared mouse brains aligned to the \phantomsection\label{introduction:id5}{\hyperref[introduction:aba]{\emph{{[}ABA{]}}}} a wide range of statistical
analysis tools with respect to the annotated brain regions in the atlas is
supported. Two types of analysis are done:
\begin{itemize}
\item {} 
Voxel statistics, which are based on the heat-map generated from the detected cell centers. These are usually represented as image stacks of mean, standard deviation, p-values with False Discovery Rate options.

\item {} 
Region statistics, which are based on the annotated regions from the reference annotation file. They are usually represented as spreadsheets containing the statistics for each region.

\end{itemize}

The Key modules are:

\begin{tabulary}{\linewidth}{|L|L|}
\hline
\textsf{\relax 
Module
} & \textsf{\relax 
Description
}\\
\hline
{\hyperref[api/ClearMap.Analysis:module-ClearMap.Analysis.Statistics]{\emph{\code{Statistics}}}}
 & 
Statistical tools for the analysis of detected cells
\\
\hline
\code{Voxelization}
 & 
For voxel-based statistics: voxelization of cells for visualization and analysis
\\
\hline
{\hyperref[api/ClearMap.Analysis:module-ClearMap.Analysis.Label]{\emph{\code{Label}}}}
 & 
For region-based statistics: tools to analyse data with the annotated reference files
\\
\hline\end{tabulary}


The use of the modules is explained in the tutorial.


\subsection{iDISCO+ Clearing Method}
\label{introduction:idisco-clearing-method}
Robust quantification of 3D datasets requires images as uniform as possible for the signal properties, both on each plane, and also at all imaging depths. The iDISCO+ method is an evolution of the iDISCO whole-mount labeling technique to improve the diffusion and background of staining in large samples \phantomsection\label{introduction:id6}{\hyperref[introduction:renier2014]{\emph{{[}Renier2014{]}}}}, and the 3DISCO clearing technique \phantomsection\label{introduction:id7}{\hyperref[introduction:erturk2012]{\emph{{[}Erturk2012{]}}}}. The iDISCO+ staining and clearing method is combined optimally with the very large field of view enabled by light sheet microscopy, in particular the ultramicroscope optical design, which enables low magnification imaging with high speed and relatively high resolution.
\begin{description}
\item[{The datasets used to develop ClearMap are usually composed of two channels:}] \leavevmode\begin{itemize}
\item {} 
The signal channel. Typically obtained in the far-red light spectrum, where the optical properties of the cleared tissue are at their best for signal-to-noise and transparency. It is recommended when possible to use nuclear reporters or proteins to facilitate the object detection.

\item {} 
The autofluorescence channel, usually collected in the blue-green light spectrum. The background tissue fluorescence highlights the major structures of the tissue to facilitate the 3D image registration. Only the contrast between regions is important here, so it doesn’t matter if the relative intensities between regions are not the same as on the reference scans.

\end{itemize}

\item[{See these videos for example of light sheet imaging of cleared tissues:}] \leavevmode\begin{itemize}
\item {} 
\href{https://www.youtube.com/watch?v=-ctRUMQjizgvbLtLYkW6hI}{Dopaminergic system in the embryonic mouse}

\item {} 
\href{https://www.youtube.com/watch?v=vbLtLYkW6hI}{Cortical and hippocampal neurons in the adult mouse brain}

\end{itemize}

\end{description}

More info can be found on the \phantomsection\label{introduction:id8}{\hyperref[introduction:idisco]{\emph{{[}iDISCO{]}}}} webpage.


\subsection{References}
\label{introduction:references}

\section{Installation}
\label{installation:installation}\label{installation::doc}

\subsection{Requirements}
\label{installation:requirements}
ClearMap is written in Python 2.7. It should run on any Python environment, but it also relies on external softwares such as \href{http://elastix.isi.uu.nl}{Elastix} which may not run optimally on Windows or Apple systems.
For typical use, we recommend a workstation running Ubuntu 14 or later with at least 4 CPU cores, 64Gb of RAM and SSD disks. 128Gb of RAM and 6 cores or above will have much increased performances. The processing time however will depend greatly on the parameters set, so your experience may be different. Also, large hard drives may be needed to host the raw data, although 1 to 4Tb of storage space should be enough for most users.


\subsection{Installation}
\label{installation:id1}
To install ClearMap, first create a folder to contain all the files for the program. Then, open a terminal window, change to the directory hosting ClearMap and download the code from GIT by running this command:

\begin{Verbatim}[commandchars=\\\{\}]
\PYG{g+gp}{\PYGZgt{}\PYGZgt{}\PYGZgt{} }\PYG{n}{git} \PYG{n}{clone} \PYG{n}{https}\PYG{p}{:}\PYG{o}{/}\PYG{o}{/}\PYG{n}{git}\PYG{o}{.}\PYG{n}{assembla}\PYG{o}{.}\PYG{n}{com}\PYG{o}{/}\PYG{n}{idisco}\PYG{o}{.}\PYG{n}{git}
\end{Verbatim}

If you’re starting from scratch, you also need to download individually the following softwares:
\begin{itemize}
\item {} 
To do the alignement, you should download \href{http://elastix.isi.uu.nl}{Elastix} (\href{http://elastix.isi.uu.nl}{http://elastix.isi.uu.nl})

\item {} 
If you wish to use the machine learning filters, download \href{http://ilastik.org}{Ilastik} (\href{http://ilastik.org}{http://ilastik.org}), version 0.5 (this is the version implemented in ClearMap). This is an optional download, only if you wish to use this more complete object detection framework for complex objects.

\end{itemize}

And then the following libraries (most of them available from the software center of Ubuntu, or via the \code{pip} command or \code{apt-get} command in a terminal):
\begin{itemize}
\item {} 
Dev tools for Python 2.7 (from the software center)

\item {} 
Matplotlib (from the software center)

\item {} 
Numpy (from the software center)

\item {} 
Scipy (from the software center)

\item {} 
Skimage (from the software center)

\item {} 
Mahotas (from the developer’s \href{http://luispedro.org/software/mahotas/}{website})

\item {} 
h5py (from the software center) (for Imaris files input/output only)

\item {} 
openCV (from the software center)

\item {} 
MayaVI (from the software center)

\item {} 
libboost-all-dev (from the software center)

\item {} 
PyOpenGL (from the software center)

\item {} 
qimage2ndarray (from the software center)

\item {} 
PyQt4 (from the software center)

\item {} 
tifffile (from the software center)

\item {} 
EVTK (from the developer \href{https://bitbucket.org/pauloh/pyevtk/overview}{website}) (only necessary, for output to vtk files)

\item {} 
libhdf5-dev (from the software center)

\item {} 
Cython (from the software center)

\end{itemize}

If you’re planning to use Ilastik, download \href{http://ukoethe.github.io/vigra/}{Vigra} (from the developer website).

We use \href{https://pythonhosted.org/spyder/}{Spyder} to run the code.


\subsection{Configuration}
\label{installation:configuration}
Open the file ClearMap/Settings.py to set the paths of installations for Ilastik and Elastix:

\begin{Verbatim}[commandchars=\\\{\}]
\PYG{g+gp}{\PYGZgt{}\PYGZgt{}\PYGZgt{} }\PYG{n}{IlastikPath} \PYG{o}{=} \PYG{l+s}{\PYGZsq{}}\PYG{l+s}{/usr/local/ilastik\PYGZhy{}05\PYGZhy{}rc\PYGZhy{}final}\PYG{l+s}{\PYGZsq{}}\PYG{p}{;}
\PYG{g+gp}{\PYGZgt{}\PYGZgt{}\PYGZgt{} }\PYG{n}{ElastixPath} \PYG{o}{=} \PYG{l+s}{\PYGZsq{}}\PYG{l+s}{/usr/local/elastix}\PYG{l+s}{\PYGZsq{}}\PYG{p}{;}
\end{Verbatim}

Note that Ilastik is optional. If you haven’t installed it, you can set the path to \code{None}. You can also set the installation to run on multiple machines by setting a host specific path:

\begin{Verbatim}[commandchars=\\\{\}]
\PYG{g+gp}{\PYGZgt{}\PYGZgt{}\PYGZgt{} }\PYG{k}{if} \PYG{n}{hostname} \PYG{o}{==} \PYG{l+s}{\PYGZsq{}}\PYG{l+s}{kagalaska.nld}\PYG{l+s}{\PYGZsq{}}\PYG{p}{:}  \PYG{c}{\PYGZsh{}Christoph’s Laptop}
\PYG{g+gp}{\PYGZgt{}\PYGZgt{}\PYGZgt{} }        \PYG{n}{IlastikPath} \PYG{o}{=} \PYG{l+s}{\PYGZsq{}}\PYG{l+s}{/home/ckirst/programs/ilastik\PYGZhy{}05/}\PYG{l+s}{\PYGZsq{}}\PYG{p}{;}
\PYG{g+gp}{\PYGZgt{}\PYGZgt{}\PYGZgt{} }        \PYG{n}{ElastixPath} \PYG{o}{=} \PYG{l+s}{\PYGZsq{}}\PYG{l+s}{/home/ckirst/programs/elastix/}\PYG{l+s}{\PYGZsq{}}\PYG{p}{;}
\PYG{g+gp}{\PYGZgt{}\PYGZgt{}\PYGZgt{} }    \PYG{k}{elif} \PYG{n}{hostname} \PYG{o}{==} \PYG{l+s}{\PYGZsq{}}\PYG{l+s}{mtllab\PYGZhy{}Ubuntu}\PYG{l+s}{\PYGZsq{}}\PYG{p}{:} \PYG{c}{\PYGZsh{}Nico’s Workstation}
\PYG{g+gp}{\PYGZgt{}\PYGZgt{}\PYGZgt{} }        \PYG{n}{IlastikPath} \PYG{o}{=} \PYG{l+s}{\PYGZsq{}}\PYG{l+s}{/usr/local/ilastik\PYGZhy{}05\PYGZhy{}rc\PYGZhy{}final}\PYG{l+s}{\PYGZsq{}}\PYG{p}{;}
\PYG{g+gp}{\PYGZgt{}\PYGZgt{}\PYGZgt{} }        \PYG{n}{ElastixPath} \PYG{o}{=} \PYG{l+s}{\PYGZsq{}}\PYG{l+s}{/usr/local/elastix}\PYG{l+s}{\PYGZsq{}}\PYG{p}{;}
\end{Verbatim}


\section{Tutorial}
\label{tutorial::doc}\label{tutorial:tutorial}
The goal of this tutorial is to explain the scripts we used to analyse samples. As an example, we will use a dataset from a Light Sheet imaged adult mouse brain stained for c-fos. The tutorial files also contain an autofluorescence file to enable the registration of the scan to the reference atlas. The tutorial files are found in the ClearMap/Scripts folder. They consist of :
\begin{itemize}
\item {} 
{\hyperref[tutorial:the-parameter-file]{\emph{The Parameter File}}} This sets the parameters individually for each sample

\item {} 
{\hyperref[tutorial:the-run-file]{\emph{The Run File}}} This will run all the commands to process each sample individually

\item {} 
{\hyperref[tutorial:analysis-tools]{\emph{Analysis Tools}}} This scripts will run the analysis tools and group statistics for all samples in the batch

\end{itemize}

A project will usually contain 1 parameter file for each sample, 1 run file for the whole experiment and 1 analysis file for the whole experiment.


\subsection{The Parameter File}
\label{tutorial:the-parameter-file}
The parameter is a Python script that will contain all the necessary informations to process each sample. An example script, \emph{parameter\_file\_template.py} is provided in the ClearMap/Scripts folder. It contains the following sections:

\begin{tabulary}{\linewidth}{|L|L|}
\hline
\textsf{\relax 
Section
} & \textsf{\relax 
Description
}\\
\hline
Import modules
 & 
load from ClearMap the functions used here
\\
\hline
Data parameters
 & 
points to the files used, their resolution and orientation
\\
\hline
Cell detection
 & 
parameters for the cell detection, and module used
\\
\hline
Heat map generation
 & 
to generate a voxelized map of the detected cells
\\
\hline
Config Parameters
 & 
the parameters for memory and processors management
\\
\hline
Run Parameters
 & 
you would usually not change these. They specify how the data will be processed
\\
\hline\end{tabulary}



\subsubsection{Import Modules}
\label{tutorial:import-modules}
You would usually not change these. They are all the functions that will be used later either in the parameter file or in the  execution file.


\subsubsection{Data parameters}
\label{tutorial:data-parameters}
This is where you point to the files used, their resolution and orientation. It also defines which atlas and annotation files to use.

To set the directory where all files will be read and written for this sample:

\begin{Verbatim}[commandchars=\\\{\}]
\PYG{g+gp}{\PYGZgt{}\PYGZgt{}\PYGZgt{} }\PYG{n}{BaseDirectory} \PYG{o}{=} \PYG{l+s}{\PYGZsq{}}\PYG{l+s}{/home/mtllab/exploration/sample1}\PYG{l+s}{\PYGZsq{}}\PYG{p}{;}
\end{Verbatim}

To set the image files used for the processing:

\begin{Verbatim}[commandchars=\\\{\}]
\PYG{g+gp}{\PYGZgt{}\PYGZgt{}\PYGZgt{} }\PYG{n}{cFosFile} \PYG{o}{=} \PYG{n}{os}\PYG{o}{.}\PYG{n}{path}\PYG{o}{.}\PYG{n}{join}\PYG{p}{(}\PYG{n}{BaseDirectory}\PYG{p}{,} \PYG{l+s}{\PYGZsq{}}\PYG{l+s}{cfos/0\PYGZus{}8x\PYGZhy{}cfos\PYGZhy{}Table Z}\PYG{l+s}{\PYGZbs{}}\PYG{l+s}{d\PYGZob{}4\PYGZcb{}.ome.tif}\PYG{l+s}{\PYGZsq{}}\PYG{p}{)}\PYG{p}{;}
\PYG{g+gp}{\PYGZgt{}\PYGZgt{}\PYGZgt{} }\PYG{n}{AutofluoFile} \PYG{o}{=} \PYG{n}{os}\PYG{o}{.}\PYG{n}{path}\PYG{o}{.}\PYG{n}{join}\PYG{p}{(}\PYG{n}{BaseDirectory}\PYG{p}{,} \PYG{l+s}{\PYGZsq{}}\PYG{l+s}{autofluo/0\PYGZus{}8xs3\PYGZhy{}autofluo\PYGZhy{}Table Z}\PYG{l+s}{\PYGZbs{}}\PYG{l+s}{d\PYGZob{}4\PYGZcb{}.ome.tif}\PYG{l+s}{\PYGZsq{}}\PYG{p}{)}\PYG{p}{;}
\end{Verbatim}

Note the use of the command \code{os.path.join} to link the set \code{BaseDirectory} with the folder where the files are. On the LaVision ultramicroscope system, the images files are generated not as stacks, but as numbered files in the ome.tif format. Each Z stack will end by \code{-Table Z0000.ome.tif}. The 0000 is the plane number. To indicate \textbf{ClearMap} to read the next planes, replace the 4 digits with the command \code{\textbackslash{}d\{4\}}. On our system, files for each channel (here, c-fos and background fluorescence) are saved in a different stack, in a different folder.

To restrict the range for the object detection:

\begin{Verbatim}[commandchars=\\\{\}]
\PYG{g+gp}{\PYGZgt{}\PYGZgt{}\PYGZgt{} }\PYG{n}{cFosFileRange} \PYG{o}{=} \PYG{p}{\PYGZob{}}\PYG{l+s}{\PYGZsq{}}\PYG{l+s}{x}\PYG{l+s}{\PYGZsq{}} \PYG{p}{:} \PYG{n+nb}{all}\PYG{p}{,} \PYG{l+s}{\PYGZsq{}}\PYG{l+s}{y}\PYG{l+s}{\PYGZsq{}} \PYG{p}{:} \PYG{p}{(}\PYG{l+m+mi}{180}\PYG{p}{,} \PYG{l+m+mi}{2600}\PYG{p}{)}\PYG{p}{,} \PYG{l+s}{\PYGZsq{}}\PYG{l+s}{z}\PYG{l+s}{\PYGZsq{}} \PYG{p}{:} \PYG{n+nb}{all}\PYG{p}{\PYGZcb{}}\PYG{p}{;}
\end{Verbatim}

This range will only affect the region used for the cell detection. It will not be taken into account for the 3D image registration to the reference Atlas, nor for the voxelization or other analysis. This is useful to limit the amount of memory used. In our example, we use the full x range, the full z range, but restrict the y range. The camera on our system, an Andor sNEO CMOS, has a sensor size of 2160 x 2660. However, the lens used on for the acquisition, an Olympus 2X 0.5NA MVPLAPO, has a strong corner deformation, so we restrict the y range because no cells can be reliably detected outside of this range.

As a reminder, in the image files, the (0, 0, 0) coordinate correspond to the upper left corner of the first plane. To the opposite, the (2160, 2660, 2400) coordinate will be the bottom right corner of the last plane (here 2400, but can vary).

When optimizing the parameters for the object detection, you should dramatically restrict the range to speed up the detection. We recommend using 500 planes, 500 pixels on each side:

\begin{Verbatim}[commandchars=\\\{\}]
\PYG{g+gp}{\PYGZgt{}\PYGZgt{}\PYGZgt{} }\PYG{n}{cFosFileRange} \PYG{o}{=} \PYG{p}{\PYGZob{}}\PYG{l+s}{\PYGZsq{}}\PYG{l+s}{x}\PYG{l+s}{\PYGZsq{}} \PYG{p}{:} \PYG{p}{(}\PYG{l+m+mi}{500}\PYG{p}{,} \PYG{l+m+mi}{1000}\PYG{p}{)}\PYG{p}{,} \PYG{l+s}{\PYGZsq{}}\PYG{l+s}{y}\PYG{l+s}{\PYGZsq{}} \PYG{p}{:} \PYG{p}{(}\PYG{l+m+mi}{500}\PYG{p}{,} \PYG{l+m+mi}{1000}\PYG{p}{)}\PYG{p}{,} \PYG{l+s}{\PYGZsq{}}\PYG{l+s}{z}\PYG{l+s}{\PYGZsq{}} \PYG{p}{:} \PYG{p}{(}\PYG{l+m+mi}{500}\PYG{p}{,} \PYG{l+m+mi}{1000}\PYG{p}{)}\PYG{p}{\PYGZcb{}}\PYG{p}{;}
\end{Verbatim}

But of course adapting the range to where the relevant objects are on your sample.

Next, to set the resolution of the original data (in µm / pixel):

\begin{Verbatim}[commandchars=\\\{\}]
\PYG{g+gp}{\PYGZgt{}\PYGZgt{}\PYGZgt{} }\PYG{n}{OriginalResolution} \PYG{o}{=} \PYG{p}{(}\PYG{l+m+mf}{4.0625}\PYG{p}{,} \PYG{l+m+mf}{4.0625}\PYG{p}{,} \PYG{l+m+mi}{3}\PYG{p}{)}\PYG{p}{;}
\end{Verbatim}

In this example, this is set for a zoom factor of 0.8X on the LaVision system with the 2X lens. This information can be found in the metadata of the tif file usually. If you don’t know the pixel size, we recommend opening the stack with the plugin BioFormat on ImageJ, and go to « image » -\textgreater{} « show info » to read the metadata. On the LaVision file, this information is at the end of the list.

The orientation of the sample has to be set to match the orientation of the Atlas reference files. It is not mandatory to acquire the sample in the same orientation as the atlas. For instance, you can acquire the left side of the brain, and map it onto the right side of the atlas:

\begin{Verbatim}[commandchars=\\\{\}]
\PYG{g+gp}{\PYGZgt{}\PYGZgt{}\PYGZgt{} }\PYG{n}{FinalOrientation} \PYG{o}{=} \PYG{p}{(}\PYG{l+m+mi}{1}\PYG{p}{,}\PYG{l+m+mi}{2}\PYG{p}{,}\PYG{l+m+mi}{3}\PYG{p}{)}\PYG{p}{;}
\end{Verbatim}

The convention is as follow (examples given, any configuration is possible):

\begin{tabulary}{\linewidth}{|L|L|}
\hline
\textsf{\relax 
Value of the tuple
} & \textsf{\relax 
Description
}\\
\hline
(1, 2, 3)
 & 
The scan has the same orientation as the atlas reference
\\
\hline
(-1, 2, 3)
 & 
The x axis is mirrored compared to the atlas
\\
\hline
(1, -2, 3)
 & 
The y axis is mirrored compared to the atlas
\\
\hline
(2, 1, 3)
 & 
Performs a rotation by exchanging the x and y axis
\\
\hline
(3, 2, 1)
 & 
Performs a rotation by exchanging the z and x axis
\\
\hline\end{tabulary}


For our samples, we use the following orientation to match our atlas files:
\begin{itemize}
\item {} 
The right side of the brain is facing the objective, lateral side up.

\item {} 
The rostral side of the brain is up

\item {} 
The dorsal side is facing left

\item {} 
The ventral side is facing right

\end{itemize}

This means that in our scans, if we want to image the right hemisphere, we use (1, 2, 3) and if we want to image the left hemisphere, we use (-1, 2, 3).

To set the output for the voxelized heat map file:

\begin{Verbatim}[commandchars=\\\{\}]
\PYG{g+gp}{\PYGZgt{}\PYGZgt{}\PYGZgt{} }\PYG{n}{VoxelizationFile} \PYG{o}{=} \PYG{n}{os}\PYG{o}{.}\PYG{n}{path}\PYG{o}{.}\PYG{n}{join}\PYG{p}{(}\PYG{n}{BaseDirectory}\PYG{p}{,} \PYG{l+s}{\PYGZsq{}}\PYG{l+s}{points\PYGZus{}voxelized.tif}\PYG{l+s}{\PYGZsq{}}\PYG{p}{)}\PYG{p}{;}
\end{Verbatim}

To set the resolution of the Atlas Files (in um/ pixel):

\begin{Verbatim}[commandchars=\\\{\}]
\PYG{g+gp}{\PYGZgt{}\PYGZgt{}\PYGZgt{} }\PYG{n}{AtlasResolution} \PYG{o}{=} \PYG{p}{(}\PYG{l+m+mi}{25}\PYG{p}{,} \PYG{l+m+mi}{25}\PYG{p}{,} \PYG{l+m+mi}{25}\PYG{p}{)}\PYG{p}{;}
\end{Verbatim}

To choose which atlas files you would like to use:

\begin{Verbatim}[commandchars=\\\{\}]
\PYG{g+gp}{\PYGZgt{}\PYGZgt{}\PYGZgt{} }\PYG{n}{PathReg}        \PYG{o}{=} \PYG{l+s}{\PYGZsq{}}\PYG{l+s}{/home/mtllab/Documents/warping}\PYG{l+s}{\PYGZsq{}}\PYG{p}{;}
\PYG{g+gp}{\PYGZgt{}\PYGZgt{}\PYGZgt{} }\PYG{n}{AtlasFile}      \PYG{o}{=} \PYG{n}{os}\PYG{o}{.}\PYG{n}{path}\PYG{o}{.}\PYG{n}{join}\PYG{p}{(}\PYG{n}{PathReg}\PYG{p}{,} \PYG{l+s}{\PYGZsq{}}\PYG{l+s}{half\PYGZus{}template\PYGZus{}25\PYGZus{}right.tif}\PYG{l+s}{\PYGZsq{}}\PYG{p}{)}\PYG{p}{;}
\PYG{g+gp}{\PYGZgt{}\PYGZgt{}\PYGZgt{} }\PYG{n}{AnnotationFile} \PYG{o}{=} \PYG{n}{os}\PYG{o}{.}\PYG{n}{path}\PYG{o}{.}\PYG{n}{join}\PYG{p}{(}\PYG{n}{PathReg}\PYG{p}{,} \PYG{l+s}{\PYGZsq{}}\PYG{l+s}{annotation\PYGZus{}25\PYGZus{}right.tif}\PYG{l+s}{\PYGZsq{}}\PYG{p}{)}\PYG{p}{;}
\end{Verbatim}

It is important to make sure that the Atlas used is in the correct orientation (see above), but also don’t contain too much information outside of the field of view. While the registration program can deal with a bit of extra « bleed » outside of the sample, this should be kept to a minimum. We usually prepare different crops of the atlas file to match the usual field of views we acquire.


\subsubsection{Cell detection}
\label{tutorial:cell-detection}
At this point, two detection methods exist: the \code{SpotDetection} and \code{Ilastik}:
\begin{itemize}
\item {} 
\code{SpotDetection} is designed for globular objects, such as neuron cell bodies or nuclei. This is the fastest method, and offers a greater degree of fine controls over the sensibility of the detection. However, it is not well suited for complex objects.

\item {} 
\code{Ilastik} is a framework that relies on the user generating a classifier through the graphical interface of the Ilastik program, by painting over a few objects and over the background. The program will then learn to classify the pixels between objects or backgrouns based on the user indications. This is a very easy way to tune very complex filters to detect complex objects or textures. However, the classification is a black box, and very dependent of the user’s classification.

\end{itemize}

In this tutorial, we will use the SpotDetection method. To choose which method to use for the cell detection:

\begin{Verbatim}[commandchars=\\\{\}]
\PYG{g+gp}{\PYGZgt{}\PYGZgt{}\PYGZgt{} }\PYG{n}{ImageProcessingMethod} \PYG{o}{=} \PYG{l+s}{\PYGZdq{}}\PYG{l+s}{SpotDetection}\PYG{l+s}{\PYGZdq{}}\PYG{p}{;}
\end{Verbatim}

The parameters for the Spot Detection methods are then sorted in « dictionaries » by theme :

\begin{tabulary}{\linewidth}{|L|L|}
\hline
\textsf{\relax 
Dictionary name
} & \textsf{\relax 
Description
}\\
\hline
correctIlluminationParameter
 & 
If you have an intensity profile for your microscope, you can correct variations in illuminations here
\\
\hline
removeBackgroundParameter
 & 
To set the background subtraction via morphological opening
\\
\hline
filterDoGParameter
 & 
To set the parameters for the Difference of Gaussian filter
\\
\hline
findExtendedMaximaParameter
 & 
If the object contains multiple peaks of intensity, this will collapse them into one peak
\\
\hline
findIntensityParameter
 & 
Often, the center of the mass of an object is not the voxel of highest intensity. This is a correction for this
\\
\hline
detectCellShapeParameter
 & 
This set the parameters for the cell shape « painting » via the watershed
\\
\hline\end{tabulary}



\paragraph{Correcting the illumination:}
\label{tutorial:correcting-the-illumination}
Because of the Gaussian shape of the light sheet and of the objecting lens vignetting, the sample illumination is not uniform. While correcting the illumination can improve the uniformity of the cell detection, it is usually not really necessary if all samples where imaged the same way, as the same anatomical features will be positioned in the same region of the lens across samples. Nevertheless, to correct for variation in the illumination use:

\begin{Verbatim}[commandchars=\\\{\}]
\PYG{g+gp}{\PYGZgt{}\PYGZgt{}\PYGZgt{} }\PYG{n}{correctIlluminationParameter} \PYG{o}{=} \PYG{p}{\PYGZob{}}
\PYG{g+gp}{\PYGZgt{}\PYGZgt{}\PYGZgt{} }   \PYG{l+s}{\PYGZdq{}}\PYG{l+s}{flatfield}\PYG{l+s}{\PYGZdq{}}  \PYG{p}{:} \PYG{n+nb+bp}{None}\PYG{p}{,}
\PYG{g+gp}{\PYGZgt{}\PYGZgt{}\PYGZgt{} }   \PYG{l+s}{\PYGZdq{}}\PYG{l+s}{background}\PYG{l+s}{\PYGZdq{}} \PYG{p}{:} \PYG{n+nb+bp}{None}\PYG{p}{,}
\PYG{g+gp}{\PYGZgt{}\PYGZgt{}\PYGZgt{} }   \PYG{l+s}{\PYGZdq{}}\PYG{l+s}{scaling}\PYG{l+s}{\PYGZdq{}}    \PYG{p}{:} \PYG{l+s}{\PYGZdq{}}\PYG{l+s}{Mean}\PYG{l+s}{\PYGZdq{}}\PYG{p}{,}
\PYG{g+gp}{\PYGZgt{}\PYGZgt{}\PYGZgt{} }   \PYG{l+s}{\PYGZdq{}}\PYG{l+s}{save}\PYG{l+s}{\PYGZdq{}}       \PYG{p}{:} \PYG{n+nb+bp}{None}\PYG{p}{,}
\PYG{g+gp}{\PYGZgt{}\PYGZgt{}\PYGZgt{} }   \PYG{l+s}{\PYGZdq{}}\PYG{l+s}{verbose}\PYG{l+s}{\PYGZdq{}}    \PYG{p}{:} \PYG{n+nb+bp}{True}
\PYG{g+gp}{\PYGZgt{}\PYGZgt{}\PYGZgt{} }\PYG{p}{\PYGZcb{}}
\end{Verbatim}

For this tutorial, we will not use the correction, so the \code{flatfield} parameter is set to \code{None}. Please note that you need to generate an intensity profile for your system if you wish to use this function.


\paragraph{Background Subtraction:}
\label{tutorial:background-subtraction}
This is the most important pre-treatment step, usually always turned on. The background subtraction via morphological opening is not very sensitive to the size parameter used, as long as it is in the range of the size of the objects detected. The parameters for the background subtraction are as follow:

\begin{Verbatim}[commandchars=\\\{\}]
\PYG{g+gp}{\PYGZgt{}\PYGZgt{}\PYGZgt{} }\PYG{n}{removeBackgroundParameter} \PYG{o}{=} \PYG{p}{\PYGZob{}}
\PYG{g+gp}{\PYGZgt{}\PYGZgt{}\PYGZgt{} }    \PYG{l+s}{\PYGZdq{}}\PYG{l+s}{size}\PYG{l+s}{\PYGZdq{}}    \PYG{p}{:} \PYG{p}{(}\PYG{l+m+mi}{7}\PYG{p}{,}\PYG{l+m+mi}{7}\PYG{p}{)}\PYG{p}{,}
\PYG{g+gp}{\PYGZgt{}\PYGZgt{}\PYGZgt{} }    \PYG{l+s}{\PYGZdq{}}\PYG{l+s}{save}\PYG{l+s}{\PYGZdq{}}    \PYG{p}{:} \PYG{n+nb+bp}{None}\PYG{p}{,}
\PYG{g+gp}{\PYGZgt{}\PYGZgt{}\PYGZgt{} }    \PYG{l+s}{\PYGZdq{}}\PYG{l+s}{verbose}\PYG{l+s}{\PYGZdq{}} \PYG{p}{:} \PYG{n+nb+bp}{True}
\PYG{g+gp}{\PYGZgt{}\PYGZgt{}\PYGZgt{} }\PYG{p}{\PYGZcb{}}
\end{Verbatim}

The parameter \code{size} is a tuple with (x,y) in pixels and correspond to an ellipsoid of this size. Of importance, you can check the result of the background subtraction by setting the \code{save} parameter to a filename. This will output a series of tif images you can open with ImageJ to check the results. For instance you could set \code{save} like this:

\begin{Verbatim}[commandchars=\\\{\}]
\PYG{g+gp}{\PYGZgt{}\PYGZgt{}\PYGZgt{} }    \PYG{l+s}{\PYGZdq{}}\PYG{l+s}{save}\PYG{l+s}{\PYGZdq{}}    \PYG{p}{:} \PYG{n}{os}\PYG{o}{.}\PYG{n}{path}\PYG{o}{.}\PYG{n}{join}\PYG{p}{(}\PYG{n}{BaseDirectory}\PYG{p}{,} \PYG{l+s}{\PYGZsq{}}\PYG{l+s}{background/background}\PYG{l+s}{\PYGZbs{}}\PYG{l+s}{d\PYGZob{}4\PYGZcb{}.ome.tif}\PYG{l+s}{\PYGZsq{}}\PYG{p}{)}\PYG{p}{;}
\end{Verbatim}

You have to use the \code{\textbackslash{}d\{4\}} notation to save the files as a series of images, otherwise only the first plane is saved!

\begin{notice}{note}{Note:}
Only use the \code{save} function when you analyse a small subset of your data, otherwise the full stack will be written to the disk. Don’t forget to turn off this parameter when you’re done optimizing the filters.
\end{notice}


\paragraph{Difference of Gaussians filter:}
\label{tutorial:difference-of-gaussians-filter}\begin{description}
\item[{filterDoGParameter = \{}] \leavevmode
``size''    : None,        \# (tuple or None)      size for the DoG filter if None, do not correct for any background
``sigma''   : None,        \# (tuple or None)      std of outer Guassian, if None autmatically determined from size
``sigma2''  : None,        \# (tuple or None)      std of inner Guassian, if None autmatically determined from size
``save''    : None,        \# (str or None)        file name to save result of this operation if None dont save to file
``verbose'' : True      \# (bool or int)        print / plot information about this step

\end{description}

\}
\begin{description}
\item[{findExtendedMaximaParameter = \{}] \leavevmode
``hMax''      : None,            \# (float or None)     h parameter for the initial h-Max transform, if None, do not perform a h-max transform
``size''      : 5,             \# (tuple)             size for the structure element for the local maxima filter
``threshold'' : 0,        \# (float or None)     include only maxima larger than a threshold, if None keep all localmaxima
``save''      : None,         \# (str or None)       file name to save result of this operation if None dont save to file
``verbose''   : True       \# (bool or int)       print / plot information about this step

\end{description}

\}
\begin{description}
\item[{findIntensityParameter = \{}] \leavevmode
``method'' : `Max',       \# (str, func, None)   method to use to determine intensity (e.g. ``Max'' or ``Mean'') if None take intensities at the given pixels
``size''   : (3,3,3)       \# (tuple)             size of the box on which to perform the \emph{method}

\end{description}

\}
\begin{description}
\item[{detectCellShapeParameter = \{}] \leavevmode
``threshold'' : 700,     \# (float or None)      threshold to determine mask, pixel below this are background if None no mask is generated
``save''      : None,        \# (str or None)        file name to save result of this operation if None dont save to file
``verbose''   : True      \# (bool or int)        print / plot information about this step if None take intensities at the given pixels

\end{description}

\}

\#\# Parameters for cell detection using spot detection algorithm
detectSpotsParameter = \{
\begin{quote}

``correctIlluminationParameter'' : correctIlluminationParameter,
``removeBackgroundParameter''    : removeBackgroundParameter,
``filterDoGParameter''           : filterDoGParameter,
``findExtendedMaximaParameter''  : findExtendedMaximaParameter,
``findIntensityParameter''       : findIntensityParameter,
``detectCellShapeParameter''     : detectCellShapeParameter
\end{quote}

\}


\subsubsection{Heat map generation}
\label{tutorial:heat-map-generation}
To set the output for the voxelized heat map file:

\begin{Verbatim}[commandchars=\\\{\}]
\PYG{g+gp}{\PYGZgt{}\PYGZgt{}\PYGZgt{} }\PYG{n}{VoxelizationFile} \PYG{o}{=} \PYG{n}{os}\PYG{o}{.}\PYG{n}{path}\PYG{o}{.}\PYG{n}{join}\PYG{p}{(}\PYG{n}{BaseDirectory}\PYG{p}{,} \PYG{l+s}{\PYGZsq{}}\PYG{l+s}{points\PYGZus{}voxelized.tif}\PYG{l+s}{\PYGZsq{}}\PYG{p}{)}\PYG{p}{;}
\end{Verbatim}


\subsection{The Run File}
\label{tutorial:the-run-file}

\subsection{Analysis Tools}
\label{tutorial:analysis-tools}

\section{ClearMap Image Analysis Tools}
\label{imageanalysis::doc}\label{imageanalysis:clearmap-image-analysis-tools}
Here we introduce the main image processing steps for the detection of nuclear-located signal with examples.

The data is a small region isolated from an iDISCO+ cleared mouse brain immunostained against c-fos. This small stack is included in the ClearMap package in the Test/Data/ImageAnalysis/ folder:

\begin{Verbatim}[commandchars=\\\{\}]
\PYG{g+gp}{\PYGZgt{}\PYGZgt{}\PYGZgt{} }\PYG{k+kn}{import} \PYG{n+nn}{os}
\PYG{g+gp}{\PYGZgt{}\PYGZgt{}\PYGZgt{} }\PYG{k+kn}{import} \PYG{n+nn}{ClearMap.IO} \PYG{k+kn}{as} \PYG{n+nn}{io}
\PYG{g+gp}{\PYGZgt{}\PYGZgt{}\PYGZgt{} }\PYG{k+kn}{import} \PYG{n+nn}{ClearMap.Settings} \PYG{k+kn}{as} \PYG{n+nn}{settings}
\PYG{g+gp}{\PYGZgt{}\PYGZgt{}\PYGZgt{} }\PYG{n}{filename} \PYG{o}{=} \PYG{n}{os}\PYG{o}{.}\PYG{n}{path}\PYG{o}{.}\PYG{n}{join}\PYG{p}{(}\PYG{n}{settings}\PYG{o}{.}\PYG{n}{ClearMapPath}\PYG{p}{,} \PYG{l+s}{\PYGZsq{}}\PYG{l+s}{Test/Data/ImageAnalysis/cfos\PYGZhy{}substack.tif}\PYG{l+s}{\PYGZsq{}}\PYG{p}{)}\PYG{p}{;}
\end{Verbatim}


\subsection{Visualizing 3D Images}
\label{imageanalysis:visualizing-3d-images}
Large images in 3d are best viewed in specialized software, such as
\href{http://www.bitplane.com/}{Imaris} for 3D rendering or \href{http://imagej.net/Welcome}{ImageJ} to parse the stacks. For the full size data, it is recommended to open the stacks in ImageJ using the « virtual stack » mode.

In ClearMap we provide some basic visualization tools to inspect the 3d data
in the module \code{ClearMap.Visualization.Plot}.

To load them run

\begin{Verbatim}[commandchars=\\\{\}]
\PYG{g+gp}{\PYGZgt{}\PYGZgt{}\PYGZgt{} }\PYG{k+kn}{import} \PYG{n+nn}{ClearMap.Visualization.Plot} \PYG{k+kn}{as} \PYG{n+nn}{plt}
\end{Verbatim}


\subsubsection{Tiled Plots}
\label{imageanalysis:tiled-plots}
In our experience results of 3d image analysis is inspected most accurately
by plotting each horizontal plane in the image in tiles that are coupled for
zooming. Intermediate results from all the steps of the SpotDetection can also be written as image stacks and opened with ImageJ.

\begin{Verbatim}[commandchars=\\\{\}]
\PYG{g+gp}{\PYGZgt{}\PYGZgt{}\PYGZgt{} }\PYG{n}{data} \PYG{o}{=} \PYG{n}{io}\PYG{o}{.}\PYG{n}{readData}\PYG{p}{(}\PYG{n}{filename}\PYG{p}{,} \PYG{n}{z} \PYG{o}{=} \PYG{p}{(}\PYG{l+m+mi}{0}\PYG{p}{,}\PYG{l+m+mi}{26}\PYG{p}{)}\PYG{p}{)}\PYG{p}{;}
\PYG{g+gp}{\PYGZgt{}\PYGZgt{}\PYGZgt{} }\PYG{n}{plt}\PYG{o}{.}\PYG{n}{plotTiling}\PYG{p}{(}\PYG{n}{data}\PYG{p}{)}\PYG{p}{;}
\end{Verbatim}

To only plot a particular subregion its possible to specify the x,y,z range.

\begin{Verbatim}[commandchars=\\\{\}]
\PYG{g+gp}{\PYGZgt{}\PYGZgt{}\PYGZgt{} }\PYG{n}{plt}\PYG{o}{.}\PYG{n}{plotTiling}\PYG{p}{(}\PYG{n}{data}\PYG{p}{,} \PYG{n}{x} \PYG{o}{=} \PYG{p}{(}\PYG{l+m+mi}{0}\PYG{p}{,}\PYG{l+m+mi}{70}\PYG{p}{)}\PYG{p}{,} \PYG{n}{y} \PYG{o}{=} \PYG{p}{(}\PYG{l+m+mi}{0}\PYG{p}{,}\PYG{l+m+mi}{50}\PYG{p}{)}\PYG{p}{,} \PYG{n}{z} \PYG{o}{=} \PYG{p}{(}\PYG{l+m+mi}{10}\PYG{p}{,}\PYG{l+m+mi}{16}\PYG{p}{)}\PYG{p}{)}\PYG{p}{;}
\end{Verbatim}

Sometimes inverse colors may be better:

\begin{Verbatim}[commandchars=\\\{\}]
\PYG{g+gp}{\PYGZgt{}\PYGZgt{}\PYGZgt{} }\PYG{n}{plt}\PYG{o}{.}\PYG{n}{plotTiling}\PYG{p}{(}\PYG{n}{data}\PYG{p}{,} \PYG{n}{inverse} \PYG{o}{=} \PYG{n+nb+bp}{True}\PYG{p}{,}  \PYG{n}{z} \PYG{o}{=} \PYG{p}{(}\PYG{l+m+mi}{10}\PYG{p}{,}\PYG{l+m+mi}{16}\PYG{p}{)}\PYG{p}{)}\PYG{p}{;}
\end{Verbatim}


\subsection{Image Statistics}
\label{imageanalysis:image-statistics}
It is useful to gather some information about the image initially.
For larger images that don’t fit in memory in ClearMap
certain statistics can be gathered in parallel via the
module {\hyperref[api/ClearMap.ImageProcessing:module-ClearMap.ImageProcessing.ImageStatistics]{\emph{\code{ClearMap.ImageProcessing.ImageStatistics}}}} module.

\begin{Verbatim}[commandchars=\\\{\}]
\PYG{g+gp}{\PYGZgt{}\PYGZgt{}\PYGZgt{} }\PYG{k+kn}{import} \PYG{n+nn}{ClearMap.ImageProcessing.ImageStatistics} \PYG{k+kn}{as} \PYG{n+nn}{stat}
\PYG{g+gp}{\PYGZgt{}\PYGZgt{}\PYGZgt{} }\PYG{k}{print} \PYG{n}{stat}\PYG{o}{.}\PYG{n}{calculateStatistics}\PYG{p}{(}\PYG{n}{filename}\PYG{p}{,} \PYG{n}{method} \PYG{o}{=} \PYG{l+s}{\PYGZsq{}}\PYG{l+s}{mean}\PYG{l+s}{\PYGZsq{}}\PYG{p}{)}
\PYG{g+go}{2305.4042155294119}
\end{Verbatim}

To get more information about the progress use the \code{verbose} option

\begin{Verbatim}[commandchars=\\\{\}]
\PYG{g+gp}{\PYGZgt{}\PYGZgt{}\PYGZgt{} }\PYG{k}{print} \PYG{n}{stat}\PYG{o}{.}\PYG{n}{calculateStatistics}\PYG{p}{(}\PYG{n}{filename}\PYG{p}{,} \PYG{n}{method} \PYG{o}{=} \PYG{l+s}{\PYGZsq{}}\PYG{l+s}{mean}\PYG{l+s}{\PYGZsq{}}\PYG{p}{,} \PYG{n}{verbose} \PYG{o}{=} \PYG{n+nb+bp}{True}\PYG{p}{)}
\PYG{g+go}{ChunkSize: Estimated chunk size 51 in 1 chunks!}
\PYG{g+go}{Number of SubStacks: 1}
\PYG{g+go}{Process 0: processing substack 0/1}
\PYG{g+go}{Process 0: file          = /home/ckirst/Science/Projects/BrainActivityMap/Analysis/ClearMap/Test/Data/ImageAnalysis/cfos\PYGZhy{}substack.tif}
\PYG{g+go}{Process 0: segmentation  = \PYGZlt{}function calculateStatisticsOnStack at 0x7fee9c25dd70\PYGZgt{}}
\PYG{g+go}{Process 0: ranges: x,y,z = \PYGZlt{}built\PYGZhy{}in function all\PYGZgt{},\PYGZlt{}built\PYGZhy{}in function all\PYGZgt{},(0, 51)}
\PYG{g+go}{Process 0: Reading data of size (250, 250, 51): elapsed time: 0:00:00}
\PYG{g+go}{Process 0: Processing substack of size (250, 250, 51): elapsed time: 0:00:00}
\PYG{g+go}{Total Time Image Statistics: elapsed time: 0:00:00}
\PYG{g+go}{2305.4042155294119}
\end{Verbatim}

Image statistics can be very helpful for modules, such as Ilastik, requiring a different bit depth than the original 16 or 12 bits files, as it helps to determine how to resample the images to a lower bit depth.


\subsection{Background Removal}
\label{imageanalysis:background-removal}
One of the first steps is often to remove background variations. The
\code{ClearMap.Imageprocessing.BackgroundRemoval} module can be used. It performs the background subtraction by morphological opening. The parameter is set as (x,y) where x and y are the diameter of an ellipsoid of a size close to the typical object detected in pixels. The intensity of the signal is greatly reduced after the filtering, but the signal-to-noise ration is increased:

\begin{Verbatim}[commandchars=\\\{\}]
\PYG{g+gp}{\PYGZgt{}\PYGZgt{}\PYGZgt{} }\PYG{k+kn}{import} \PYG{n+nn}{ClearMap.ImageProcessing.BackgroundRemoval} \PYG{k+kn}{as} \PYG{n+nn}{bgr}
\PYG{g+gp}{\PYGZgt{}\PYGZgt{}\PYGZgt{} }\PYG{n}{dataBGR} \PYG{o}{=} \PYG{n}{bgr}\PYG{o}{.}\PYG{n}{removeBackground}\PYG{p}{(}\PYG{n}{data}\PYG{o}{.}\PYG{n}{asype}\PYG{p}{(}\PYG{l+s}{\PYGZsq{}}\PYG{l+s}{float}\PYG{l+s}{\PYGZsq{}}\PYG{p}{)}\PYG{p}{,} \PYG{n}{size}\PYG{o}{=}\PYG{p}{(}\PYG{l+m+mi}{5}\PYG{p}{,}\PYG{l+m+mi}{5}\PYG{p}{)}\PYG{p}{,} \PYG{n}{verbose} \PYG{o}{=} \PYG{n+nb+bp}{True}\PYG{p}{)}\PYG{p}{;}
\PYG{g+gp}{\PYGZgt{}\PYGZgt{}\PYGZgt{} }\PYG{n}{plt}\PYG{o}{.}\PYG{n}{plotTiling}\PYG{p}{(}\PYG{n}{dataBGR}\PYG{p}{,} \PYG{n}{inverse} \PYG{o}{=} \PYG{n+nb+bp}{True}\PYG{p}{,} \PYG{n}{z} \PYG{o}{=} \PYG{p}{(}\PYG{l+m+mi}{10}\PYG{p}{,}\PYG{l+m+mi}{16}\PYG{p}{)}\PYG{p}{)}\PYG{p}{;}
\end{Verbatim}

Note that if the background feature size is choosen to small, this may result in removal of cells:

\begin{Verbatim}[commandchars=\\\{\}]
\PYG{g+gp}{\PYGZgt{}\PYGZgt{}\PYGZgt{} }\PYG{n}{dataBGR} \PYG{o}{=} \PYG{n}{bgr}\PYG{o}{.}\PYG{n}{removeBackground}\PYG{p}{(}\PYG{n}{data}\PYG{o}{.}\PYG{n}{astype}\PYG{p}{(}\PYG{l+s}{\PYGZsq{}}\PYG{l+s}{float}\PYG{l+s}{\PYGZsq{}}\PYG{p}{)}\PYG{p}{,} \PYG{n}{size}\PYG{o}{=}\PYG{p}{(}\PYG{l+m+mi}{3}\PYG{p}{,}\PYG{l+m+mi}{3}\PYG{p}{)}\PYG{p}{,} \PYG{n}{verbose} \PYG{o}{=} \PYG{n+nb+bp}{True}\PYG{p}{)}\PYG{p}{;}
\PYG{g+gp}{\PYGZgt{}\PYGZgt{}\PYGZgt{} }\PYG{n}{plt}\PYG{o}{.}\PYG{n}{plotTiling}\PYG{p}{(}\PYG{n}{dataBGR}\PYG{p}{,} \PYG{n}{inverse} \PYG{o}{=} \PYG{n+nb+bp}{True}\PYG{p}{,} \PYG{n}{z} \PYG{o}{=} \PYG{p}{(}\PYG{l+m+mi}{10}\PYG{p}{,}\PYG{l+m+mi}{16}\PYG{p}{)}\PYG{p}{)}\PYG{p}{;}
\end{Verbatim}


\subsection{Image Filter}
\label{imageanalysis:image-filter}
A useful feature is to filter an image. Here the
\code{ClearMap.Imageprocessing.Filter} package can be used.

To detect cells center, the difference of Gaussians filter is a powerful way to increase the contrast between the cells and the background. The size is set as (x,y,z), and usually x, y and z are about the typical size in pixels of the detected object. As after the background subtraction, the intensity of the signal is again reduced after the filtering, but the signal-to-noise ration is dramatically increased:

\begin{Verbatim}[commandchars=\\\{\}]
\PYG{g+gp}{\PYGZgt{}\PYGZgt{}\PYGZgt{} }\PYG{k+kn}{from} \PYG{n+nn}{ClearMap.ImageProcessing.Filter.DoGFilter} \PYG{k+kn}{import} \PYG{n}{filterDoG}
\PYG{g+gp}{\PYGZgt{}\PYGZgt{}\PYGZgt{} }\PYG{n}{dataDoG} \PYG{o}{=} \PYG{n}{filterDoG}\PYG{p}{(}\PYG{n}{dataBGR}\PYG{p}{,} \PYG{n}{size}\PYG{o}{=}\PYG{p}{(}\PYG{l+m+mi}{8}\PYG{p}{,}\PYG{l+m+mi}{8}\PYG{p}{,}\PYG{l+m+mi}{4}\PYG{p}{)}\PYG{p}{,} \PYG{n}{verbose} \PYG{o}{=} \PYG{n+nb+bp}{True}\PYG{p}{)}\PYG{p}{;}
\PYG{g+gp}{\PYGZgt{}\PYGZgt{}\PYGZgt{} }\PYG{n}{plt}\PYG{o}{.}\PYG{n}{plotTiling}\PYG{p}{(}\PYG{n}{dataDoG}\PYG{p}{,} \PYG{n}{inverse} \PYG{o}{=} \PYG{n+nb+bp}{True}\PYG{p}{,} \PYG{n}{z} \PYG{o}{=} \PYG{p}{(}\PYG{l+m+mi}{10}\PYG{p}{,}\PYG{l+m+mi}{16}\PYG{p}{)}\PYG{p}{)}\PYG{p}{;}
\end{Verbatim}


\subsection{Maxima Detection}
\label{imageanalysis:maxima-detection}
The \code{ClearMap.ImageProcessing.MaximaDetection} module contains a set
of useful functions for the detection of local maxima.
A labeled image can be visualized using the
\code{ClearMap.Visualization.Plot.plotOverlayLabel()} routine.

\begin{Verbatim}[commandchars=\\\{\}]
\PYG{g+gp}{\PYGZgt{}\PYGZgt{}\PYGZgt{} }\PYG{k+kn}{from} \PYG{n+nn}{ClearMap.ImageProcessing.MaximaDetection} \PYG{k+kn}{import} \PYG{n}{findExtendedMaxima}
\PYG{g+gp}{\PYGZgt{}\PYGZgt{}\PYGZgt{} }\PYG{n}{dataMax} \PYG{o}{=} \PYG{n}{findExtendedMaxima}\PYG{p}{(}\PYG{n}{dataDoG}\PYG{p}{,} \PYG{n}{hMax} \PYG{o}{=} \PYG{n+nb+bp}{None}\PYG{p}{,} \PYG{n}{verbose} \PYG{o}{=} \PYG{n+nb+bp}{True}\PYG{p}{,} \PYG{n}{threshold} \PYG{o}{=} \PYG{l+m+mi}{10}\PYG{p}{)}\PYG{p}{;}
\PYG{g+gp}{\PYGZgt{}\PYGZgt{}\PYGZgt{} }\PYG{n}{plt}\PYG{o}{.}\PYG{n}{plotOverlayLabel}\PYG{p}{(}  \PYG{n}{dataDoG} \PYG{o}{/} \PYG{n}{dataDoG}\PYG{o}{.}\PYG{n}{max}\PYG{p}{(}\PYG{p}{)}\PYG{p}{,} \PYG{n}{dataMax}\PYG{o}{.}\PYG{n}{astype}\PYG{p}{(}\PYG{l+s}{\PYGZsq{}}\PYG{l+s}{int}\PYG{l+s}{\PYGZsq{}}\PYG{p}{)}\PYG{p}{,} \PYG{n}{z} \PYG{o}{=} \PYG{p}{(}\PYG{l+m+mi}{10}\PYG{p}{,}\PYG{l+m+mi}{16}\PYG{p}{)}\PYG{p}{)}
\end{Verbatim}

Its easier to see when zoomed in:

\begin{Verbatim}[commandchars=\\\{\}]
\PYG{g+gp}{\PYGZgt{}\PYGZgt{}\PYGZgt{} }\PYG{n}{plt}\PYG{o}{.}\PYG{n}{plotOverlayLabel}\PYG{p}{(}  \PYG{n}{dataDoG} \PYG{o}{/} \PYG{n}{dataDoG}\PYG{o}{.}\PYG{n}{max}\PYG{p}{(}\PYG{p}{)}\PYG{p}{,} \PYG{n}{dataMax}\PYG{o}{.}\PYG{n}{astype}\PYG{p}{(}\PYG{l+s}{\PYGZsq{}}\PYG{l+s}{int}\PYG{l+s}{\PYGZsq{}}\PYG{p}{)}\PYG{p}{,} \PYG{n}{z} \PYG{o}{=} \PYG{p}{(}\PYG{l+m+mi}{10}\PYG{p}{,}\PYG{l+m+mi}{16}\PYG{p}{)}\PYG{p}{,} \PYG{n}{x} \PYG{o}{=} \PYG{p}{(}\PYG{l+m+mi}{50}\PYG{p}{,}\PYG{l+m+mi}{100}\PYG{p}{)}\PYG{p}{,}\PYG{n}{y} \PYG{o}{=} \PYG{p}{(}\PYG{l+m+mi}{50}\PYG{p}{,}\PYG{l+m+mi}{100}\PYG{p}{)}\PYG{p}{)}
\end{Verbatim}

Note that for some cells, a maxima label in this subset might not be visible as
maxima are detected in the entire image in 3D and the actual maxima
might lie in layers not shown above or below the current planes.

Once the maxima are detected the cell coordinates can be determined:

\begin{Verbatim}[commandchars=\\\{\}]
\PYG{g+gp}{\PYGZgt{}\PYGZgt{}\PYGZgt{} }\PYG{k+kn}{from} \PYG{n+nn}{ClearMap.ImageProcessing.MaximaDetection} \PYG{k+kn}{import} \PYG{n}{findCenterOfMaxima}
\PYG{g+gp}{\PYGZgt{}\PYGZgt{}\PYGZgt{} }\PYG{n}{cells} \PYG{o}{=} \PYG{n}{findCenterOfMaxima}\PYG{p}{(}\PYG{n}{data}\PYG{p}{,} \PYG{n}{dataMax}\PYG{p}{)}\PYG{p}{;}
\PYG{g+gp}{\PYGZgt{}\PYGZgt{}\PYGZgt{} }\PYG{k}{print} \PYG{n}{cells}\PYG{o}{.}\PYG{n}{shape}
\PYG{g+go}{(3670, 3)}
\end{Verbatim}

We can also overlay the cell coordinates in an image:

\begin{Verbatim}[commandchars=\\\{\}]
\PYG{g+gp}{\PYGZgt{}\PYGZgt{}\PYGZgt{} }\PYG{n}{plt}\PYG{o}{.}\PYG{n}{plotOverlayPoints}\PYG{p}{(}\PYG{n}{data}\PYG{p}{,} \PYG{n}{cells}\PYG{p}{,} \PYG{n}{z} \PYG{o}{=} \PYG{p}{(}\PYG{l+m+mi}{10}\PYG{p}{,}\PYG{l+m+mi}{16}\PYG{p}{)}\PYG{p}{)}
\end{Verbatim}


\subsection{Cell Shape Detection}
\label{imageanalysis:cell-shape-detection}
Finally once the cell centers are detected the
\code{ClearMap.ImageProcessing.CellShapedetection} module can be used to detect
the cell shape via a watershed.

\begin{Verbatim}[commandchars=\\\{\}]
\PYG{g+gp}{\PYGZgt{}\PYGZgt{}\PYGZgt{} }\PYG{k+kn}{from} \PYG{n+nn}{ClearMap.ImageProcessing.CellSizeDetection} \PYG{k+kn}{import} \PYG{n}{detectCellShape}
\PYG{g+gp}{\PYGZgt{}\PYGZgt{}\PYGZgt{} }\PYG{n}{dataShape} \PYG{o}{=} \PYG{n}{detectCellShape}\PYG{p}{(}\PYG{n}{dataDoG}\PYG{p}{,} \PYG{n}{cells}\PYG{p}{,} \PYG{n}{threshold} \PYG{o}{=} \PYG{l+m+mi}{15}\PYG{p}{)}\PYG{p}{;}
\PYG{g+gp}{\PYGZgt{}\PYGZgt{}\PYGZgt{} }\PYG{n}{plt}\PYG{o}{.}\PYG{n}{plotOverlayLabel}\PYG{p}{(}\PYG{n}{dataDoG} \PYG{o}{/} \PYG{n}{dataDoG}\PYG{o}{.}\PYG{n}{max}\PYG{p}{(}\PYG{p}{)}\PYG{p}{,} \PYG{n}{dataShape}\PYG{p}{,} \PYG{n}{z} \PYG{o}{=} \PYG{p}{(}\PYG{l+m+mi}{10}\PYG{p}{,}\PYG{l+m+mi}{16}\PYG{p}{)}\PYG{p}{)}
\end{Verbatim}

Now we can perform some measurements:

\begin{Verbatim}[commandchars=\\\{\}]
\PYG{g+gp}{\PYGZgt{}\PYGZgt{}\PYGZgt{} }\PYG{k+kn}{from} \PYG{n+nn}{ClearMap.ImageProcessing.CellSizeDetection} \PYG{k+kn}{import} \PYG{n}{findCellSize}\PYG{p}{,} \PYG{n}{findCellIntensity}
\PYG{g+gp}{\PYGZgt{}\PYGZgt{}\PYGZgt{} }\PYG{n}{cellSizes} \PYG{o}{=} \PYG{n}{findCellSize}\PYG{p}{(}\PYG{n}{dataShape}\PYG{p}{,} \PYG{n}{maxLabel} \PYG{o}{=} \PYG{n}{cells}\PYG{o}{.}\PYG{n}{shape}\PYG{p}{[}\PYG{l+m+mi}{0}\PYG{p}{]}\PYG{p}{)}\PYG{p}{;}
\PYG{g+gp}{\PYGZgt{}\PYGZgt{}\PYGZgt{} }\PYG{n}{cellIntensities} \PYG{o}{=} \PYG{n}{findCellIntensity}\PYG{p}{(}\PYG{n}{dataBGR}\PYG{p}{,} \PYG{n}{dataShape}\PYG{p}{,}  \PYG{n}{maxLabel} \PYG{o}{=} \PYG{n}{cells}\PYG{o}{.}\PYG{n}{shape}\PYG{p}{[}\PYG{l+m+mi}{0}\PYG{p}{]}\PYG{p}{)}\PYG{p}{;}
\end{Verbatim}

and plot those:

\begin{Verbatim}[commandchars=\\\{\}]
\PYG{g+gp}{\PYGZgt{}\PYGZgt{}\PYGZgt{} }\PYG{k+kn}{import} \PYG{n+nn}{matplotlib.pyplot} \PYG{k+kn}{as} \PYG{n+nn}{mpl}
\PYG{g+gp}{\PYGZgt{}\PYGZgt{}\PYGZgt{} }\PYG{n}{mpl}\PYG{o}{.}\PYG{n}{figure}\PYG{p}{(}\PYG{p}{)}
\PYG{g+gp}{\PYGZgt{}\PYGZgt{}\PYGZgt{} }\PYG{n}{mpl}\PYG{o}{.}\PYG{n}{plot}\PYG{p}{(}\PYG{n}{cellSizes}\PYG{p}{,} \PYG{n}{cellIntensities}\PYG{p}{,} \PYG{l+s}{\PYGZsq{}}\PYG{l+s}{.}\PYG{l+s}{\PYGZsq{}}\PYG{p}{)}
\PYG{g+gp}{\PYGZgt{}\PYGZgt{}\PYGZgt{} }\PYG{n}{mpl}\PYG{o}{.}\PYG{n}{xlabel}\PYG{p}{(}\PYG{l+s}{\PYGZsq{}}\PYG{l+s}{cell size [voxel]}\PYG{l+s}{\PYGZsq{}}\PYG{p}{)}
\PYG{g+gp}{\PYGZgt{}\PYGZgt{}\PYGZgt{} }\PYG{n}{mpl}\PYG{o}{.}\PYG{n}{ylabel}\PYG{p}{(}\PYG{l+s}{\PYGZsq{}}\PYG{l+s}{cell intensity [au]}\PYG{l+s}{\PYGZsq{}}\PYG{p}{)}
\end{Verbatim}


\chapter{ClearMap functions}
\label{index:clearmap-functions}

\section{ClearMap package}
\label{api/ClearMap:clearmap-package}\label{api/ClearMap::doc}\phantomsection\label{api/ClearMap:module-ClearMap}\index{ClearMap (module)}
\emph{ClearMap} Registration, Image Analysis and Statistics Library.

\emph{ClearMap} is a python toolbox for the analysis and registration of volumetric
data from cleared tissues.

\emph{ClearMap} is targeted towards large lightsheet volumetric imaging data
of iDISCO+ cleared mouse brains samples, their registration to the Alan brain atlas,
volumetric image processing and statistical analysis.

Author
\begin{quote}

Christoph Kirst,
The Rockefeller University, New York City, 2015
\end{quote}

License
\begin{quote}

GNU GENERAL PUBLIC LICENSE Version 3
\end{quote}


\subsection{Subpackages}
\label{api/ClearMap:subpackages}

\subsubsection{ClearMap.IO package}
\label{api/ClearMap.IO:clearmap-io-package}\label{api/ClearMap.IO::doc}\label{api/ClearMap.IO:module-ClearMap.IO}\index{ClearMap.IO (module)}
This sub-package provides routines to read and write data
\begin{description}
\item[{Two types of data files are discriminated:}] \leavevmode\begin{itemize}
\item {} 
{\hyperref[api/ClearMap.IO:image-data]{\emph{Image data}}}

\item {} 
{\hyperref[api/ClearMap.IO:point-data]{\emph{Point data}}}

\end{itemize}

\end{description}

Image data are files with data from the microscopy or results representing
visualization of the analysis in e.g. volumetric form.

Point data are lists of e.g. cell coordinates or measured
intensitites.


\paragraph{Image data}
\label{api/ClearMap.IO:image-data}
Images are represented internally as numpy arrays. CleaMap assumes images
in arrays are aranged as {[}x,y{]}, {[}x,y,z{]} or {[}x,y,z,c{]} where x,y,z correspond to
the x,y,z coordinates as when viewed in an image viewer such as ImageJ.
The c coordinate is a possible color channel.

\begin{notice}{note}{Note:}
Many image libraries read images as {[}y,x,z{]} or {[}y,x{]} arrays!
\end{notice}

The ClearMap toolbox supports a range of (volumetric) image formats:

\begin{tabulary}{\linewidth}{|L|L|L|}
\hline
\textsf{\relax 
Format
} & \textsf{\relax 
Descrition
} & \textsf{\relax 
Module
}\\
\hline
TIF
 & 
tif images and stacks
 & 
\code{TIF}
\\
\hline
RAW / MHD
 & 
raw image files with optional mhd header file
 & 
\code{RAW}
\\
\hline
NRRD
 & 
nearly raw raster data files
 & 
{\hyperref[api/ClearMap.IO:module-ClearMap.IO.NRRD]{\emph{\code{NRRD}}}}
\\
\hline
IMS
 & 
imaris image file
 & 
\code{Imaris}
\\
\hline
reg exp
 & 
folder, file list or file pattern of a stack of 2d images
 & 
{\hyperref[api/ClearMap.IO:module-ClearMap.IO.FileList]{\emph{\code{FileList}}}}
\\
\hline\end{tabulary}


The image format is inferred automatically from the file name extension.

For example to read image data use {\hyperref[api/ClearMap.IO:ClearMap.IO.IO.readData]{\emph{\code{readData()}}}}:

\begin{Verbatim}[commandchars=\\\{\}]
\PYG{g+gp}{\PYGZgt{}\PYGZgt{}\PYGZgt{} }\PYG{k+kn}{import} \PYG{n+nn}{os}
\PYG{g+gp}{\PYGZgt{}\PYGZgt{}\PYGZgt{} }\PYG{k+kn}{import} \PYG{n+nn}{ClearMap.IO} \PYG{k+kn}{as} \PYG{n+nn}{io}
\PYG{g+gp}{\PYGZgt{}\PYGZgt{}\PYGZgt{} }\PYG{k+kn}{import} \PYG{n+nn}{ClearMap.Settings} \PYG{k+kn}{as} \PYG{n+nn}{settings}
\PYG{g+gp}{\PYGZgt{}\PYGZgt{}\PYGZgt{} }\PYG{n}{filename} \PYG{o}{=} \PYG{n}{os}\PYG{o}{.}\PYG{n}{path}\PYG{o}{.}\PYG{n}{join}\PYG{p}{(}\PYG{n}{settings}\PYG{o}{.}\PYG{n}{ClearMapPath}\PYG{p}{,}\PYG{l+s}{\PYGZsq{}}\PYG{l+s}{Test/Data/Tif/test.tif}\PYG{l+s}{\PYGZsq{}}\PYG{p}{)}\PYG{p}{;}
\PYG{g+gp}{\PYGZgt{}\PYGZgt{}\PYGZgt{} }\PYG{n}{data} \PYG{o}{=} \PYG{n}{io}\PYG{o}{.}\PYG{n}{readData}\PYG{p}{(}\PYG{n}{filename}\PYG{p}{)}\PYG{p}{;}
\PYG{g+gp}{\PYGZgt{}\PYGZgt{}\PYGZgt{} }\PYG{k}{print} \PYG{n}{data}\PYG{o}{.}\PYG{n}{shape}
\PYG{g+go}{(20, 50, 10)}
\end{Verbatim}

To write image data use {\hyperref[api/ClearMap.IO:ClearMap.IO.IO.writeData]{\emph{\code{writeData()}}}}:

\begin{Verbatim}[commandchars=\\\{\}]
\PYG{g+gp}{\PYGZgt{}\PYGZgt{}\PYGZgt{} }\PYG{k+kn}{import} \PYG{n+nn}{os}\PYG{o}{,} \PYG{n+nn}{numpy}
\PYG{g+gp}{\PYGZgt{}\PYGZgt{}\PYGZgt{} }\PYG{k+kn}{import} \PYG{n+nn}{ClearMap.IO} \PYG{k+kn}{as} \PYG{n+nn}{io}
\PYG{g+gp}{\PYGZgt{}\PYGZgt{}\PYGZgt{} }\PYG{k+kn}{import} \PYG{n+nn}{ClearMap.Settings} \PYG{k+kn}{as} \PYG{n+nn}{settings}
\PYG{g+gp}{\PYGZgt{}\PYGZgt{}\PYGZgt{} }\PYG{n}{filename} \PYG{o}{=} \PYG{n}{os}\PYG{o}{.}\PYG{n}{path}\PYG{o}{.}\PYG{n}{join}\PYG{p}{(}\PYG{n}{settings}\PYG{o}{.}\PYG{n}{ClearMapPath}\PYG{p}{,}\PYG{l+s}{\PYGZsq{}}\PYG{l+s}{Test/Data/Tif/test.tif}\PYG{l+s}{\PYGZsq{}}\PYG{p}{)}\PYG{p}{;}
\PYG{g+gp}{\PYGZgt{}\PYGZgt{}\PYGZgt{} }\PYG{n}{data} \PYG{o}{=} \PYG{n}{numpy}\PYG{o}{.}\PYG{n}{random}\PYG{o}{.}\PYG{n}{rand}\PYG{p}{(}\PYG{l+m+mi}{20}\PYG{p}{,}\PYG{l+m+mi}{50}\PYG{p}{,}\PYG{l+m+mi}{10}\PYG{p}{)}\PYG{p}{;}
\PYG{g+gp}{\PYGZgt{}\PYGZgt{}\PYGZgt{} }\PYG{n}{data}\PYG{p}{[}\PYG{l+m+mi}{5}\PYG{p}{:}\PYG{l+m+mi}{15}\PYG{p}{,} \PYG{l+m+mi}{20}\PYG{p}{:}\PYG{l+m+mi}{45}\PYG{p}{,} \PYG{l+m+mi}{2}\PYG{p}{:}\PYG{l+m+mi}{9}\PYG{p}{]} \PYG{o}{=} \PYG{l+m+mi}{0}\PYG{p}{;}
\PYG{g+gp}{\PYGZgt{}\PYGZgt{}\PYGZgt{} }\PYG{n}{data} \PYG{o}{=} \PYG{l+m+mi}{20} \PYG{o}{*} \PYG{n}{data}\PYG{p}{;}
\PYG{g+gp}{\PYGZgt{}\PYGZgt{}\PYGZgt{} }\PYG{n}{data} \PYG{o}{=} \PYG{n}{data}\PYG{o}{.}\PYG{n}{astype}\PYG{p}{(}\PYG{l+s}{\PYGZsq{}}\PYG{l+s}{int32}\PYG{l+s}{\PYGZsq{}}\PYG{p}{)}\PYG{p}{;}
\PYG{g+gp}{\PYGZgt{}\PYGZgt{}\PYGZgt{} }\PYG{n}{res} \PYG{o}{=} \PYG{n}{io}\PYG{o}{.}\PYG{n}{writeData}\PYG{p}{(}\PYG{n}{filename}\PYG{p}{,} \PYG{n}{data}\PYG{p}{)}\PYG{p}{;}
\PYG{g+gp}{\PYGZgt{}\PYGZgt{}\PYGZgt{} }\PYG{k}{print} \PYG{n}{io}\PYG{o}{.}\PYG{n}{dataSize}\PYG{p}{(}\PYG{n}{res}\PYG{p}{)}\PYG{p}{;}
\PYG{g+go}{(20, 50, 10)}
\end{Verbatim}

Generally, the IO module is desinged to work with image sources which can be
either files or already loaded numpy arrays. This is important to enable flexible
parallel processing, without rewriting the data analysis routines.

For example:

\begin{Verbatim}[commandchars=\\\{\}]
\PYG{g+gp}{\PYGZgt{}\PYGZgt{}\PYGZgt{} }\PYG{k+kn}{import} \PYG{n+nn}{numpy}
\PYG{g+gp}{\PYGZgt{}\PYGZgt{}\PYGZgt{} }\PYG{k+kn}{import} \PYG{n+nn}{ClearMap.IO} \PYG{k+kn}{as} \PYG{n+nn}{io}
\PYG{g+gp}{\PYGZgt{}\PYGZgt{}\PYGZgt{} }\PYG{n}{data} \PYG{o}{=} \PYG{n}{numpy}\PYG{o}{.}\PYG{n}{random}\PYG{o}{.}\PYG{n}{rand}\PYG{p}{(}\PYG{l+m+mi}{20}\PYG{p}{,}\PYG{l+m+mi}{50}\PYG{p}{,}\PYG{l+m+mi}{10}\PYG{p}{)}\PYG{p}{;}
\PYG{g+gp}{\PYGZgt{}\PYGZgt{}\PYGZgt{} }\PYG{n}{res} \PYG{o}{=} \PYG{n}{io}\PYG{o}{.}\PYG{n}{writeData}\PYG{p}{(}\PYG{n+nb+bp}{None}\PYG{p}{,} \PYG{n}{data}\PYG{p}{)}\PYG{p}{;}
\PYG{g+gp}{\PYGZgt{}\PYGZgt{}\PYGZgt{} }\PYG{k}{print} \PYG{n}{res}\PYG{o}{.}\PYG{n}{shape}\PYG{p}{;}
\PYG{g+go}{(20, 50, 10)}
\end{Verbatim}

Range parameter can be passed in order to only load sub sets of image data,
usefull when the imags are very large. For example to load a sub-image

\begin{Verbatim}[commandchars=\\\{\}]
\PYG{g+gp}{\PYGZgt{}\PYGZgt{}\PYGZgt{} }\PYG{k+kn}{import} \PYG{n+nn}{os}\PYG{o}{,} \PYG{n+nn}{numpy}
\PYG{g+gp}{\PYGZgt{}\PYGZgt{}\PYGZgt{} }\PYG{k+kn}{import} \PYG{n+nn}{ClearMap.IO} \PYG{k+kn}{as} \PYG{n+nn}{io}
\PYG{g+gp}{\PYGZgt{}\PYGZgt{}\PYGZgt{} }\PYG{k+kn}{import} \PYG{n+nn}{ClearMap.Settings} \PYG{k+kn}{as} \PYG{n+nn}{settings}
\PYG{g+gp}{\PYGZgt{}\PYGZgt{}\PYGZgt{} }\PYG{n}{filename} \PYG{o}{=} \PYG{n}{os}\PYG{o}{.}\PYG{n}{path}\PYG{o}{.}\PYG{n}{join}\PYG{p}{(}\PYG{n}{settings}\PYG{o}{.}\PYG{n}{ClearMapPath}\PYG{p}{,}\PYG{l+s}{\PYGZsq{}}\PYG{l+s}{Test/Data/Tif/test.tif}\PYG{l+s}{\PYGZsq{}}\PYG{p}{)}\PYG{p}{;}
\PYG{g+gp}{\PYGZgt{}\PYGZgt{}\PYGZgt{} }\PYG{n}{res} \PYG{o}{=} \PYG{n}{io}\PYG{o}{.}\PYG{n}{readData}\PYG{p}{(}\PYG{n}{filename}\PYG{p}{,} \PYG{n}{data}\PYG{p}{,} \PYG{n}{x} \PYG{o}{=} \PYG{p}{(}\PYG{l+m+mi}{0}\PYG{p}{,}\PYG{l+m+mi}{3}\PYG{p}{)}\PYG{p}{,} \PYG{n}{y} \PYG{o}{=} \PYG{p}{(}\PYG{l+m+mi}{4}\PYG{p}{,}\PYG{l+m+mi}{6}\PYG{p}{)}\PYG{p}{,} \PYG{n}{z} \PYG{o}{=} \PYG{p}{(}\PYG{l+m+mi}{1}\PYG{p}{,}\PYG{l+m+mi}{4}\PYG{p}{)}\PYG{p}{)}\PYG{p}{;}
\PYG{g+gp}{\PYGZgt{}\PYGZgt{}\PYGZgt{} }\PYG{k}{print} \PYG{n}{res}\PYG{o}{.}\PYG{n}{shape}\PYG{p}{;}
\PYG{g+go}{(3, 2, 3)}
\end{Verbatim}


\paragraph{Point data}
\label{api/ClearMap.IO:point-data}
ClearMap also supports several data formats for storing arrays of points, such
as cell center coordinates or intensities.

Points are assumed to be an array of coordinates where the first array index
is the point number and the second the spatial dimension, i.e. {[}i,d{]}
The spatial dimension can be extended with additional dimensions
for intensity ,easires or other properties.

Points can also be given as tuples (coordinate arrray, property array).

ClearMap supports the following files formats fro point like data:

\begin{tabulary}{\linewidth}{|L|L|L|}
\hline
\textsf{\relax 
Format
} & \textsf{\relax 
Descrition
} & \textsf{\relax 
Module
}\\
\hline
CSV
 & 
comma separated values in text file
 & 
{\hyperref[api/ClearMap.IO:module-ClearMap.IO.CSV]{\emph{\code{CSV}}}}
\\
\hline
NPY
 & 
numpy binary file
 & 
{\hyperref[api/ClearMap.IO:module-ClearMap.IO.NPY]{\emph{\code{NPY}}}}
\\
\hline
VTK
 & 
vtk point data file
 & 
{\hyperref[api/ClearMap.IO:module-ClearMap.IO.VTK]{\emph{\code{VTK}}}}
\\
\hline\end{tabulary}


The point file format is inferred automatically from the file name extension.

For example to read point data use {\hyperref[api/ClearMap.IO:ClearMap.IO.IO.readPoints]{\emph{\code{readPoints()}}}}:

\begin{Verbatim}[commandchars=\\\{\}]
\PYG{g+gp}{\PYGZgt{}\PYGZgt{}\PYGZgt{} }\PYG{k+kn}{import} \PYG{n+nn}{os}
\PYG{g+gp}{\PYGZgt{}\PYGZgt{}\PYGZgt{} }\PYG{k+kn}{import} \PYG{n+nn}{ClearMap.IO} \PYG{k+kn}{as} \PYG{n+nn}{io}
\PYG{g+gp}{\PYGZgt{}\PYGZgt{}\PYGZgt{} }\PYG{k+kn}{import} \PYG{n+nn}{ClearMap.Settings} \PYG{k+kn}{as} \PYG{n+nn}{settings}
\PYG{g+gp}{\PYGZgt{}\PYGZgt{}\PYGZgt{} }\PYG{n}{filename} \PYG{o}{=} \PYG{n}{os}\PYG{o}{.}\PYG{n}{path}\PYG{o}{.}\PYG{n}{join}\PYG{p}{(}\PYG{n}{settings}\PYG{o}{.}\PYG{n}{ClearMapPath}\PYG{p}{,} \PYG{l+s}{\PYGZsq{}}\PYG{l+s}{Test/ImageProcessing/points.txt}\PYG{l+s}{\PYGZsq{}}\PYG{p}{)}\PYG{p}{;}
\PYG{g+gp}{\PYGZgt{}\PYGZgt{}\PYGZgt{} }\PYG{n}{points} \PYG{o}{=} \PYG{n}{io}\PYG{o}{.}\PYG{n}{readPoints}\PYG{p}{(}\PYG{n}{filename}\PYG{p}{)}\PYG{p}{;}
\PYG{g+gp}{\PYGZgt{}\PYGZgt{}\PYGZgt{} }\PYG{k}{print} \PYG{n}{points}\PYG{o}{.}\PYG{n}{shape}
\PYG{g+go}{(5, 3)}
\end{Verbatim}

and to write it use {\hyperref[api/ClearMap.IO:ClearMap.IO.IO.writePoints]{\emph{\code{writePoints()}}}}:

\begin{Verbatim}[commandchars=\\\{\}]
\PYG{g+gp}{\PYGZgt{}\PYGZgt{}\PYGZgt{} }\PYG{k+kn}{import} \PYG{n+nn}{os}\PYG{o}{,} \PYG{n+nn}{numpy}
\PYG{g+gp}{\PYGZgt{}\PYGZgt{}\PYGZgt{} }\PYG{k+kn}{import} \PYG{n+nn}{ClearMap.IO} \PYG{k+kn}{as} \PYG{n+nn}{io}
\PYG{g+gp}{\PYGZgt{}\PYGZgt{}\PYGZgt{} }\PYG{k+kn}{import} \PYG{n+nn}{ClearMap.Settings} \PYG{k+kn}{as} \PYG{n+nn}{settings}
\PYG{g+gp}{\PYGZgt{}\PYGZgt{}\PYGZgt{} }\PYG{n}{filename} \PYG{o}{=} \PYG{n}{os}\PYG{o}{.}\PYG{n}{path}\PYG{o}{.}\PYG{n}{join}\PYG{p}{(}\PYG{n}{settings}\PYG{o}{.}\PYG{n}{ClearMapPath}\PYG{p}{,} \PYG{l+s}{\PYGZsq{}}\PYG{l+s}{Test/ImageProcessing/points.txt}\PYG{l+s}{\PYGZsq{}}\PYG{p}{)}\PYG{p}{;}
\PYG{g+gp}{\PYGZgt{}\PYGZgt{}\PYGZgt{} }\PYG{n}{points} \PYG{o}{=} \PYG{n}{numpy}\PYG{o}{.}\PYG{n}{random}\PYG{o}{.}\PYG{n}{rand}\PYG{p}{(}\PYG{l+m+mi}{5}\PYG{p}{,}\PYG{l+m+mi}{3}\PYG{p}{)}\PYG{p}{;}
\PYG{g+gp}{\PYGZgt{}\PYGZgt{}\PYGZgt{} }\PYG{n}{io}\PYG{o}{.}\PYG{n}{writePoints}\PYG{p}{(}\PYG{n}{filename}\PYG{p}{,} \PYG{n}{points}\PYG{p}{)}\PYG{p}{;}
\end{Verbatim}


\paragraph{Summary}
\label{api/ClearMap.IO:summary}\begin{itemize}
\item {} 
All routines accesing data or data properties accept file name strings or numpy arrays or None

\item {} 
Numerical arrays represent data and point coordinates as {[}x,y,z{]} or {[}x,y{]}

\end{itemize}


\paragraph{ClearMap.IO.IO module}
\label{api/ClearMap.IO:clearmap-io-io-module}\label{api/ClearMap.IO:module-ClearMap.IO.IO}\index{ClearMap.IO.IO (module)}
IO interface to read microscope and point data

Main module to distribute read and writing of individual data formats
to the specialized sub-modules

See {\hyperref[api/ClearMap.IO:module-ClearMap.IO]{\emph{\code{ClearMap.IO}}}} for details.

Author
\begin{quote}

Christoph Kirst, The Rockefeller University, New York City, 2015
\end{quote}
\index{pointFileExtensions (in module ClearMap.IO.IO)}

\begin{fulllineitems}
\phantomsection\label{api/ClearMap.IO:ClearMap.IO.IO.pointFileExtensions}\pysigline{\bfcode{pointFileExtensions}\strong{ = {[}'csv', `txt', `npy', `vtk', `ims'{]}}}
list of extensions supported as a point data file

\end{fulllineitems}

\index{pointFileTypes (in module ClearMap.IO.IO)}

\begin{fulllineitems}
\phantomsection\label{api/ClearMap.IO:ClearMap.IO.IO.pointFileTypes}\pysigline{\bfcode{pointFileTypes}\strong{ = {[}'CSV', `NPY', `VTK', `Imaris'{]}}}
list of point data file types

\end{fulllineitems}

\index{pointFileExtensionToType (in module ClearMap.IO.IO)}

\begin{fulllineitems}
\phantomsection\label{api/ClearMap.IO:ClearMap.IO.IO.pointFileExtensionToType}\pysigline{\bfcode{pointFileExtensionToType}\strong{ = \{`txt': `CSV', `vtk': `VTK', `csv': `CSV', `npy': `NPY', `ims': `Imaris'\}}}
map from point file extensions to point file types

\end{fulllineitems}

\index{dataFileExtensions (in module ClearMap.IO.IO)}

\begin{fulllineitems}
\phantomsection\label{api/ClearMap.IO:ClearMap.IO.IO.dataFileExtensions}\pysigline{\bfcode{dataFileExtensions}\strong{ = {[}'tif', `tiff', `mhd', `raw', `ims', `nrrd'{]}}}
list of extensions supported as a image data file

\end{fulllineitems}

\index{dataFileTypes (in module ClearMap.IO.IO)}

\begin{fulllineitems}
\phantomsection\label{api/ClearMap.IO:ClearMap.IO.IO.dataFileTypes}\pysigline{\bfcode{dataFileTypes}\strong{ = {[}'FileList', `TIF', `RAW', `NRRD', `Imaris'{]}}}
list of image data file types

\end{fulllineitems}

\index{dataFileExtensionToType (in module ClearMap.IO.IO)}

\begin{fulllineitems}
\phantomsection\label{api/ClearMap.IO:ClearMap.IO.IO.dataFileExtensionToType}\pysigline{\bfcode{dataFileExtensionToType}\strong{ = \{`tiff': `TIF', `mhd': `RAW', `nrrd': `NRRD', `raw': `RAW', `ims': `Imaris', `tif': `TIF'\}}}
map from image file extensions to image file types

\end{fulllineitems}

\index{fileExtension() (in module ClearMap.IO.IO)}

\begin{fulllineitems}
\phantomsection\label{api/ClearMap.IO:ClearMap.IO.IO.fileExtension}\pysiglinewithargsret{\bfcode{fileExtension}}{\emph{filename}}{}
Returns file extension if exists
\begin{quote}\begin{description}
\item[{Parameters}] \leavevmode
\textbf{filename} (\emph{str}) --
file name

\item[{Returns}] \leavevmode
\emph{str} --
file extension or None

\end{description}\end{quote}

\end{fulllineitems}

\index{isFile() (in module ClearMap.IO.IO)}

\begin{fulllineitems}
\phantomsection\label{api/ClearMap.IO:ClearMap.IO.IO.isFile}\pysiglinewithargsret{\bfcode{isFile}}{\emph{source}}{}
Checks if filename is a real file, returns false if it is directory or regular expression
\begin{quote}\begin{description}
\item[{Parameters}] \leavevmode
\textbf{source} (\emph{str}) --
source file name

\item[{Returns}] \leavevmode
\emph{bool} --
true if source is a real file

\end{description}\end{quote}

\end{fulllineitems}

\index{isFileExpression() (in module ClearMap.IO.IO)}

\begin{fulllineitems}
\phantomsection\label{api/ClearMap.IO:ClearMap.IO.IO.isFileExpression}\pysiglinewithargsret{\bfcode{isFileExpression}}{\emph{source}}{}
Checks if filename is a regular expression denoting a file list
\begin{quote}\begin{description}
\item[{Parameters}] \leavevmode
\textbf{source} (\emph{str}) --
source file name

\item[{Returns}] \leavevmode
\emph{bool} --
true if source is regular expression with a digit label

\end{description}\end{quote}

\end{fulllineitems}

\index{isPointFile() (in module ClearMap.IO.IO)}

\begin{fulllineitems}
\phantomsection\label{api/ClearMap.IO:ClearMap.IO.IO.isPointFile}\pysiglinewithargsret{\bfcode{isPointFile}}{\emph{source}}{}
Checks if a file is a valid point data file
\begin{quote}\begin{description}
\item[{Parameters}] \leavevmode
\textbf{source} (\emph{str}) --
source file name

\item[{Returns}] \leavevmode
\emph{bool} --
true if source is a point data file

\end{description}\end{quote}

\end{fulllineitems}

\index{isDataFile() (in module ClearMap.IO.IO)}

\begin{fulllineitems}
\phantomsection\label{api/ClearMap.IO:ClearMap.IO.IO.isDataFile}\pysiglinewithargsret{\bfcode{isDataFile}}{\emph{source}}{}
Checks if a file is a valid image data file
\begin{quote}\begin{description}
\item[{Parameters}] \leavevmode
\textbf{source} (\emph{str}) --
source file name

\item[{Returns}] \leavevmode
\emph{bool} --
true if source is an image data file

\end{description}\end{quote}

\end{fulllineitems}

\index{createDirectory() (in module ClearMap.IO.IO)}

\begin{fulllineitems}
\phantomsection\label{api/ClearMap.IO:ClearMap.IO.IO.createDirectory}\pysiglinewithargsret{\bfcode{createDirectory}}{\emph{filename}}{}
Creates the directory of the file if it does not exists
\begin{quote}\begin{description}
\item[{Parameters}] \leavevmode
\textbf{filename} (\emph{str}) --
file name

\item[{Returns}] \leavevmode
\emph{str} --
directory name

\end{description}\end{quote}

\end{fulllineitems}

\index{pointFileNameToType() (in module ClearMap.IO.IO)}

\begin{fulllineitems}
\phantomsection\label{api/ClearMap.IO:ClearMap.IO.IO.pointFileNameToType}\pysiglinewithargsret{\bfcode{pointFileNameToType}}{\emph{filename}}{}
Returns type of a point file
\begin{quote}\begin{description}
\item[{Parameters}] \leavevmode
\textbf{filename} (\emph{str}) --
file name

\item[{Returns}] \leavevmode
\emph{str} --
point data type in {\hyperref[api/ClearMap.IO:ClearMap.IO.IO.pointFileTypes]{\emph{\code{pointFileTypes}}}}

\end{description}\end{quote}

\end{fulllineitems}

\index{dataFileNameToType() (in module ClearMap.IO.IO)}

\begin{fulllineitems}
\phantomsection\label{api/ClearMap.IO:ClearMap.IO.IO.dataFileNameToType}\pysiglinewithargsret{\bfcode{dataFileNameToType}}{\emph{filename}}{}
Returns type of a image data file
\begin{quote}\begin{description}
\item[{Parameters}] \leavevmode
\textbf{filename} (\emph{str}) --
file name

\item[{Returns}] \leavevmode
\emph{str} --
image data type in {\hyperref[api/ClearMap.IO:ClearMap.IO.IO.dataFileTypes]{\emph{\code{dataFileTypes}}}}

\end{description}\end{quote}

\end{fulllineitems}

\index{dataFileNameToModule() (in module ClearMap.IO.IO)}

\begin{fulllineitems}
\phantomsection\label{api/ClearMap.IO:ClearMap.IO.IO.dataFileNameToModule}\pysiglinewithargsret{\bfcode{dataFileNameToModule}}{\emph{filename}}{}
Return the module that handles io for a data file
\begin{quote}\begin{description}
\item[{Parameters}] \leavevmode
\textbf{filename} (\emph{str}) --
file name

\item[{Returns}] \leavevmode
\emph{object} --
sub-module that handles a specific data type

\end{description}\end{quote}

\end{fulllineitems}

\index{pointFileNameToModule() (in module ClearMap.IO.IO)}

\begin{fulllineitems}
\phantomsection\label{api/ClearMap.IO:ClearMap.IO.IO.pointFileNameToModule}\pysiglinewithargsret{\bfcode{pointFileNameToModule}}{\emph{filename}}{}
Return the module that handles io for a point file
\begin{quote}\begin{description}
\item[{Parameters}] \leavevmode
\textbf{filename} (\emph{str}) --
file name

\item[{Returns}] \leavevmode
\emph{object} --
sub-module that handles a specific point file type

\end{description}\end{quote}

\end{fulllineitems}

\index{dataSize() (in module ClearMap.IO.IO)}

\begin{fulllineitems}
\phantomsection\label{api/ClearMap.IO:ClearMap.IO.IO.dataSize}\pysiglinewithargsret{\bfcode{dataSize}}{\emph{source}, \emph{x=\textless{}built-in function all\textgreater{}}, \emph{y=\textless{}built-in function all\textgreater{}}, \emph{z=\textless{}built-in function all\textgreater{}}, \emph{**args}}{}
Returns array size of the image data needed when read from file and reduced to specified ranges
\begin{quote}\begin{description}
\item[{Parameters}] \leavevmode\begin{itemize}
\item {} 
\textbf{source} (\emph{array or str}) --
source data

\item {} 
\textbf{x,y,z} (\emph{tuple or all}) --
range specifications, \code{all} is full range

\end{itemize}

\item[{Returns}] \leavevmode
\emph{tuple} --
size of the image data after reading and range reduction

\end{description}\end{quote}

\end{fulllineitems}

\index{dataZSize() (in module ClearMap.IO.IO)}

\begin{fulllineitems}
\phantomsection\label{api/ClearMap.IO:ClearMap.IO.IO.dataZSize}\pysiglinewithargsret{\bfcode{dataZSize}}{\emph{source}, \emph{z=\textless{}built-in function all\textgreater{}}, \emph{**args}}{}
Returns size of the array in the third dimension, None if 2D data
\begin{quote}\begin{description}
\item[{Parameters}] \leavevmode\begin{itemize}
\item {} 
\textbf{source} (\emph{array or str}) --
source data

\item {} 
\textbf{z} (\emph{tuple or all}) --
z-range specification, \code{all} is full range

\end{itemize}

\item[{Returns}] \leavevmode
\emph{int} --
size of the image data in z after reading and range reduction

\end{description}\end{quote}

\end{fulllineitems}

\index{toDataRange() (in module ClearMap.IO.IO)}

\begin{fulllineitems}
\phantomsection\label{api/ClearMap.IO:ClearMap.IO.IO.toDataRange}\pysiglinewithargsret{\bfcode{toDataRange}}{\emph{size}, \emph{r=\textless{}built-in function all\textgreater{}}}{}
Converts range r to numeric range (min,max) given the full array size
\begin{quote}\begin{description}
\item[{Parameters}] \leavevmode\begin{itemize}
\item {} 
\textbf{size} (\emph{tuple}) --
source data size

\item {} 
\textbf{r} (\emph{tuple or all}) --
range specification, \code{all} is full range

\end{itemize}

\item[{Returns}] \leavevmode
\emph{tuple} --
absolute range as pair of integers

\end{description}\end{quote}


\strong{See also:}


{\hyperref[api/ClearMap.IO:ClearMap.IO.IO.toDataSize]{\emph{\code{toDataSize()}}}}, {\hyperref[api/ClearMap.IO:ClearMap.IO.IO.dataSizeFromDataRange]{\emph{\code{dataSizeFromDataRange()}}}}



\end{fulllineitems}

\index{toDataSize() (in module ClearMap.IO.IO)}

\begin{fulllineitems}
\phantomsection\label{api/ClearMap.IO:ClearMap.IO.IO.toDataSize}\pysiglinewithargsret{\bfcode{toDataSize}}{\emph{size}, \emph{r=\textless{}built-in function all\textgreater{}}}{}
Converts full size to actual size given range r
\begin{quote}\begin{description}
\item[{Parameters}] \leavevmode\begin{itemize}
\item {} 
\textbf{size} (\emph{tuple}) --
data size

\item {} 
\textbf{r} (\emph{tuple or all}) --
range specification, \code{all} is full range

\end{itemize}

\item[{Returns}] \leavevmode
\emph{int} --
data size

\end{description}\end{quote}


\strong{See also:}


{\hyperref[api/ClearMap.IO:ClearMap.IO.IO.toDataRange]{\emph{\code{toDataRange()}}}}, {\hyperref[api/ClearMap.IO:ClearMap.IO.IO.dataSizeFromDataRange]{\emph{\code{dataSizeFromDataRange()}}}}



\end{fulllineitems}

\index{dataSizeFromDataRange() (in module ClearMap.IO.IO)}

\begin{fulllineitems}
\phantomsection\label{api/ClearMap.IO:ClearMap.IO.IO.dataSizeFromDataRange}\pysiglinewithargsret{\bfcode{dataSizeFromDataRange}}{\emph{dataSize}, \emph{x=\textless{}built-in function all\textgreater{}}, \emph{y=\textless{}built-in function all\textgreater{}}, \emph{z=\textless{}built-in function all\textgreater{}}, \emph{**args}}{}
Converts full data size to actual size given ranges for x,y,z
\begin{quote}\begin{description}
\item[{Parameters}] \leavevmode\begin{itemize}
\item {} 
\textbf{dataSize} (\emph{tuple}) --
data size

\item {} 
\textbf{x,z,y} (\emph{tuple or all}) --
range specifications, \code{all} is full range

\end{itemize}

\item[{Returns}] \leavevmode
\emph{tuple} --
data size as tuple of integers

\end{description}\end{quote}


\strong{See also:}


{\hyperref[api/ClearMap.IO:ClearMap.IO.IO.toDataRange]{\emph{\code{toDataRange()}}}}, {\hyperref[api/ClearMap.IO:ClearMap.IO.IO.toDataSize]{\emph{\code{toDataSize()}}}}



\end{fulllineitems}

\index{dataToRange() (in module ClearMap.IO.IO)}

\begin{fulllineitems}
\phantomsection\label{api/ClearMap.IO:ClearMap.IO.IO.dataToRange}\pysiglinewithargsret{\bfcode{dataToRange}}{\emph{data}, \emph{x=\textless{}built-in function all\textgreater{}}, \emph{y=\textless{}built-in function all\textgreater{}}, \emph{z=\textless{}built-in function all\textgreater{}}, \emph{**args}}{}
Reduces data to specified ranges
\begin{quote}\begin{description}
\item[{Parameters}] \leavevmode\begin{itemize}
\item {} 
\textbf{data} (\emph{array}) --
full data array

\item {} 
\textbf{x,z,y} (\emph{tuple or all}) --
range specifications, \code{all} is full range

\end{itemize}

\item[{Returns}] \leavevmode
\emph{array} --
reduced data

\end{description}\end{quote}


\strong{See also:}


{\hyperref[api/ClearMap.IO:ClearMap.IO.IO.dataSizeFromDataRange]{\emph{\code{dataSizeFromDataRange()}}}}



\end{fulllineitems}

\index{readData() (in module ClearMap.IO.IO)}

\begin{fulllineitems}
\phantomsection\label{api/ClearMap.IO:ClearMap.IO.IO.readData}\pysiglinewithargsret{\bfcode{readData}}{\emph{source}, \emph{**args}}{}
Read data from one of the supported formats
\begin{quote}\begin{description}
\item[{Parameters}] \leavevmode\begin{itemize}
\item {} 
\textbf{source} (\emph{str, array or None}) --
full data array, if numpy array simply reduce its range

\item {} 
\textbf{x,z,y} (\emph{tuple or all}) --
range specifications, \code{all} is full range

\item {} 
\textbf{**args} --
further arguments specific to image data format reader

\end{itemize}

\item[{Returns}] \leavevmode
\emph{array} --
data as numpy array

\end{description}\end{quote}


\strong{See also:}


{\hyperref[api/ClearMap.IO:ClearMap.IO.IO.writeData]{\emph{\code{writeData()}}}}



\end{fulllineitems}

\index{writeData() (in module ClearMap.IO.IO)}

\begin{fulllineitems}
\phantomsection\label{api/ClearMap.IO:ClearMap.IO.IO.writeData}\pysiglinewithargsret{\bfcode{writeData}}{\emph{sink}, \emph{data}, \emph{**args}}{}
Write data to one of the supported formats
\begin{quote}\begin{description}
\item[{Parameters}] \leavevmode\begin{itemize}
\item {} 
\textbf{sink} (\emph{str, array or None}) --
the destination for the data, if None the data is returned directly

\item {} 
\textbf{data} (\emph{array or None}) --
data to be written

\item {} 
\textbf{**args} --
further arguments specific to image data format writer

\end{itemize}

\item[{Returns}] \leavevmode
\emph{array, str or None} --
data or file name of the written data

\end{description}\end{quote}


\strong{See also:}


{\hyperref[api/ClearMap.IO:ClearMap.IO.IO.readData]{\emph{\code{readData()}}}}



\end{fulllineitems}

\index{copyFile() (in module ClearMap.IO.IO)}

\begin{fulllineitems}
\phantomsection\label{api/ClearMap.IO:ClearMap.IO.IO.copyFile}\pysiglinewithargsret{\bfcode{copyFile}}{\emph{source}, \emph{sink}}{}
Copy a file from source to sink
\begin{quote}\begin{description}
\item[{Parameters}] \leavevmode\begin{itemize}
\item {} 
\textbf{source} (\emph{str}) --
file name of source

\item {} 
\textbf{sink} (\emph{str}) --
file name of sink

\end{itemize}

\item[{Returns}] \leavevmode
\emph{str} --
name of the copied file

\end{description}\end{quote}


\strong{See also:}


{\hyperref[api/ClearMap.IO:ClearMap.IO.IO.copyData]{\emph{\code{copyData()}}}}, {\hyperref[api/ClearMap.IO:ClearMap.IO.IO.convertData]{\emph{\code{convertData()}}}}



\end{fulllineitems}

\index{copyData() (in module ClearMap.IO.IO)}

\begin{fulllineitems}
\phantomsection\label{api/ClearMap.IO:ClearMap.IO.IO.copyData}\pysiglinewithargsret{\bfcode{copyData}}{\emph{source}, \emph{sink}}{}
Copy a data file from source to sink, which can consist of multiple files
\begin{quote}\begin{description}
\item[{Parameters}] \leavevmode\begin{itemize}
\item {} 
\textbf{source} (\emph{str}) --
file name of source

\item {} 
\textbf{sink} (\emph{str}) --
file name of sink

\end{itemize}

\item[{Returns}] \leavevmode
\emph{str} --
name of the copied file

\end{description}\end{quote}


\strong{See also:}


{\hyperref[api/ClearMap.IO:ClearMap.IO.IO.copyFile]{\emph{\code{copyFile()}}}}, {\hyperref[api/ClearMap.IO:ClearMap.IO.IO.convertData]{\emph{\code{convertData()}}}}



\end{fulllineitems}

\index{convertData() (in module ClearMap.IO.IO)}

\begin{fulllineitems}
\phantomsection\label{api/ClearMap.IO:ClearMap.IO.IO.convertData}\pysiglinewithargsret{\bfcode{convertData}}{\emph{source}, \emph{sink}, \emph{**args}}{}
Transforms data from source format to sink format
\begin{quote}\begin{description}
\item[{Parameters}] \leavevmode\begin{itemize}
\item {} 
\textbf{source} (\emph{str}) --
file name of source

\item {} 
\textbf{sink} (\emph{str}) --
file name of sink

\end{itemize}

\item[{Returns}] \leavevmode
\emph{str} --
name of the copied file

\end{description}\end{quote}

\begin{notice}{warning}{Warning:}
Not optimized for large image data sets
\end{notice}


\strong{See also:}


{\hyperref[api/ClearMap.IO:ClearMap.IO.IO.copyFile]{\emph{\code{copyFile()}}}}, {\hyperref[api/ClearMap.IO:ClearMap.IO.IO.copyData]{\emph{\code{copyData()}}}}



\end{fulllineitems}

\index{toMultiChannelData() (in module ClearMap.IO.IO)}

\begin{fulllineitems}
\phantomsection\label{api/ClearMap.IO:ClearMap.IO.IO.toMultiChannelData}\pysiglinewithargsret{\bfcode{toMultiChannelData}}{\emph{*args}}{}
Concatenate single channel arrays to one multi channel array
\begin{quote}\begin{description}
\item[{Parameters}] \leavevmode
\textbf{*args} (\emph{arrays}) --
arrays to be concatenated

\item[{Returns}] \leavevmode
\emph{array} --
concatenated multi-channel array

\end{description}\end{quote}

\end{fulllineitems}

\index{pointsToCoordinates() (in module ClearMap.IO.IO)}

\begin{fulllineitems}
\phantomsection\label{api/ClearMap.IO:ClearMap.IO.IO.pointsToCoordinates}\pysiglinewithargsret{\bfcode{pointsToCoordinates}}{\emph{points}}{}
Converts a (coordiantes, properties) tuple to the coordinates only
\begin{quote}\begin{description}
\item[{Parameters}] \leavevmode
\textbf{points} (\emph{array or tuple}) --
point data to be reduced to coordinates

\item[{Returns}] \leavevmode
\emph{array} --
coordiante data

\end{description}\end{quote}
\paragraph{Notes}

Todo: Move this to a class that handles points and their meta data

\end{fulllineitems}

\index{pointsToProperties() (in module ClearMap.IO.IO)}

\begin{fulllineitems}
\phantomsection\label{api/ClearMap.IO:ClearMap.IO.IO.pointsToProperties}\pysiglinewithargsret{\bfcode{pointsToProperties}}{\emph{points}}{}
Converts a (coordiante, properties) tuple to the properties only
\begin{quote}\begin{description}
\item[{Parameters}] \leavevmode
\textbf{points} (\emph{array or tuple}) --
point data to be reduced to properties

\item[{Returns}] \leavevmode
\emph{array} --
property data

\end{description}\end{quote}
\paragraph{Notes}

Todo: Move this to a class that handles points and their meta data

\end{fulllineitems}

\index{pointsToCoordinatesAndProperties() (in module ClearMap.IO.IO)}

\begin{fulllineitems}
\phantomsection\label{api/ClearMap.IO:ClearMap.IO.IO.pointsToCoordinatesAndProperties}\pysiglinewithargsret{\bfcode{pointsToCoordinatesAndProperties}}{\emph{points}}{}
Converts points in various formats to a (coordinates, properties) tuple
\begin{quote}\begin{description}
\item[{Parameters}] \leavevmode
\textbf{points} (\emph{array or tuple}) --
point data to be converted to (coordinates, properties) tuple

\item[{Returns}] \leavevmode
\emph{tuple} --
(coordinates, properties) tuple

\end{description}\end{quote}
\paragraph{Notes}

Todo: Move this to a class that handles points and their meta data

\end{fulllineitems}

\index{pointsToCoordinatesAndPropertiesFileNames() (in module ClearMap.IO.IO)}

\begin{fulllineitems}
\phantomsection\label{api/ClearMap.IO:ClearMap.IO.IO.pointsToCoordinatesAndPropertiesFileNames}\pysiglinewithargsret{\bfcode{pointsToCoordinatesAndPropertiesFileNames}}{\emph{filename}, \emph{propertiesPostfix='\_intensities'}, \emph{**args}}{}
Generates a tuple of filenames to store coordinates and properties data separately
\begin{quote}\begin{description}
\item[{Parameters}] \leavevmode\begin{itemize}
\item {} 
\textbf{filename} (\emph{str}) --
point data file name

\item {} 
\textbf{propertiesPostfix} (\emph{str}) --
postfix on file name to indicate property data

\end{itemize}

\item[{Returns}] \leavevmode
\emph{tuple} --
(file name, file name for properties)

\end{description}\end{quote}
\paragraph{Notes}

Todo: Move this to a class that handles points and their meta data

\end{fulllineitems}

\index{pointShiftFromRange() (in module ClearMap.IO.IO)}

\begin{fulllineitems}
\phantomsection\label{api/ClearMap.IO:ClearMap.IO.IO.pointShiftFromRange}\pysiglinewithargsret{\bfcode{pointShiftFromRange}}{\emph{dataSize}, \emph{x=\textless{}built-in function all\textgreater{}}, \emph{y=\textless{}built-in function all\textgreater{}}, \emph{z=\textless{}built-in function all\textgreater{}}, \emph{**args}}{}
Calculate shift of points given a specific range restriction
\begin{quote}\begin{description}
\item[{Parameters}] \leavevmode\begin{itemize}
\item {} 
\textbf{dataSize} (\emph{str}) --
data size of the full image

\item {} 
\textbf{x,y,z} (\emph{tuples or all}) --
range specifications

\end{itemize}

\item[{Returns}] \leavevmode
\emph{tuple} --
shift of points from original origin of data to origin of range reduced data

\end{description}\end{quote}

\end{fulllineitems}

\index{pointsToRange() (in module ClearMap.IO.IO)}

\begin{fulllineitems}
\phantomsection\label{api/ClearMap.IO:ClearMap.IO.IO.pointsToRange}\pysiglinewithargsret{\bfcode{pointsToRange}}{\emph{points}, \emph{dataSize=\textless{}built-in function all\textgreater{}}, \emph{x=\textless{}built-in function all\textgreater{}}, \emph{y=\textless{}built-in function all\textgreater{}}, \emph{z=\textless{}built-in function all\textgreater{}}, \emph{shift=False}, \emph{**args}}{}
Restrict points to a specific range
\begin{quote}\begin{description}
\item[{Parameters}] \leavevmode\begin{itemize}
\item {} 
\textbf{points} (\emph{array or str}) --
point source

\item {} 
\textbf{dataSize} (\emph{str}) --
data size of the full image

\item {} 
\textbf{x,y,z} (\emph{tuples or all}) --
range specifications

\item {} 
\textbf{shift} (\emph{bool}) --
shift points to relative coordinates in the reduced image

\end{itemize}

\item[{Returns}] \leavevmode
\emph{tuple} --
points reduced in range and optionally shifted to the range reduced origin

\end{description}\end{quote}

\end{fulllineitems}

\index{readPoints() (in module ClearMap.IO.IO)}

\begin{fulllineitems}
\phantomsection\label{api/ClearMap.IO:ClearMap.IO.IO.readPoints}\pysiglinewithargsret{\bfcode{readPoints}}{\emph{source}, \emph{**args}}{}
Read a list of points from csv or vtk
\begin{quote}\begin{description}
\item[{Parameters}] \leavevmode\begin{itemize}
\item {} 
\textbf{source} (\emph{str, array, tuple or None}) --
the data source file

\item {} 
\textbf{**args} --
further arguments specific to point data format reader

\end{itemize}

\item[{Returns}] \leavevmode
\emph{array or tuple or None} --
point data of source

\end{description}\end{quote}


\strong{See also:}


{\hyperref[api/ClearMap.IO:ClearMap.IO.IO.writePoints]{\emph{\code{writePoints()}}}}



\end{fulllineitems}

\index{writePoints() (in module ClearMap.IO.IO)}

\begin{fulllineitems}
\phantomsection\label{api/ClearMap.IO:ClearMap.IO.IO.writePoints}\pysiglinewithargsret{\bfcode{writePoints}}{\emph{sink}, \emph{points}, \emph{**args}}{}
Write a list of points to csv, vtk or ims files
\begin{quote}\begin{description}
\item[{Parameters}] \leavevmode\begin{itemize}
\item {} 
\textbf{sink} (\emph{str or None}) --
the destination for the point data

\item {} 
\textbf{points} (\emph{array or tuple or None}) --
the point data, optionally as (coordinates, properties) tuple

\item {} 
\textbf{**args} --
further arguments specific to point data format writer

\end{itemize}

\item[{Returns}] \leavevmode
\emph{str or array or tuple or None} --
point data of source

\end{description}\end{quote}


\strong{See also:}


{\hyperref[api/ClearMap.IO:ClearMap.IO.IO.readPoints]{\emph{\code{readPoints()}}}}



\end{fulllineitems}

\index{writeTable() (in module ClearMap.IO.IO)}

\begin{fulllineitems}
\phantomsection\label{api/ClearMap.IO:ClearMap.IO.IO.writeTable}\pysiglinewithargsret{\bfcode{writeTable}}{\emph{filename}, \emph{table}}{}
Writes a numpy array with column names to a csv file.
\begin{quote}\begin{description}
\item[{Parameters}] \leavevmode\begin{itemize}
\item {} 
\textbf{filename} (\emph{str}) --
filename to save table to

\item {} 
\textbf{table} (\emph{annotated array}) --
table to write to file

\end{itemize}

\item[{Returns}] \leavevmode
\emph{str} --
file name

\end{description}\end{quote}

\end{fulllineitems}



\paragraph{ClearMap.IO.CSV module}
\label{api/ClearMap.IO:clearmap-io-csv-module}\label{api/ClearMap.IO:module-ClearMap.IO.CSV}\index{ClearMap.IO.CSV (module)}
Interface to write csv files of cell coordinates / intensities

The module utilizes the csv file writer/reader from numpy.
\paragraph{Example}

\begin{Verbatim}[commandchars=\\\{\}]
\PYG{g+gp}{\PYGZgt{}\PYGZgt{}\PYGZgt{} }\PYG{k+kn}{import} \PYG{n+nn}{os}\PYG{o}{,} \PYG{n+nn}{numpy}
\PYG{g+gp}{\PYGZgt{}\PYGZgt{}\PYGZgt{} }\PYG{k+kn}{import} \PYG{n+nn}{ClearMap.IO.CSV} \PYG{k+kn}{as} \PYG{n+nn}{csv}
\PYG{g+gp}{\PYGZgt{}\PYGZgt{}\PYGZgt{} }\PYG{k+kn}{import} \PYG{n+nn}{ClearMap.Settings} \PYG{k+kn}{as} \PYG{n+nn}{settings}
\PYG{g+gp}{\PYGZgt{}\PYGZgt{}\PYGZgt{} }\PYG{n}{filename} \PYG{o}{=} \PYG{n}{os}\PYG{o}{.}\PYG{n}{path}\PYG{o}{.}\PYG{n}{join}\PYG{p}{(}\PYG{n}{settings}\PYG{o}{.}\PYG{n}{ClearMapPath}\PYG{p}{,} \PYG{l+s}{\PYGZsq{}}\PYG{l+s}{Test/ImageProcessing/points.txt}\PYG{l+s}{\PYGZsq{}}\PYG{p}{)}\PYG{p}{;}
\PYG{g+gp}{\PYGZgt{}\PYGZgt{}\PYGZgt{} }\PYG{n}{points} \PYG{o}{=} \PYG{n}{numpy}\PYG{o}{.}\PYG{n}{random}\PYG{o}{.}\PYG{n}{rand}\PYG{p}{(}\PYG{l+m+mi}{5}\PYG{p}{,}\PYG{l+m+mi}{3}\PYG{p}{)}\PYG{p}{;}
\PYG{g+gp}{\PYGZgt{}\PYGZgt{}\PYGZgt{} }\PYG{n}{csv}\PYG{o}{.}\PYG{n}{writePoints}\PYG{p}{(}\PYG{n}{filename}\PYG{p}{,} \PYG{n}{points}\PYG{p}{)}\PYG{p}{;}
\PYG{g+gp}{\PYGZgt{}\PYGZgt{}\PYGZgt{} }\PYG{n}{points2} \PYG{o}{=} \PYG{n}{csv}\PYG{o}{.}\PYG{n}{readPoints}\PYG{p}{(}\PYG{n}{filename}\PYG{p}{)}\PYG{p}{;}
\PYG{g+gp}{\PYGZgt{}\PYGZgt{}\PYGZgt{} }\PYG{k}{print} \PYG{n}{points2}\PYG{o}{.}\PYG{n}{shape}
\PYG{g+go}{(5, 3)}
\end{Verbatim}
\index{writePoints() (in module ClearMap.IO.CSV)}

\begin{fulllineitems}
\phantomsection\label{api/ClearMap.IO:ClearMap.IO.CSV.writePoints}\pysiglinewithargsret{\bfcode{writePoints}}{\emph{filename}, \emph{points}, \emph{**args}}{}
Write point data to csv file
\begin{quote}\begin{description}
\item[{Parameters}] \leavevmode\begin{itemize}
\item {} 
\textbf{filename} (\emph{str}) --
file name

\item {} 
\textbf{points} (\emph{array}) --
point data

\end{itemize}

\item[{Returns}] \leavevmode
\emph{str} --
file name

\end{description}\end{quote}

\end{fulllineitems}

\index{readPoints() (in module ClearMap.IO.CSV)}

\begin{fulllineitems}
\phantomsection\label{api/ClearMap.IO:ClearMap.IO.CSV.readPoints}\pysiglinewithargsret{\bfcode{readPoints}}{\emph{filename}, \emph{**args}}{}
Read point data to csv file
\begin{quote}\begin{description}
\item[{Parameters}] \leavevmode\begin{itemize}
\item {} 
\textbf{filename} (\emph{str}) --
file name

\item {} 
\textbf{**args} --
arguments for \code{pointsToRange()}

\end{itemize}

\item[{Returns}] \leavevmode
\emph{str} --
file name

\end{description}\end{quote}

\end{fulllineitems}

\index{test() (in module ClearMap.IO.CSV)}

\begin{fulllineitems}
\phantomsection\label{api/ClearMap.IO:ClearMap.IO.CSV.test}\pysiglinewithargsret{\bfcode{test}}{}{}
Test CSV module

\end{fulllineitems}



\paragraph{ClearMap.IO.FileList module}
\label{api/ClearMap.IO:module-ClearMap.IO.FileList}\label{api/ClearMap.IO:clearmap-io-filelist-module}\index{ClearMap.IO.FileList (module)}
Interface to read/write image stacks saved as a list of files

The filename is given as regular expression as described
\href{https://docs.python.org/2/library/re.html}{here}.

It is assumd that there is a single digit like regular expression in the file
name, i.e. \code{\textbackslash{}d\{4\}} indicates a placeholder for an integer with four digits using traling 0s
and \code{\textbackslash{}d\{\}} would jus asume an integer with variable size.

For example: \code{/test\textbackslash{}d\{4\}.tif} or  \code{/test\textbackslash{}d\{\}.tif}
\paragraph{Examples}

\begin{Verbatim}[commandchars=\\\{\}]
\PYG{g+gp}{\PYGZgt{}\PYGZgt{}\PYGZgt{} }\PYG{k+kn}{import} \PYG{n+nn}{os}\PYG{o}{,} \PYG{n+nn}{numpy}
\PYG{g+gp}{\PYGZgt{}\PYGZgt{}\PYGZgt{} }\PYG{k+kn}{import} \PYG{n+nn}{ClearMap.Settings} \PYG{k+kn}{as} \PYG{n+nn}{settings}
\PYG{g+gp}{\PYGZgt{}\PYGZgt{}\PYGZgt{} }\PYG{k+kn}{import} \PYG{n+nn}{ClearMap.IO.FileList} \PYG{k+kn}{as} \PYG{n+nn}{fl}
\PYG{g+gp}{\PYGZgt{}\PYGZgt{}\PYGZgt{} }\PYG{n}{filename} \PYG{o}{=} \PYG{n}{os}\PYG{o}{.}\PYG{n}{path}\PYG{o}{.}\PYG{n}{join}\PYG{p}{(}\PYG{n}{settings}\PYG{o}{.}\PYG{n}{ClearMapPath}\PYG{p}{,} \PYG{l+s}{\PYGZsq{}}\PYG{l+s}{Test/Data/FileList/test}\PYG{l+s}{\PYGZbs{}}\PYG{l+s}{d\PYGZob{}4\PYGZcb{}.tif}\PYG{l+s}{\PYGZsq{}}\PYG{p}{)}
\PYG{g+gp}{\PYGZgt{}\PYGZgt{}\PYGZgt{} }\PYG{n}{data} \PYG{o}{=} \PYG{n}{numpy}\PYG{o}{.}\PYG{n}{random}\PYG{o}{.}\PYG{n}{rand}\PYG{p}{(}\PYG{l+m+mi}{20}\PYG{p}{,}\PYG{l+m+mi}{50}\PYG{p}{,}\PYG{l+m+mi}{10}\PYG{p}{)}\PYG{p}{;}
\PYG{g+gp}{\PYGZgt{}\PYGZgt{}\PYGZgt{} }\PYG{n}{data} \PYG{o}{=} \PYG{n}{data}\PYG{o}{.}\PYG{n}{astype}\PYG{p}{(}\PYG{l+s}{\PYGZsq{}}\PYG{l+s}{int32}\PYG{l+s}{\PYGZsq{}}\PYG{p}{)}\PYG{p}{;}
\PYG{g+gp}{\PYGZgt{}\PYGZgt{}\PYGZgt{} }\PYG{n}{fl}\PYG{o}{.}\PYG{n}{writeData}\PYG{p}{(}\PYG{n}{filename}\PYG{p}{,} \PYG{n}{data}\PYG{p}{)}\PYG{p}{;}
\PYG{g+gp}{\PYGZgt{}\PYGZgt{}\PYGZgt{} }\PYG{n}{img} \PYG{o}{=} \PYG{n}{fl}\PYG{o}{.}\PYG{n}{readData}\PYG{p}{(}\PYG{n}{filename}\PYG{p}{)}\PYG{p}{;}
\PYG{g+gp}{\PYGZgt{}\PYGZgt{}\PYGZgt{} }\PYG{k}{print} \PYG{n}{img}\PYG{o}{.}\PYG{n}{shape}
\PYG{g+go}{(20, 50, 10)}
\end{Verbatim}
\index{readFileList() (in module ClearMap.IO.FileList)}

\begin{fulllineitems}
\phantomsection\label{api/ClearMap.IO:ClearMap.IO.FileList.readFileList}\pysiglinewithargsret{\bfcode{readFileList}}{\emph{filename}}{}
Returns list of files that match the regular expression
\begin{quote}\begin{description}
\item[{Parameters}] \leavevmode
\textbf{filename} (\emph{str}) --
file name as regular expression

\item[{Returns}] \leavevmode
\emph{str, list} --
path of files, file names that match the regular expression

\end{description}\end{quote}

\end{fulllineitems}

\index{splitFileExpression() (in module ClearMap.IO.FileList)}

\begin{fulllineitems}
\phantomsection\label{api/ClearMap.IO:ClearMap.IO.FileList.splitFileExpression}\pysiglinewithargsret{\bfcode{splitFileExpression}}{\emph{filename}}{}
Split the regular expression at the digit place holder
\begin{quote}\begin{description}
\item[{Parameters}] \leavevmode
\textbf{filename} (\emph{str}) --
file name as regular expression

\item[{Returns}] \leavevmode
\emph{tuple} --
file header, file extension, digit format

\end{description}\end{quote}

\end{fulllineitems}

\index{fileExperssionToFileName() (in module ClearMap.IO.FileList)}

\begin{fulllineitems}
\phantomsection\label{api/ClearMap.IO:ClearMap.IO.FileList.fileExperssionToFileName}\pysiglinewithargsret{\bfcode{fileExperssionToFileName}}{\emph{filename}, \emph{z}}{}
Insert a number into the regular expression
\begin{quote}\begin{description}
\item[{Parameters}] \leavevmode\begin{itemize}
\item {} 
\textbf{filename} (\emph{str}) --
file name as regular expression

\item {} 
\textbf{z} (\emph{int}) --
z slice index

\end{itemize}

\item[{Returns}] \leavevmode
\emph{str} --
file name

\end{description}\end{quote}

\end{fulllineitems}

\index{dataSize() (in module ClearMap.IO.FileList)}

\begin{fulllineitems}
\phantomsection\label{api/ClearMap.IO:ClearMap.IO.FileList.dataSize}\pysiglinewithargsret{\bfcode{dataSize}}{\emph{filename}, \emph{**args}}{}
Returns size of data stored as a file list
\begin{quote}\begin{description}
\item[{Parameters}] \leavevmode\begin{itemize}
\item {} 
\textbf{filename} (\emph{str}) --
file name as regular expression

\item {} 
\textbf{x,y,z} (\emph{tuple}) --
data range specifications

\end{itemize}

\item[{Returns}] \leavevmode
\emph{tuple} --
data size

\end{description}\end{quote}

\end{fulllineitems}

\index{dataZSize() (in module ClearMap.IO.FileList)}

\begin{fulllineitems}
\phantomsection\label{api/ClearMap.IO:ClearMap.IO.FileList.dataZSize}\pysiglinewithargsret{\bfcode{dataZSize}}{\emph{filename}, \emph{z=\textless{}built-in function all\textgreater{}}, \emph{**args}}{}
Returns size of data stored as a file list
\begin{quote}\begin{description}
\item[{Parameters}] \leavevmode\begin{itemize}
\item {} 
\textbf{filename} (\emph{str}) --
file name as regular expression

\item {} 
\textbf{z} (\emph{tuple}) --
z data range specification

\end{itemize}

\item[{Returns}] \leavevmode
\emph{int} --
z data size

\end{description}\end{quote}

\end{fulllineitems}

\index{readDataFiles() (in module ClearMap.IO.FileList)}

\begin{fulllineitems}
\phantomsection\label{api/ClearMap.IO:ClearMap.IO.FileList.readDataFiles}\pysiglinewithargsret{\bfcode{readDataFiles}}{\emph{filename}, \emph{x=\textless{}built-in function all\textgreater{}}, \emph{y=\textless{}built-in function all\textgreater{}}, \emph{z=\textless{}built-in function all\textgreater{}}, \emph{**args}}{}
Read data from individual images assuming they are the z slices
\begin{quote}\begin{description}
\item[{Parameters}] \leavevmode\begin{itemize}
\item {} 
\textbf{filename} (\emph{str}) --
file name as regular expression

\item {} 
\textbf{x,y,z} (\emph{tuple}) --
data range specifications

\end{itemize}

\item[{Returns}] \leavevmode
\emph{array} --
image data

\end{description}\end{quote}

\end{fulllineitems}

\index{readData() (in module ClearMap.IO.FileList)}

\begin{fulllineitems}
\phantomsection\label{api/ClearMap.IO:ClearMap.IO.FileList.readData}\pysiglinewithargsret{\bfcode{readData}}{\emph{filename}, \emph{**args}}{}
Read image stack from single or multiple images
\begin{quote}\begin{description}
\item[{Parameters}] \leavevmode\begin{itemize}
\item {} 
\textbf{filename} (\emph{str}) --
file name as regular expression

\item {} 
\textbf{x,y,z} (\emph{tuple}) --
data range specifications

\end{itemize}

\item[{Returns}] \leavevmode
\emph{array} --
image data

\end{description}\end{quote}

\end{fulllineitems}

\index{writeData() (in module ClearMap.IO.FileList)}

\begin{fulllineitems}
\phantomsection\label{api/ClearMap.IO:ClearMap.IO.FileList.writeData}\pysiglinewithargsret{\bfcode{writeData}}{\emph{filename}, \emph{data}, \emph{startIndex=0}}{}
Write image stack to single or multiple image files
\begin{quote}\begin{description}
\item[{Parameters}] \leavevmode\begin{itemize}
\item {} 
\textbf{filename} (\emph{str}) --
file name as regular expression

\item {} 
\textbf{data} (\emph{array}) --
image data

\item {} 
\textbf{startIndex} (\emph{int}) --
index of first z-slice

\end{itemize}

\item[{Returns}] \leavevmode
\emph{str} --
file name as regular expression

\end{description}\end{quote}

\end{fulllineitems}

\index{copyData() (in module ClearMap.IO.FileList)}

\begin{fulllineitems}
\phantomsection\label{api/ClearMap.IO:ClearMap.IO.FileList.copyData}\pysiglinewithargsret{\bfcode{copyData}}{\emph{source}, \emph{sink}}{}
Copy a data file from source to sink when for entire list of files
\begin{quote}\begin{description}
\item[{Parameters}] \leavevmode\begin{itemize}
\item {} 
\textbf{source} (\emph{str}) --
file name pattern of source

\item {} 
\textbf{sink} (\emph{str}) --
file name pattern of sink

\end{itemize}

\item[{Returns}] \leavevmode
\emph{str} --
file name patttern of the copy

\end{description}\end{quote}

\end{fulllineitems}

\index{test() (in module ClearMap.IO.FileList)}

\begin{fulllineitems}
\phantomsection\label{api/ClearMap.IO:ClearMap.IO.FileList.test}\pysiglinewithargsret{\bfcode{test}}{}{}
Test FileList module

\end{fulllineitems}



\paragraph{ClearMap.IO.Imaris module}
\label{api/ClearMap.IO:clearmap-io-imaris-module}

\paragraph{ClearMap.IO.NPY module}
\label{api/ClearMap.IO:clearmap-io-npy-module}\label{api/ClearMap.IO:module-ClearMap.IO.NPY}\index{ClearMap.IO.NPY (module)}
Interface to write binary files for point like data

The interface is based on the numpy library.
\paragraph{Example}

\begin{Verbatim}[commandchars=\\\{\}]
\PYG{g+gp}{\PYGZgt{}\PYGZgt{}\PYGZgt{} }\PYG{k+kn}{import} \PYG{n+nn}{os}\PYG{o}{,} \PYG{n+nn}{numpy}
\PYG{g+gp}{\PYGZgt{}\PYGZgt{}\PYGZgt{} }\PYG{k+kn}{import} \PYG{n+nn}{ClearMap.Settings} \PYG{k+kn}{as} \PYG{n+nn}{settings}
\PYG{g+gp}{\PYGZgt{}\PYGZgt{}\PYGZgt{} }\PYG{k+kn}{import} \PYG{n+nn}{ClearMap.IO.NPY} \PYG{k+kn}{as} \PYG{n+nn}{npy}
\PYG{g+gp}{\PYGZgt{}\PYGZgt{}\PYGZgt{} }\PYG{n}{filename} \PYG{o}{=} \PYG{n}{os}\PYG{o}{.}\PYG{n}{path}\PYG{o}{.}\PYG{n}{join}\PYG{p}{(}\PYG{n}{settings}\PYG{o}{.}\PYG{n}{ClearMapPath}\PYG{p}{,} \PYG{l+s}{\PYGZsq{}}\PYG{l+s}{Test/Data/NPY/points.npy}\PYG{l+s}{\PYGZsq{}}\PYG{p}{)}\PYG{p}{;}
\PYG{g+gp}{\PYGZgt{}\PYGZgt{}\PYGZgt{} }\PYG{n}{points} \PYG{o}{=} \PYG{n}{npy}\PYG{o}{.}\PYG{n}{readPoints}\PYG{p}{(}\PYG{n}{filename}\PYG{p}{)}\PYG{p}{;}
\PYG{g+gp}{\PYGZgt{}\PYGZgt{}\PYGZgt{} }\PYG{k}{print} \PYG{n}{points}\PYG{o}{.}\PYG{n}{shape}
\PYG{g+go}{(5, 3)}
\end{Verbatim}

Author
\begin{quote}

Christoph Kirst, The Rockefeller University, New York City, 2015
\end{quote}
\index{writePoints() (in module ClearMap.IO.NPY)}

\begin{fulllineitems}
\phantomsection\label{api/ClearMap.IO:ClearMap.IO.NPY.writePoints}\pysiglinewithargsret{\bfcode{writePoints}}{\emph{filename}, \emph{points}, \emph{**args}}{}
\end{fulllineitems}

\index{readPoints() (in module ClearMap.IO.NPY)}

\begin{fulllineitems}
\phantomsection\label{api/ClearMap.IO:ClearMap.IO.NPY.readPoints}\pysiglinewithargsret{\bfcode{readPoints}}{\emph{filename}, \emph{**args}}{}
\end{fulllineitems}

\index{test() (in module ClearMap.IO.NPY)}

\begin{fulllineitems}
\phantomsection\label{api/ClearMap.IO:ClearMap.IO.NPY.test}\pysiglinewithargsret{\bfcode{test}}{}{}
Test NPY module

\end{fulllineitems}



\paragraph{ClearMap.IO.NRRD module}
\label{api/ClearMap.IO:clearmap-io-nrrd-module}\label{api/ClearMap.IO:module-ClearMap.IO.NRRD}\index{ClearMap.IO.NRRD (module)}
Interface to NRRD volumetric image data files.

The interface is based on nrrd.py, an all-python (and numpy)
implementation for reading and writing nrrd files.
See \href{http://teem.sourceforge.net/nrrd/format.html}{http://teem.sourceforge.net/nrrd/format.html} for the specification.
\paragraph{Example}

\begin{Verbatim}[commandchars=\\\{\}]
\PYG{g+gp}{\PYGZgt{}\PYGZgt{}\PYGZgt{} }\PYG{k+kn}{import} \PYG{n+nn}{os}\PYG{o}{,} \PYG{n+nn}{numpy}
\PYG{g+gp}{\PYGZgt{}\PYGZgt{}\PYGZgt{} }\PYG{k+kn}{import} \PYG{n+nn}{ClearMap.Settings} \PYG{k+kn}{as} \PYG{n+nn}{settings}
\PYG{g+gp}{\PYGZgt{}\PYGZgt{}\PYGZgt{} }\PYG{k+kn}{import} \PYG{n+nn}{ClearMap.IO.NRRD} \PYG{k+kn}{as} \PYG{n+nn}{nrrd}
\PYG{g+gp}{\PYGZgt{}\PYGZgt{}\PYGZgt{} }\PYG{n}{filename} \PYG{o}{=} \PYG{n}{os}\PYG{o}{.}\PYG{n}{path}\PYG{o}{.}\PYG{n}{join}\PYG{p}{(}\PYG{n}{settings}\PYG{o}{.}\PYG{n}{ClearMapPath}\PYG{p}{,} \PYG{l+s}{\PYGZsq{}}\PYG{l+s}{Test/Data/Nrrd/test.nrrd}\PYG{l+s}{\PYGZsq{}}\PYG{p}{)}\PYG{p}{;}
\PYG{g+gp}{\PYGZgt{}\PYGZgt{}\PYGZgt{} }\PYG{n}{data} \PYG{o}{=} \PYG{n}{nrrd}\PYG{o}{.}\PYG{n}{readData}\PYG{p}{(}\PYG{n}{filename}\PYG{p}{)}\PYG{p}{;}
\PYG{g+gp}{\PYGZgt{}\PYGZgt{}\PYGZgt{} }\PYG{k}{print} \PYG{n}{data}\PYG{o}{.}\PYG{n}{shape}
\PYG{g+go}{(20, 50, 10)}
\end{Verbatim}

Author
\begin{quote}

Copyright 2011 Maarten Everts and David Hammond.

Modified to integrate into ClearMap framework:
- Christoph Kirst, The Rockefeller University, New York City, 2015
\end{quote}
\index{NrrdError}

\begin{fulllineitems}
\phantomsection\label{api/ClearMap.IO:ClearMap.IO.NRRD.NrrdError}\pysigline{\strong{exception }\bfcode{NrrdError}}
Bases: \code{exceptions.Exception}

Exceptions for Nrrd class.

\end{fulllineitems}

\index{parse\_nrrdvector() (in module ClearMap.IO.NRRD)}

\begin{fulllineitems}
\phantomsection\label{api/ClearMap.IO:ClearMap.IO.NRRD.parse_nrrdvector}\pysiglinewithargsret{\bfcode{parse\_nrrdvector}}{\emph{inp}}{}
Parse a vector from a nrrd header, return a list.

\end{fulllineitems}

\index{parse\_optional\_nrrdvector() (in module ClearMap.IO.NRRD)}

\begin{fulllineitems}
\phantomsection\label{api/ClearMap.IO:ClearMap.IO.NRRD.parse_optional_nrrdvector}\pysiglinewithargsret{\bfcode{parse\_optional\_nrrdvector}}{\emph{inp}}{}
Parse a vector from a nrrd header that can also be none.

\end{fulllineitems}

\index{readHeader() (in module ClearMap.IO.NRRD)}

\begin{fulllineitems}
\phantomsection\label{api/ClearMap.IO:ClearMap.IO.NRRD.readHeader}\pysiglinewithargsret{\bfcode{readHeader}}{\emph{filename}}{}
Parse the fields in the nrrd header

nrrdfile can be any object which supports the iterator protocol and
returns a string each time its next() method is called — file objects and
list objects are both suitable. If csvfile is a file object, it must be
opened with the ‘b’ flag on platforms where that makes a difference
(e.g. Windows)

\begin{Verbatim}[commandchars=\\\{\}]
\PYG{g+gp}{\PYGZgt{}\PYGZgt{}\PYGZgt{} }\PYG{n}{readHeader}\PYG{p}{(}\PYG{p}{(}\PYG{l+s}{\PYGZdq{}}\PYG{l+s}{NRRD0005}\PYG{l+s}{\PYGZdq{}}\PYG{p}{,} \PYG{l+s}{\PYGZdq{}}\PYG{l+s}{type: float}\PYG{l+s}{\PYGZdq{}}\PYG{p}{,} \PYG{l+s}{\PYGZdq{}}\PYG{l+s}{dimension: 3}\PYG{l+s}{\PYGZdq{}}\PYG{p}{)}\PYG{p}{)}
\PYG{g+go}{\PYGZob{}\PYGZsq{}type\PYGZsq{}: \PYGZsq{}float\PYGZsq{}, \PYGZsq{}dimension\PYGZsq{}: 3, \PYGZsq{}keyvaluepairs\PYGZsq{}: \PYGZob{}\PYGZcb{}\PYGZcb{}}
\PYG{g+gp}{\PYGZgt{}\PYGZgt{}\PYGZgt{} }\PYG{n}{readHeader}\PYG{p}{(}\PYG{p}{(}\PYG{l+s}{\PYGZdq{}}\PYG{l+s}{NRRD0005}\PYG{l+s}{\PYGZdq{}}\PYG{p}{,} \PYG{l+s}{\PYGZdq{}}\PYG{l+s}{my extra info:=my : colon\PYGZhy{}separated : values}\PYG{l+s}{\PYGZdq{}}\PYG{p}{)}\PYG{p}{)}
\PYG{g+go}{\PYGZob{}\PYGZsq{}keyvaluepairs\PYGZsq{}: \PYGZob{}\PYGZsq{}my extra info\PYGZsq{}: \PYGZsq{}my : colon\PYGZhy{}separated : values\PYGZsq{}\PYGZcb{}\PYGZcb{}}
\end{Verbatim}

\end{fulllineitems}

\index{readData() (in module ClearMap.IO.NRRD)}

\begin{fulllineitems}
\phantomsection\label{api/ClearMap.IO:ClearMap.IO.NRRD.readData}\pysiglinewithargsret{\bfcode{readData}}{\emph{filename}, \emph{**args}}{}
Read nrrd file image data
\begin{quote}\begin{description}
\item[{Parameters}] \leavevmode\begin{itemize}
\item {} 
\textbf{filename} (\emph{str}) --
file name as regular expression

\item {} 
\textbf{x,y,z} (\emph{tuple}) --
data range specifications

\end{itemize}

\item[{Returns}] \leavevmode
\emph{array} --
image data

\end{description}\end{quote}

\end{fulllineitems}

\index{writeData() (in module ClearMap.IO.NRRD)}

\begin{fulllineitems}
\phantomsection\label{api/ClearMap.IO:ClearMap.IO.NRRD.writeData}\pysiglinewithargsret{\bfcode{writeData}}{\emph{filename}, \emph{data}, \emph{options=\{\}}, \emph{separateHeader=False}}{}
Write data to nrrd file
\begin{quote}\begin{description}
\item[{Parameters}] \leavevmode\begin{itemize}
\item {} 
\textbf{filename} (\emph{str}) --
file name as regular expression

\item {} 
\textbf{data} (\emph{array}) --
image data

\item {} 
\textbf{options} (\emph{dict}) --
options dictionary

\item {} 
\textbf{separateHeader} (\emph{bool}) --
write a separate header file

\end{itemize}

\item[{Returns}] \leavevmode
\emph{str} --
nrrd output file name

\end{description}\end{quote}

To sample date use \emph{options{[}'spacings'{]} = {[}s1, s2, s3{]}} for
3d data with sampling deltas \emph{s1}, \emph{s2}, and \emph{s3} in each dimension.

\end{fulllineitems}

\index{dataSize() (in module ClearMap.IO.NRRD)}

\begin{fulllineitems}
\phantomsection\label{api/ClearMap.IO:ClearMap.IO.NRRD.dataSize}\pysiglinewithargsret{\bfcode{dataSize}}{\emph{filename}, \emph{**args}}{}
Read data size from nrrd image
\begin{quote}\begin{description}
\item[{Parameters}] \leavevmode\begin{itemize}
\item {} 
\textbf{filename} (\emph{str}) --
file name as regular expression

\item {} 
\textbf{x,y,z} (\emph{tuple}) --
data range specifications

\end{itemize}

\item[{Returns}] \leavevmode
\emph{tuple} --
data size

\end{description}\end{quote}

\end{fulllineitems}

\index{dataZSize() (in module ClearMap.IO.NRRD)}

\begin{fulllineitems}
\phantomsection\label{api/ClearMap.IO:ClearMap.IO.NRRD.dataZSize}\pysiglinewithargsret{\bfcode{dataZSize}}{\emph{filename}, \emph{z=\textless{}built-in function all\textgreater{}}, \emph{**args}}{}
Read data z size from nrrd image
\begin{quote}\begin{description}
\item[{Parameters}] \leavevmode\begin{itemize}
\item {} 
\textbf{filename} (\emph{str}) --
file name as regular expression

\item {} 
\textbf{z} (\emph{tuple}) --
z data range specification

\end{itemize}

\item[{Returns}] \leavevmode
\emph{int} --
z data size

\end{description}\end{quote}

\end{fulllineitems}

\index{copyData() (in module ClearMap.IO.NRRD)}

\begin{fulllineitems}
\phantomsection\label{api/ClearMap.IO:ClearMap.IO.NRRD.copyData}\pysiglinewithargsret{\bfcode{copyData}}{\emph{source}, \emph{sink}}{}
Copy an nrrd file from source to sink
\begin{quote}\begin{description}
\item[{Parameters}] \leavevmode\begin{itemize}
\item {} 
\textbf{source} (\emph{str}) --
file name pattern of source

\item {} 
\textbf{sink} (\emph{str}) --
file name pattern of sink

\end{itemize}

\item[{Returns}] \leavevmode
\emph{str} --
file name of the copy

\end{description}\end{quote}
\paragraph{Notes}

Todo: dealt with nrdh header files!

\end{fulllineitems}

\index{test() (in module ClearMap.IO.NRRD)}

\begin{fulllineitems}
\phantomsection\label{api/ClearMap.IO:ClearMap.IO.NRRD.test}\pysiglinewithargsret{\bfcode{test}}{}{}
Test NRRD IO module

\end{fulllineitems}



\paragraph{ClearMap.IO.RAW module}
\label{api/ClearMap.IO:clearmap-io-raw-module}

\paragraph{ClearMap.IO.TIF module}
\label{api/ClearMap.IO:clearmap-io-tif-module}

\paragraph{ClearMap.IO.VTK module}
\label{api/ClearMap.IO:module-ClearMap.IO.VTK}\label{api/ClearMap.IO:clearmap-io-vtk-module}\index{ClearMap.IO.VTK (module)}
Interface to write points to VTK files
\paragraph{Notes}
\begin{itemize}
\item {} 
points are assumed to be in {[}x,y,z{]} coordinates as standard in ClearMap

\item {} 
reading of points not supported at the moment!

\end{itemize}

Author
\begin{quote}

Christoph Kirst, The Rockefeller University, New York City, 2015

Modified from matlab code by Kannan Umadevi Venkataraju
\end{quote}
\index{writePoints() (in module ClearMap.IO.VTK)}

\begin{fulllineitems}
\phantomsection\label{api/ClearMap.IO:ClearMap.IO.VTK.writePoints}\pysiglinewithargsret{\bfcode{writePoints}}{\emph{filename}, \emph{points}, \emph{labelImage=None}}{}
Write point data to vtk file
\begin{quote}\begin{description}
\item[{Parameters}] \leavevmode\begin{itemize}
\item {} 
\textbf{filename} (\emph{str}) --
file name

\item {} 
\textbf{points} (\emph{array}) --
point data

\item {} 
\textbf{labelImage} (\emph{str, array or None}) --
optional label image to determine point label

\end{itemize}

\item[{Returns}] \leavevmode
\emph{str} --
file name

\end{description}\end{quote}

\end{fulllineitems}

\index{readPoints() (in module ClearMap.IO.VTK)}

\begin{fulllineitems}
\phantomsection\label{api/ClearMap.IO:ClearMap.IO.VTK.readPoints}\pysiglinewithargsret{\bfcode{readPoints}}{\emph{filename}, \emph{**args}}{}
Read points form vtk file
\paragraph{Notes}
\begin{itemize}
\item {} 
Not implmented yet !

\end{itemize}

\end{fulllineitems}



\subsubsection{ClearMap.Alignment package}
\label{api/ClearMap.Alignment::doc}\label{api/ClearMap.Alignment:clearmap-alignment-package}\phantomsection\label{api/ClearMap.Alignment:module-ClearMap.Alignment}\index{ClearMap.Alignment (module)}
This sub-package provides an interface to alignment tools in order to
register cleared samples to atlases or reference samples.

Supported functionality:
\begin{itemize}
\item {} 
resampling and reorientation of large volumetric images in the
\code{Resampling} module.

\item {} 
registering volumetric data onto references via
\href{http://elastix.isi.uu.nl/}{Elastix} in the
\code{Elastix} module.

\end{itemize}

Main routines for resampling are:
\code{resampleData()}
and \code{resamplePoints()}.

Main routines for elastix registration are:
\code{alignData()},
\code{transformData()} and
\code{transformPoints()}.


\paragraph{ClearMap.Alignment.Elastix module}
\label{api/ClearMap.Alignment:clearmap-alignment-elastix-module}

\paragraph{ClearMap.Alignment.Resampling module}
\label{api/ClearMap.Alignment:clearmap-alignment-resampling-module}

\subsubsection{ClearMap.ImageProcessing package}
\label{api/ClearMap.ImageProcessing:clearmap-imageprocessing-package}\label{api/ClearMap.ImageProcessing::doc}\label{api/ClearMap.ImageProcessing:module-ClearMap.ImageProcessing}\index{ClearMap.ImageProcessing (module)}
This sub-package provides routines for volumetric image processing in parallel

This part of the \emph{ClearMap} toolbox is desinged in a modular way to allow for
fast and flexible extension and addition of specific image processing
algorithms.
\begin{description}
\item[{The toolbox part consists of two parts:}] \leavevmode\begin{itemize}
\item {} 
{\hyperref[api/ClearMap.ImageProcessing:volumetric-image-processing]{\emph{Volumetric Image Processing}}}

\item {} 
{\hyperref[api/ClearMap.ImageProcessing:parallel-image-processing]{\emph{Parallel Image Processing}}}

\end{itemize}

\end{description}


\paragraph{Volumetric Image Processing}
\label{api/ClearMap.ImageProcessing:volumetric-image-processing}
The image processing routines provided in the standard package are listed below

\begin{tabulary}{\linewidth}{|L|L|}
\hline
\textsf{\relax 
Module
} & \textsf{\relax 
Descrition
}\\
\hline
\code{BackgroundRemoval}
 & 
Background estimation and removal via morphological opening
\\
\hline
\code{IlluminationCorrection}
 & 
Correction of vignetting and other illumination errors
\\
\hline
\code{GreyReconstruction}
 & 
Reconstruction of images
\\
\hline
{\hyperref[api/ClearMap.ImageProcessing.Filter:module-ClearMap.ImageProcessing.Filter]{\emph{\code{Filter}}}}
 & 
Filtering of images via a large set of filter kernels
\\
\hline
\code{MaximaDetection}
 & 
Detection of maxima and h-max transforms
\\
\hline
\code{SpotDetection}
 & 
Detection of local peaks / spots / nuclei
\\
\hline
\code{CellDetection}
 & 
Detection of cells
\\
\hline
\code{CellSizeDetection}
 & 
Detection of cell shapes and volumes via e.g. watershed
\\
\hline
\code{IlastikClassification}
 & 
Classification of voxels via interface to \href{http://ilastik.org/}{Ilastik}
\\
\hline\end{tabulary}


While some of these modules provide basic volumetric image processing
functionality some routines combine those functions to provide predefined
higher level cell detection, cell size and intensity measurements.

The higher level routines are optimized for iDISCO+ cleared mouse brain samples
stained for cfos expression. Other data sets might require a redesign of these
higher level functions.


\paragraph{Parallel Image Processing}
\label{api/ClearMap.ImageProcessing:parallel-image-processing}
For large volumetric image data sets from e.g. light sheet microscopy
parallel processing is essential to speed up calculations.

In this toolbox the image processing is parallized via splitting a volumetric
image stack into several sub-stacks, typically in z-direction. Because most of
the image processig steps are non-local sub-stacks are created with overlaps
and the results rejoined accordingly to minimize boundary effects.

Parallel processing is handled via the
{\hyperref[api/ClearMap.ImageProcessing:module-ClearMap.ImageProcessing.StackProcessing]{\emph{\code{StackProcessing}}}} module.


\paragraph{External Packages}
\label{api/ClearMap.ImageProcessing:external-packages}
The {\hyperref[api/ClearMap.ImageProcessing:module-ClearMap.ImageProcessing]{\emph{\code{ImageProcessing}}}} module makes use of external image
processing packages including:
\begin{itemize}
\item {} 
\href{http://opencv.org/}{Open Cv2}

\item {} 
\href{http://www.scipy.org/}{Scipy}

\item {} 
\href{http://scikit-image.org/docs/dev/api/skimage.html}{Scikit-Image}

\item {} 
\href{http://ilastik.org/}{Ilastik}

\end{itemize}

Routines form these packages were freely choosen to optimize for speed and
memory consumption


\paragraph{ClearMap.ImageProcessing.StackProcessing module}
\label{api/ClearMap.ImageProcessing:clearmap-imageprocessing-stackprocessing-module}\label{api/ClearMap.ImageProcessing:module-ClearMap.ImageProcessing.StackProcessing}\index{ClearMap.ImageProcessing.StackProcessing (module)}
Process a image stack in parallel or sequentially

In this toolbox image processing is parallized via splitting a volumetric
image stack into several sub-stacks, typically in z-direction. As most of
the image processig steps are non-local sub-stacks are created with overlaps
and the results rejoined accordingly to minimize boundary effects.

Parallel processing is handled via this module.


\subparagraph{Sub-Stacks}
\label{api/ClearMap.ImageProcessing:sub-stacks}\label{api/ClearMap.ImageProcessing:substack}
The parallel processing module creates a dictionary with information on
the sub-stack as follows:

\begin{tabulary}{\linewidth}{|L|L|}
\hline
\textsf{\relax 
Key
} & \textsf{\relax 
Description
}\\
\hline
\code{stackId}
 & 
id of the sub-stack
\\
\hline
\code{nStacks}
 & 
total number of sub-stacks
\\
\hline
\code{source}
 & 
source file/folder/pattern of the stack
\\
\hline
\code{x}, \code{y}, \code{z}
 & 
the range of the sub-stack with in the full image
\\
\hline
\code{zCenters}
 & 
tuple of the centers of the overlaps
\\
\hline
\code{zCenterIndices}
 & 
tuple of the original indices of the centers of
the overlaps
\\
\hline
\code{zSubStackCenterIndices}
 & 
tuple of the indices of the sub-stack that
correspond to the overlap centers
\\
\hline\end{tabulary}


For exmaple the {\hyperref[api/ClearMap.ImageProcessing:ClearMap.ImageProcessing.StackProcessing.writeSubStack]{\emph{\code{writeSubStack()}}}} routine makes uses of this information
to write out only the sub-parts of the image that is will contribute to the
final total image.
\index{printSubStackInfo() (in module ClearMap.ImageProcessing.StackProcessing)}

\begin{fulllineitems}
\phantomsection\label{api/ClearMap.ImageProcessing:ClearMap.ImageProcessing.StackProcessing.printSubStackInfo}\pysiglinewithargsret{\bfcode{printSubStackInfo}}{\emph{subStack}, \emph{out=\textless{}open file `\textless{}stdout\textgreater{}'}, \emph{mode `w'\textgreater{}}}{}
Print information about the sub-stack
\begin{quote}\begin{description}
\item[{Parameters}] \leavevmode\begin{itemize}
\item {} 
\textbf{subStack} (\emph{dict}) --
the sub-stack info

\item {} 
\textbf{out} (\emph{object}) --
the object to write the information to

\end{itemize}

\end{description}\end{quote}

\end{fulllineitems}

\index{writeSubStack() (in module ClearMap.ImageProcessing.StackProcessing)}

\begin{fulllineitems}
\phantomsection\label{api/ClearMap.ImageProcessing:ClearMap.ImageProcessing.StackProcessing.writeSubStack}\pysiglinewithargsret{\bfcode{writeSubStack}}{\emph{filename}, \emph{img}, \emph{subStack=None}}{}
Write the non-redundant part of a sub-stack to disk

The routine is used to write out images when porcessed in parallel.
It assumes that the filename is a patterned file name.
\begin{quote}\begin{description}
\item[{Parameters}] \leavevmode\begin{itemize}
\item {} 
\textbf{filename} (\emph{str or None}) --
file name pattern as described in
\code{FileList}, if None return as array

\item {} 
\textbf{img} (\emph{array}) --
image data of sub-stack

\item {} 
\textbf{subStack} (\emph{dict or None}) --
sub-stack information, if None write entire image
see {\hyperref[api/ClearMap.ImageProcessing:substack]{\emph{Sub-Stacks}}}

\end{itemize}

\item[{Returns}] \leavevmode
\emph{str or array} --
the file name pattern or image

\end{description}\end{quote}

\end{fulllineitems}

\index{joinPoints() (in module ClearMap.ImageProcessing.StackProcessing)}

\begin{fulllineitems}
\phantomsection\label{api/ClearMap.ImageProcessing:ClearMap.ImageProcessing.StackProcessing.joinPoints}\pysiglinewithargsret{\bfcode{joinPoints}}{\emph{results}, \emph{subStacks=None}, \emph{shiftPoints=True}, \emph{**args}}{}
Joins a list of points obtained from processing a stack in chunks
\begin{quote}\begin{description}
\item[{Parameters}] \leavevmode\begin{itemize}
\item {} 
\textbf{results} (\emph{list}) --
list of point results from the individual sub-processes

\item {} 
\textbf{subStacks} (\emph{list or None}) --
list of all sub-stack information, see {\hyperref[api/ClearMap.ImageProcessing:substack]{\emph{Sub-Stacks}}}

\item {} 
\textbf{shiftPoints} (\emph{bool}) --
if True shift points to refer to origin of the image stack considered
when range specification is given. If False, absolute
position in entire image stack.

\end{itemize}

\item[{Returns}] \leavevmode
\emph{tuple} --
joined points, joined intensities

\end{description}\end{quote}

\end{fulllineitems}

\index{calculateChunkSize() (in module ClearMap.ImageProcessing.StackProcessing)}

\begin{fulllineitems}
\phantomsection\label{api/ClearMap.ImageProcessing:ClearMap.ImageProcessing.StackProcessing.calculateChunkSize}\pysiglinewithargsret{\bfcode{calculateChunkSize}}{\emph{size}, \emph{processes=2}, \emph{chunkSizeMax=100}, \emph{chunkSizeMin=30}, \emph{chunkOverlap=15}, \emph{chunkOptimization=True}, \emph{chunkOptimizationSize=\textless{}built-in function all\textgreater{}}, \emph{verbose=True}}{}
Calculates the chunksize and other info for parallel processing

The sub stack information is described in {\hyperref[api/ClearMap.ImageProcessing:substack]{\emph{Sub-Stacks}}}
\begin{quote}\begin{description}
\item[{Parameters}] \leavevmode\begin{itemize}
\item {} 
\textbf{processes} (\emph{int}) --
number of parallel processes

\item {} 
\textbf{chunkSizeMax} (\emph{int}) --
maximal size of a sub-stack

\item {} 
\textbf{chunkSizeMin} (\emph{int}) --
minial size of a sub-stack

\item {} 
\textbf{chunkOverlap} (\emph{int}) --
minimal sub-stack overlap

\item {} 
\textbf{chunkOptimization} (\emph{bool}) --
optimize chunck sizes to best fit number of processes

\item {} 
\textbf{chunkOptimizationSize} (\emph{bool or all}) --
if True only decrease the chunk size when optimizing

\item {} 
\textbf{verbose} (\emph{bool}) --
print information on sub-stack generation

\end{itemize}

\item[{Returns}] \leavevmode
\emph{tuple} --
number of chunks, z-ranges of each chunk, z-centers in overlap regions

\end{description}\end{quote}

\end{fulllineitems}

\index{calculateSubStacks() (in module ClearMap.ImageProcessing.StackProcessing)}

\begin{fulllineitems}
\phantomsection\label{api/ClearMap.ImageProcessing:ClearMap.ImageProcessing.StackProcessing.calculateSubStacks}\pysiglinewithargsret{\bfcode{calculateSubStacks}}{\emph{source}, \emph{z=\textless{}built-in function all\textgreater{}}, \emph{x=\textless{}built-in function all\textgreater{}}, \emph{y=\textless{}built-in function all\textgreater{}}, \emph{**args}}{}
Calculates the chunksize and other info for parallel processing and returns a list of sub-stack objects

The sub-stack information is described in {\hyperref[api/ClearMap.ImageProcessing:substack]{\emph{Sub-Stacks}}}
\begin{quote}\begin{description}
\item[{Parameters}] \leavevmode\begin{itemize}
\item {} 
\textbf{source} (\emph{str}) --
image source

\item {} 
\textbf{x,y,z} (\emph{tuple or all}) --
range specifications

\item {} 
\textbf{processes} (\emph{int}) --
number of parallel processes

\item {} 
\textbf{chunkSizeMax} (\emph{int}) --
maximal size of a sub-stack

\item {} 
\textbf{chunkSizeMin} (\emph{int}) --
minial size of a sub-stack

\item {} 
\textbf{chunkOverlap} (\emph{int}) --
minimal sub-stack overlap

\item {} 
\textbf{chunkOptimization} (\emph{bool}) --
optimize chunck sizes to best fit number of processes

\item {} 
\textbf{chunkOptimizationSize} (\emph{bool or all}) --
if True only decrease the chunk size when optimizing

\item {} 
\textbf{verbose} (\emph{bool}) --
print information on sub-stack generation

\end{itemize}

\item[{Returns}] \leavevmode
\emph{list} --
list of sub-stack objects

\end{description}\end{quote}

\end{fulllineitems}

\index{noProcessing() (in module ClearMap.ImageProcessing.StackProcessing)}

\begin{fulllineitems}
\phantomsection\label{api/ClearMap.ImageProcessing:ClearMap.ImageProcessing.StackProcessing.noProcessing}\pysiglinewithargsret{\bfcode{noProcessing}}{\emph{img}, \emph{**parameter}}{}
Perform no image processing at all and return original image

Used as the default functon in {\hyperref[api/ClearMap.ImageProcessing:ClearMap.ImageProcessing.StackProcessing.parallelProcessStack]{\emph{\code{parallelProcessStack()}}}} and
{\hyperref[api/ClearMap.ImageProcessing:ClearMap.ImageProcessing.StackProcessing.sequentiallyProcessStack]{\emph{\code{sequentiallyProcessStack()}}}}.
\begin{quote}\begin{description}
\item[{Parameters}] \leavevmode
\textbf{img} (\emph{array}) --
imag

\item[{Returns}] \leavevmode
\emph{(array)} --
the original image

\end{description}\end{quote}

\end{fulllineitems}

\index{parallelProcessStack() (in module ClearMap.ImageProcessing.StackProcessing)}

\begin{fulllineitems}
\phantomsection\label{api/ClearMap.ImageProcessing:ClearMap.ImageProcessing.StackProcessing.parallelProcessStack}\pysiglinewithargsret{\bfcode{parallelProcessStack}}{\emph{source}, \emph{x=\textless{}built-in function all\textgreater{}}, \emph{y=\textless{}built-in function all\textgreater{}}, \emph{z=\textless{}built-in function all\textgreater{}}, \emph{sink=None}, \emph{processes=2}, \emph{chunkSizeMax=100}, \emph{chunkSizeMin=30}, \emph{chunkOverlap=15}, \emph{chunkOptimization=True}, \emph{chunkOptimizationSize=\textless{}built-in function all\textgreater{}}, \emph{function=\textless{}function noProcessing\textgreater{}}, \emph{join=\textless{}function joinPoints\textgreater{}}, \emph{verbose=False}, \emph{**parameter}}{}
Parallel process a image stack

Main routine that distributes image processing on paralllel processes.
\begin{quote}\begin{description}
\item[{Parameters}] \leavevmode\begin{itemize}
\item {} 
\textbf{source} (\emph{str}) --
image source

\item {} 
\textbf{x,y,z} (\emph{tuple or all}) --
range specifications

\item {} 
\textbf{sink} (\emph{str or None}) --
destination for the result

\item {} 
\textbf{processes} (\emph{int}) --
number of parallel processes

\item {} 
\textbf{chunkSizeMax} (\emph{int}) --
maximal size of a sub-stack

\item {} 
\textbf{chunkSizeMin} (\emph{int}) --
minial size of a sub-stack

\item {} 
\textbf{chunkOverlap} (\emph{int}) --
minimal sub-stack overlap

\item {} 
\textbf{chunkOptimization} (\emph{bool}) --
optimize chunck sizes to best fit number of processes

\item {} 
\textbf{chunkOptimizationSize} (\emph{bool or all}) --
if True only decrease the chunk size when optimizing

\item {} 
\textbf{function} (\emph{function}) --
the main image processing script

\item {} 
\textbf{join} (\emph{function}) --
the fuction to join the results from the image processing script

\item {} 
\textbf{verbose} (\emph{bool}) --
print information on sub-stack generation

\end{itemize}

\item[{Returns}] \leavevmode
\emph{str or array} --
results of the image processing

\end{description}\end{quote}

\end{fulllineitems}

\index{sequentiallyProcessStack() (in module ClearMap.ImageProcessing.StackProcessing)}

\begin{fulllineitems}
\phantomsection\label{api/ClearMap.ImageProcessing:ClearMap.ImageProcessing.StackProcessing.sequentiallyProcessStack}\pysiglinewithargsret{\bfcode{sequentiallyProcessStack}}{\emph{source}, \emph{x=\textless{}built-in function all\textgreater{}}, \emph{y=\textless{}built-in function all\textgreater{}}, \emph{z=\textless{}built-in function all\textgreater{}}, \emph{sink=None}, \emph{chunkSizeMax=100}, \emph{chunkSizeMin=30}, \emph{chunkOverlap=15}, \emph{function=\textless{}function noProcessing\textgreater{}}, \emph{join=\textless{}function joinPoints\textgreater{}}, \emph{verbose=False}, \emph{**parameter}}{}
Sequential image processing on a stack

Main routine that sequentially processes a large image on sub-stacks.
\begin{quote}\begin{description}
\item[{Parameters}] \leavevmode\begin{itemize}
\item {} 
\textbf{source} (\emph{str}) --
image source

\item {} 
\textbf{x,y,z} (\emph{tuple or all}) --
range specifications

\item {} 
\textbf{sink} (\emph{str or None}) --
destination for the result

\item {} 
\textbf{processes} (\emph{int}) --
number of parallel processes

\item {} 
\textbf{chunkSizeMax} (\emph{int}) --
maximal size of a sub-stack

\item {} 
\textbf{chunkSizeMin} (\emph{int}) --
minial size of a sub-stack

\item {} 
\textbf{chunkOverlap} (\emph{int}) --
minimal sub-stack overlap

\item {} 
\textbf{chunkOptimization} (\emph{bool}) --
optimize chunck sizes to best fit number of processes

\item {} 
\textbf{chunkOptimizationSize} (\emph{bool or all}) --
if True only decrease the chunk size when optimizing

\item {} 
\textbf{function} (\emph{function}) --
the main image processing script

\item {} 
\textbf{join} (\emph{function}) --
the fuction to join the results from the image processing script

\item {} 
\textbf{verbose} (\emph{bool}) --
print information on sub-stack generation

\end{itemize}

\item[{Returns}] \leavevmode
\emph{str or array} --
results of the image processing

\end{description}\end{quote}

\end{fulllineitems}



\paragraph{ClearMap.ImageProcessing.CellDetection module}
\label{api/ClearMap.ImageProcessing:clearmap-imageprocessing-celldetection-module}

\paragraph{ClearMap.ImageProcessing.CellSizeDetection module}
\label{api/ClearMap.ImageProcessing:clearmap-imageprocessing-cellsizedetection-module}

\paragraph{Subpackages}
\label{api/ClearMap.ImageProcessing:subpackages}

\subparagraph{ClearMap.ImageProcessing.Filter package}
\label{api/ClearMap.ImageProcessing.Filter:module-ClearMap.ImageProcessing.Filter}\label{api/ClearMap.ImageProcessing.Filter::doc}\label{api/ClearMap.ImageProcessing.Filter:clearmap-imageprocessing-filter-package}\index{ClearMap.ImageProcessing.Filter (module)}
This sub-package provides various volumetric filter kernels and structure elments

A set of linear filters can be applied to the data using
\code{LinearFilter}.

Because its utility for cell detection the difference of Gaussians filter
is implemented directly in \code{DoGFilter}.

The fitler kernels defined in {\hyperref[api/ClearMap.ImageProcessing.Filter:module-ClearMap.ImageProcessing.Filter.FilterKernel]{\emph{\code{FilterKernel}}}}
can be used in combination with the \code{Convolution}
module.

Structured elements defined in
\code{StructureElements} can be used in
combination with various morphological operations, e.g. used in the
:mod:\textasciitilde{}ClearMap.ImageProcessing.BackgroundRemoval{}` module.


\subparagraph{ClearMap.ImageProcessing.Filter.LinearFilter module}
\label{api/ClearMap.ImageProcessing.Filter:clearmap-imageprocessing-filter-linearfilter-module}

\subparagraph{ClearMap.ImageProcessing.Filter.DoGFilter module}
\label{api/ClearMap.ImageProcessing.Filter:clearmap-imageprocessing-filter-dogfilter-module}

\subparagraph{ClearMap.ImageProcessing.Filter.Convolution module}
\label{api/ClearMap.ImageProcessing.Filter:module-ClearMap.ImageProcessing.Filter.Convolution}\label{api/ClearMap.ImageProcessing.Filter:clearmap-imageprocessing-filter-convolution-module}\index{ClearMap.ImageProcessing.Filter.Convolution (module)}
Convolve volumetric data with a 3d kernel, optimized for memory / float32 use

Based on \href{http://docs.scipy.org/doc/scipy/reference/signal.html}{scipy.signal}
routines.

Author
\begin{quote}

Original code from \href{http://docs.scipy.org/doc/scipy/reference/signal.html}{scipy.signal}.

Modified by Chirstoph Kirst to optimize memory and sped and integration into ClearMap.
The Rockefeller University, New York City, 2015
\end{quote}
\index{convolve() (in module ClearMap.ImageProcessing.Filter.Convolution)}

\begin{fulllineitems}
\phantomsection\label{api/ClearMap.ImageProcessing.Filter:ClearMap.ImageProcessing.Filter.Convolution.convolve}\pysiglinewithargsret{\bfcode{convolve}}{\emph{x}, \emph{k}, \emph{mode='same'}}{}
Convolve array with kernel using float32 / complex64, optimized for memory consumption and speed
\begin{quote}\begin{description}
\item[{Parameters}] \leavevmode\begin{itemize}
\item {} 
\textbf{x} (\emph{array}) --
data to be convolved

\item {} 
\textbf{k} (\emph{array}) --
filter kernel

\end{itemize}

\item[{Returns}] \leavevmode
\emph{array} --
convolution

\end{description}\end{quote}

\end{fulllineitems}



\subparagraph{ClearMap.ImageProcessing.Filter.FilterKernel module}
\label{api/ClearMap.ImageProcessing.Filter:clearmap-imageprocessing-filter-filterkernel-module}\label{api/ClearMap.ImageProcessing.Filter:module-ClearMap.ImageProcessing.Filter.FilterKernel}\index{ClearMap.ImageProcessing.Filter.FilterKernel (module)}
Implementation of various volumetric filter kernels


\subparagraph{Filter Type}
\label{api/ClearMap.ImageProcessing.Filter:filtertypes}\label{api/ClearMap.ImageProcessing.Filter:filter-type}
Filter types defined by the \code{ftype} key include:

\begin{tabulary}{\linewidth}{|L|L|}
\hline
\textsf{\relax 
Type
} & \textsf{\relax 
Descrition
}\\
\hline
\code{mean}
 & 
uniform averaging filter
\\
\hline
\code{gaussian}
 & 
Gaussian filter
\\
\hline
\code{log}
 & 
Laplacian of Gaussian filter (LoG)
\\
\hline
\code{dog}
 & 
Difference of Gaussians filter (DoG)
\\
\hline
\code{sphere}
 & 
Sphere filter
\\
\hline
\code{disk}
 & 
Disk filter
\\
\hline\end{tabulary}

\index{filterKernel() (in module ClearMap.ImageProcessing.Filter.FilterKernel)}

\begin{fulllineitems}
\phantomsection\label{api/ClearMap.ImageProcessing.Filter:ClearMap.ImageProcessing.Filter.FilterKernel.filterKernel}\pysiglinewithargsret{\bfcode{filterKernel}}{\emph{ftype='Gaussian'}, \emph{size=(5}, \emph{5)}, \emph{sigma=None}, \emph{radius=None}, \emph{sigma2=None}}{}
Creates a filter kernel of a special type
\begin{quote}\begin{description}
\item[{Parameters}] \leavevmode\begin{itemize}
\item {} 
\textbf{ftype} (\emph{str}) --
filter type, see {\hyperref[api/ClearMap.ImageProcessing.Filter:filtertypes]{\emph{Filter Type}}}

\item {} 
\textbf{size} (\emph{array or tuple}) --
size of the filter kernel

\item {} 
\textbf{sigma} (\emph{tuple or float}) --
std for the first gaussian (if present)

\item {} 
\textbf{radius} (\emph{tuple or float}) --
radius of the kernel (if applicable)

\item {} 
\textbf{sigma2} (\emph{tuple or float}) --
std of a second gaussian (if present)

\end{itemize}

\item[{Returns}] \leavevmode
\emph{array} --
structure element

\end{description}\end{quote}

\end{fulllineitems}

\index{filterKernel2D() (in module ClearMap.ImageProcessing.Filter.FilterKernel)}

\begin{fulllineitems}
\phantomsection\label{api/ClearMap.ImageProcessing.Filter:ClearMap.ImageProcessing.Filter.FilterKernel.filterKernel2D}\pysiglinewithargsret{\bfcode{filterKernel2D}}{\emph{ftype='Gaussian'}, \emph{size=(5}, \emph{5)}, \emph{sigma=None}, \emph{sigma2=None}, \emph{radius=None}}{}
Creates a 2d filter kernel of a special type
\begin{quote}\begin{description}
\item[{Parameters}] \leavevmode\begin{itemize}
\item {} 
\textbf{ftype} (\emph{str}) --
filter type, see {\hyperref[api/ClearMap.ImageProcessing.Filter:filtertypes]{\emph{Filter Type}}}

\item {} 
\textbf{size} (\emph{array or tuple}) --
size of the filter kernel

\item {} 
\textbf{sigma} (\emph{tuple or float}) --
std for the first gaussian (if present)

\item {} 
\textbf{radius} (\emph{tuple or float}) --
radius of the kernel (if applicable)

\item {} 
\textbf{sigma2} (\emph{tuple or float}) --
std of a second gaussian (if present)

\end{itemize}

\item[{Returns}] \leavevmode
\emph{array} --
structure element

\end{description}\end{quote}

\end{fulllineitems}

\index{filterKernel3D() (in module ClearMap.ImageProcessing.Filter.FilterKernel)}

\begin{fulllineitems}
\phantomsection\label{api/ClearMap.ImageProcessing.Filter:ClearMap.ImageProcessing.Filter.FilterKernel.filterKernel3D}\pysiglinewithargsret{\bfcode{filterKernel3D}}{\emph{ftype='Gaussian'}, \emph{size=(5}, \emph{5}, \emph{5)}, \emph{sigma=None}, \emph{sigma2=None}, \emph{radius=None}}{}
Creates a 3d filter kernel of a special type
\begin{quote}\begin{description}
\item[{Parameters}] \leavevmode\begin{itemize}
\item {} 
\textbf{ftype} (\emph{str}) --
filter type, see {\hyperref[api/ClearMap.ImageProcessing.Filter:filtertypes]{\emph{Filter Type}}}

\item {} 
\textbf{size} (\emph{array or tuple}) --
size of the filter kernel

\item {} 
\textbf{sigma} (\emph{tuple or float}) --
std for the first gaussian (if present)

\item {} 
\textbf{radius} (\emph{tuple or float}) --
radius of the kernel (if applicable)

\item {} 
\textbf{sigma2} (\emph{tuple or float}) --
std of a second gaussian (if present)

\end{itemize}

\item[{Returns}] \leavevmode
\emph{array} --
structure element

\end{description}\end{quote}

\end{fulllineitems}

\index{test() (in module ClearMap.ImageProcessing.Filter.FilterKernel)}

\begin{fulllineitems}
\phantomsection\label{api/ClearMap.ImageProcessing.Filter:ClearMap.ImageProcessing.Filter.FilterKernel.test}\pysiglinewithargsret{\bfcode{test}}{}{}
Test FilterKernel module

\end{fulllineitems}



\subparagraph{ClearMap.ImageProcessing.Filter.StructureElement module}
\label{api/ClearMap.ImageProcessing.Filter:clearmap-imageprocessing-filter-structureelement-module}\label{api/ClearMap.ImageProcessing.Filter:module-ClearMap.ImageProcessing.Filter.StructureElement}\index{ClearMap.ImageProcessing.Filter.StructureElement (module)}
Routines to generate various structure elements

Structured elements defined by the \code{setype} key include:


\subparagraph{Structure Element Types}
\label{api/ClearMap.ImageProcessing.Filter:structureelementtypes}\label{api/ClearMap.ImageProcessing.Filter:structure-element-types}
\begin{tabulary}{\linewidth}{|L|L|}
\hline
\textsf{\relax 
Type
} & \textsf{\relax 
Descrition
}\\
\hline
\code{sphere}
 & 
Sphere structure
\\
\hline
\code{disk}
 & 
Disk structure
\\
\hline\end{tabulary}


\begin{notice}{note}{Note:}
To be extended!
\end{notice}
\index{structureElement() (in module ClearMap.ImageProcessing.Filter.StructureElement)}

\begin{fulllineitems}
\phantomsection\label{api/ClearMap.ImageProcessing.Filter:ClearMap.ImageProcessing.Filter.StructureElement.structureElement}\pysiglinewithargsret{\bfcode{structureElement}}{\emph{setype='Disk'}, \emph{sesize=(3}, \emph{3)}}{}
Creates specific 2d and 3d structuring elements
\begin{quote}\begin{description}
\item[{Parameters}] \leavevmode\begin{itemize}
\item {} 
\textbf{setype} (\emph{str}) --
structure element type, see {\hyperref[api/ClearMap.ImageProcessing.Filter:structureelementtypes]{\emph{Structure Element Types}}}

\item {} 
\textbf{sesize} (\emph{array or tuple}) --
size of the structure element

\end{itemize}

\item[{Returns}] \leavevmode
\emph{array} --
structure element

\end{description}\end{quote}

\end{fulllineitems}

\index{structureElementOffsets() (in module ClearMap.ImageProcessing.Filter.StructureElement)}

\begin{fulllineitems}
\phantomsection\label{api/ClearMap.ImageProcessing.Filter:ClearMap.ImageProcessing.Filter.StructureElement.structureElementOffsets}\pysiglinewithargsret{\bfcode{structureElementOffsets}}{\emph{sesize}}{}
Calculates offsets for a structural element given its size
\begin{quote}\begin{description}
\item[{Parameters}] \leavevmode
\textbf{sesize} (\emph{array or tuple}) --
size of the structure element

\item[{Returns}] \leavevmode
\emph{array} --
offsets to center taking care of even/odd ummber of elements

\end{description}\end{quote}

\end{fulllineitems}

\index{structureElement2D() (in module ClearMap.ImageProcessing.Filter.StructureElement)}

\begin{fulllineitems}
\phantomsection\label{api/ClearMap.ImageProcessing.Filter:ClearMap.ImageProcessing.Filter.StructureElement.structureElement2D}\pysiglinewithargsret{\bfcode{structureElement2D}}{\emph{setype='Disk'}, \emph{sesize=(3}, \emph{3)}}{}
Creates specific 2d structuring elements
\begin{quote}\begin{description}
\item[{Parameters}] \leavevmode\begin{itemize}
\item {} 
\textbf{setype} (\emph{str}) --
structure element type, see {\hyperref[api/ClearMap.ImageProcessing.Filter:structureelementtypes]{\emph{Structure Element Types}}}

\item {} 
\textbf{sesize} (\emph{array or tuple}) --
size of the structure element

\end{itemize}

\item[{Returns}] \leavevmode
\emph{array} --
structure element

\end{description}\end{quote}

\end{fulllineitems}

\index{structureElement3D() (in module ClearMap.ImageProcessing.Filter.StructureElement)}

\begin{fulllineitems}
\phantomsection\label{api/ClearMap.ImageProcessing.Filter:ClearMap.ImageProcessing.Filter.StructureElement.structureElement3D}\pysiglinewithargsret{\bfcode{structureElement3D}}{\emph{setype='Disk'}, \emph{sesize=(3}, \emph{3}, \emph{3)}}{}
Creates specific 3d structuring elements
\begin{quote}\begin{description}
\item[{Parameters}] \leavevmode\begin{itemize}
\item {} 
\textbf{setype} (\emph{str}) --
structure element type, see {\hyperref[api/ClearMap.ImageProcessing.Filter:structureelementtypes]{\emph{Structure Element Types}}}

\item {} 
\textbf{sesize} (\emph{array or tuple}) --
size of the structure element

\end{itemize}

\item[{Returns}] \leavevmode
\emph{array} --
structure element

\end{description}\end{quote}

\end{fulllineitems}



\paragraph{ClearMap.ImageProcessing.IlluminationCorrection module}
\label{api/ClearMap.ImageProcessing:clearmap-imageprocessing-illuminationcorrection-module}

\paragraph{ClearMap.ImageProcessing.BackgroundRemoval module}
\label{api/ClearMap.ImageProcessing:clearmap-imageprocessing-backgroundremoval-module}

\paragraph{ClearMap.ImageProcessing.GreyReconstruction module}
\label{api/ClearMap.ImageProcessing:clearmap-imageprocessing-greyreconstruction-module}

\paragraph{ClearMap.ImageProcessing.SpotDetection module}
\label{api/ClearMap.ImageProcessing:clearmap-imageprocessing-spotdetection-module}

\paragraph{ClearMap.ImageProcessing.MaximaDetection module}
\label{api/ClearMap.ImageProcessing:clearmap-imageprocessing-maximadetection-module}

\paragraph{ClearMap.ImageProcessing.IlastikClassification module}
\label{api/ClearMap.ImageProcessing:clearmap-imageprocessing-ilastikclassification-module}

\paragraph{ClearMap.ImageProcessing.ImageStatistics module}
\label{api/ClearMap.ImageProcessing:module-ClearMap.ImageProcessing.ImageStatistics}\label{api/ClearMap.ImageProcessing:clearmap-imageprocessing-imagestatistics-module}\index{ClearMap.ImageProcessing.ImageStatistics (module)}
Functions to gather iamge statistics in large volumetric images

The main routines extract information from a large volumetric image, such as
the maximum or mean.
\index{calculateStatistics() (in module ClearMap.ImageProcessing.ImageStatistics)}

\begin{fulllineitems}
\phantomsection\label{api/ClearMap.ImageProcessing:ClearMap.ImageProcessing.ImageStatistics.calculateStatistics}\pysiglinewithargsret{\bfcode{calculateStatistics}}{\emph{source}, \emph{sink=None}, \emph{calculateStatisticsParameter=None}, \emph{method='Max'}, \emph{remove=True}, \emph{processMethod=\textless{}built-in function all\textgreater{}}, \emph{verbose=False}, \emph{**parameter}}{}
Calculate statisticsfrom image data

This is a main script to start extracting statistics of volumetric image data.
\begin{quote}\begin{description}
\item[{Parameters}] \leavevmode\begin{itemize}
\item {} 
\textbf{source} (\emph{str or array}) --
Image source

\item {} 
\textbf{sink} (\emph{str or None}) --
destination for the results

\item {} 
\textbf{calculateStatisticsParameter} (\emph{dict}) --

\begin{tabulary}{\linewidth}{|L|L|L|}
\hline
\textsf{\relax 
Name
} & \textsf{\relax 
Type
} & \textsf{\relax 
Descritption
}\\
\hline
\emph{method}
 & 
(str or function)
 & 
function to extract statistic, must be trivially distributable
if None, do not extract information
\\
\hline
\emph{remove}
 & 
(bool)
 & 
remove redundant overlap
\\
\hline
\emph{verbose}
 & 
(bool or int)
 & 
print / plot information about this step
\\
\hline\end{tabulary}


\item {} 
\textbf{method} (\emph{str or function})

\item {} 
\textbf{processMethod} (\emph{str or all}) --
`sequential' or `parallel'. if all its choosen automatically

\item {} 
\textbf{verbose} (\emph{bool}) --
print info

\item {} 
\textbf{**parameter} (\emph{dict}) --
parameter for the image processing sub-routines

\end{itemize}

\item[{Returns}] \leavevmode
list of statistics

\end{description}\end{quote}

\end{fulllineitems}

\index{calculateStatisticsOnStack() (in module ClearMap.ImageProcessing.ImageStatistics)}

\begin{fulllineitems}
\phantomsection\label{api/ClearMap.ImageProcessing:ClearMap.ImageProcessing.ImageStatistics.calculateStatisticsOnStack}\pysiglinewithargsret{\bfcode{calculateStatisticsOnStack}}{\emph{img}, \emph{calculateStatisticsParameter=None}, \emph{method='Max'}, \emph{remove=True}, \emph{verbose=False}, \emph{subStack=None}, \emph{out=\textless{}open file `\textless{}stdout\textgreater{}'}, \emph{mode `w'\textgreater{}}, \emph{**parameter}}{}
Calculate a statistics from a large volumetric image

The statistics is assumed to be trivially distributable, i.e. max or mean.
\begin{quote}\begin{description}
\item[{Parameters}] \leavevmode\begin{itemize}
\item {} 
\textbf{img} (\emph{array}) --
image data

\item {} 
\textbf{calculateStatisticsParameter} (\emph{dict}) --

\begin{tabulary}{\linewidth}{|L|L|L|}
\hline
\textsf{\relax 
Name
} & \textsf{\relax 
Type
} & \textsf{\relax 
Descritption
}\\
\hline
\emph{method}
 & 
(str or function)
 & 
function to extract statistic, must be trivially distributable
if None, do not extract information
\\
\hline
\emph{remove}
 & 
(bool)
 & 
remove redundant overlap
\\
\hline
\emph{verbose}
 & 
(bool or int)
 & 
print / plot information about this step
\\
\hline\end{tabulary}


\item {} 
\textbf{subStack} (\emph{dict or None}) --
sub-stack information

\item {} 
\textbf{verbose} (\emph{bool}) --
print progress info

\item {} 
\textbf{out} (\emph{object}) --
object to write progress info to

\end{itemize}

\item[{Returns}] \leavevmode
\emph{array or number} --
extracted statistics

\end{description}\end{quote}

\begin{notice}{note}{Note:}
One might need to choose zero overlap in the stacks to function properly!
\end{notice}

\end{fulllineitems}

\index{joinStatistics() (in module ClearMap.ImageProcessing.ImageStatistics)}

\begin{fulllineitems}
\phantomsection\label{api/ClearMap.ImageProcessing:ClearMap.ImageProcessing.ImageStatistics.joinStatistics}\pysiglinewithargsret{\bfcode{joinStatistics}}{\emph{results}, \emph{calculateStatisticsParameter=None}, \emph{method='Max'}, \emph{subStacks=None}, \emph{**parameter}}{}
Joins a list of calculated statistics
\begin{quote}\begin{description}
\item[{Parameters}] \leavevmode\begin{itemize}
\item {} 
\textbf{results} (\emph{list}) --
list of statics results from the individual sub-processes

\item {} 
\textbf{calculateStatisticsParameter} (\emph{dict}) --

\begin{tabulary}{\linewidth}{|L|L|L|}
\hline
\textsf{\relax 
Name
} & \textsf{\relax 
Type
} & \textsf{\relax 
Descritption
}\\
\hline
\emph{method}
 & 
(str or function)
 & 
function to extract statistic, must be trivially distributable
if None, do not extract information
\\
\hline\end{tabulary}


\item {} 
\textbf{subStacks} (\emph{list or None}) --
list of all sub-stack information, see {\hyperref[api/ClearMap.ImageProcessing:substack]{\emph{Sub-Stacks}}}

\end{itemize}

\item[{Returns}] \leavevmode
\emph{list or object} --
joined statistics

\end{description}\end{quote}

\end{fulllineitems}



\subsubsection{ClearMap.Analysis package}
\label{api/ClearMap.Analysis:clearmap-analysis-package}\label{api/ClearMap.Analysis::doc}\label{api/ClearMap.Analysis:module-ClearMap.Analysis}\index{ClearMap.Analysis (module)}
ClearMap analysis and statistics toolbox.

This part of ClearMap provides a toolbox for the statistical analysis and
visualization of detected cells or structures and region specific analysis
of annoated data.

For cleared mouse brains aligned to the Allen brain atlas a wide range of
statistical analysis tools with respect to the anotated brain regions in
the atlas is supported.

Key moduls are:

\begin{tabulary}{\linewidth}{|L|L|}
\hline
\textsf{\relax 
Module
} & \textsf{\relax 
Descrition
}\\
\hline
\code{Voxelization}
 & 
Voxelization of cells for visualization and analysis
\\
\hline
{\hyperref[api/ClearMap.Analysis:module-ClearMap.Analysis.Statistics]{\emph{\code{Statistics}}}}
 & 
Statistical tools for the analysis of detected cells
\\
\hline
{\hyperref[api/ClearMap.Analysis:module-ClearMap.Analysis.Label]{\emph{\code{Label}}}}
 & 
Tools to analysise data with espect to annotated refereneces
\\
\hline\end{tabulary}



\paragraph{Subpackages}
\label{api/ClearMap.Analysis:subpackages}

\subparagraph{ClearMap.Analysis.Tools package}
\label{api/ClearMap.Analysis.Tools:clearmap-analysis-tools-package}\label{api/ClearMap.Analysis.Tools:module-ClearMap.Analysis.Tools}\label{api/ClearMap.Analysis.Tools::doc}\index{ClearMap.Analysis.Tools (module)}
Analysis and statistics tools not in standard python packages.


\subparagraph{ClearMap.Analysis.Tools.Extrapolate module}
\label{api/ClearMap.Analysis.Tools:clearmap-analysis-tools-extrapolate-module}\label{api/ClearMap.Analysis.Tools:module-ClearMap.Analysis.Tools.Extrapolate}\index{ClearMap.Analysis.Tools.Extrapolate (module)}
Method to extend interpolation objects to constantly / linearly extrapolate.
\index{extrap1d() (in module ClearMap.Analysis.Tools.Extrapolate)}

\begin{fulllineitems}
\phantomsection\label{api/ClearMap.Analysis.Tools:ClearMap.Analysis.Tools.Extrapolate.extrap1d}\pysiglinewithargsret{\bfcode{extrap1d}}{\emph{x}, \emph{y}, \emph{interpolation='linear'}, \emph{exterpolation='constant'}}{}
Interpolate on given values and extrapolate outside the given data
\begin{quote}\begin{description}
\item[{Parameters}] \leavevmode\begin{itemize}
\item {} 
\textbf{x} (\emph{numpy.array}) --
x values of the data to interpolate

\item {} 
\textbf{y} (\emph{numpy.array}) --
y values of the data to interpolate

\item {} 
\textbf{interpolation} (\emph{Optional{[}str{]}}) --
interpolation method, see kind of scipy.interpolate.interp1d, default: ``linear''

\item {} 
\textbf{exterpolation} (\emph{Optional{[}str{]}}) --
interpolation method, either ``linear'' or ``constant''

\end{itemize}

\item[{Returns}] \leavevmode
\emph{(function)} --
inter- and extra-polation function

\end{description}\end{quote}

\end{fulllineitems}

\index{extrap1dFromInterp1d() (in module ClearMap.Analysis.Tools.Extrapolate)}

\begin{fulllineitems}
\phantomsection\label{api/ClearMap.Analysis.Tools:ClearMap.Analysis.Tools.Extrapolate.extrap1dFromInterp1d}\pysiglinewithargsret{\bfcode{extrap1dFromInterp1d}}{\emph{interpolator}, \emph{exterpolation='constant'}}{}
Extend interpolation function to extrapolate outside the given data
\begin{quote}\begin{description}
\item[{Parameters}] \leavevmode\begin{itemize}
\item {} 
\textbf{interpolator} (\emph{function}) --
interpolating function, see e.g. scipy.interpolate.interp1d

\item {} 
\textbf{exterpolation} (\emph{Optional{[}str{]}}) --
interpolation method, either ``linear'' or ``constant''

\end{itemize}

\item[{Returns}] \leavevmode
\emph{(function)} --
inter- and extra-polation function

\end{description}\end{quote}

\end{fulllineitems}



\subparagraph{ClearMap.Analysis.Tools.MultipleComparisonCorrection module}
\label{api/ClearMap.Analysis.Tools:clearmap-analysis-tools-multiplecomparisoncorrection-module}\label{api/ClearMap.Analysis.Tools:module-ClearMap.Analysis.Tools.MultipleComparisonCorrection}\index{ClearMap.Analysis.Tools.MultipleComparisonCorrection (module)}
Correction methods for multiple comparison tests
\index{correctPValues() (in module ClearMap.Analysis.Tools.MultipleComparisonCorrection)}

\begin{fulllineitems}
\phantomsection\label{api/ClearMap.Analysis.Tools:ClearMap.Analysis.Tools.MultipleComparisonCorrection.correctPValues}\pysiglinewithargsret{\bfcode{correctPValues}}{\emph{pvalues}, \emph{method='BH'}}{}
Corrects p-values for multiple testing using various methods
\begin{quote}\begin{description}
\item[{Parameters}] \leavevmode\begin{itemize}
\item {} 
\textbf{pvalues} (\emph{array}) --
list of p values to be corrected

\item {} 
\textbf{method} (\emph{Optional{[}str{]}}) --
method to use: BH = FDR = Benjamini-Hochberg, B = FWER = Bonferoni

\end{itemize}

\end{description}\end{quote}
\paragraph{References}
\begin{itemize}
\item {} 
\href{http://www.jstor.org/stable/2346101?seq=1\#page\_scan\_tab\_contents}{Benjamini Hochberg, 1995}

\item {} 
\href{http://www.tandfonline.com/doi/abs/10.1080/01621459.1961.10482090\#.VmHWUHbH6KE}{Bonferoni correction}

\item {} 
\href{https://www.r-project.org/}{R statistics package}

\end{itemize}
\paragraph{Notes}
\begin{itemize}
\item {} 
modified from \href{http://statsmodels.sourceforge.net/ipdirective/generated/scikits.statsmodels.sandbox.stats.multicomp.multipletests.html}{http://statsmodels.sourceforge.net/ipdirective/generated/scikits.statsmodels.sandbox.stats.multicomp.multipletests.html}

\end{itemize}

\end{fulllineitems}

\index{estimateQValues() (in module ClearMap.Analysis.Tools.MultipleComparisonCorrection)}

\begin{fulllineitems}
\phantomsection\label{api/ClearMap.Analysis.Tools:ClearMap.Analysis.Tools.MultipleComparisonCorrection.estimateQValues}\pysiglinewithargsret{\bfcode{estimateQValues}}{\emph{pvalues}, \emph{m=None}, \emph{pi0=None}, \emph{verbose=False}, \emph{lowMemory=False}}{}
Estimates q-values from p-values
\begin{quote}\begin{description}
\item[{Parameters}] \leavevmode\begin{itemize}
\item {} 
\textbf{pvalues} (\emph{array}) --
list of p-values

\item {} 
\textbf{m} (\emph{int or None}) --
number of tests. If None, m = pvalues.size

\item {} 
\textbf{pi0} (\emph{float or None}) --
estimate of m\_0 / m which is the (true null / total tests) ratio, if None estimation via cubic spline.

\item {} 
\textbf{verbose} (\emph{bool}) --
print info during execution

\item {} 
\textbf{lowMemory} (\emph{bool}) --
if true use low memory version

\end{itemize}

\end{description}\end{quote}
\paragraph{Notes}
\begin{itemize}
\item {} 
The q-value of a particular feature can be described as the expected proportion of
false  positives  among  all  features  as  or  more  extreme  than  the observed one

\item {} 
The estimated q-values are increasing in the same order as the p-values

\end{itemize}
\paragraph{References}
\begin{itemize}
\item {} 
\href{http://www.pnas.org/content/100/16/9440.full}{Storey and Tibshirani, 2003}

\item {} 
modified from \href{https://github.com/nfusi/qvalue}{https://github.com/nfusi/qvalue}

\end{itemize}

\end{fulllineitems}



\subparagraph{ClearMap.Analysis.Tools.StatisticalTests module}
\label{api/ClearMap.Analysis.Tools:module-ClearMap.Analysis.Tools.StatisticalTests}\label{api/ClearMap.Analysis.Tools:clearmap-analysis-tools-statisticaltests-module}\index{ClearMap.Analysis.Tools.StatisticalTests (module)}
Some statistics tests not in standard python packages
\index{testCramerVonMises2Sample() (in module ClearMap.Analysis.Tools.StatisticalTests)}

\begin{fulllineitems}
\phantomsection\label{api/ClearMap.Analysis.Tools:ClearMap.Analysis.Tools.StatisticalTests.testCramerVonMises2Sample}\pysiglinewithargsret{\bfcode{testCramerVonMises2Sample}}{\emph{x}, \emph{y}}{}
Computes the Cramer von Mises two sample test.

This is a two-sided test for the null hypothesis that 2 independent samples
are drawn from the same continuous distribution.
\begin{quote}\begin{description}
\item[{Parameters}] \leavevmode\begin{itemize}
\item {} 
\textbf{x, y} (\emph{sequence of 1-D ndarrays}) --
two arrays of sample observations

\item {} 
\textbf{assumed to be drawn from a continuous distribution, sample sizes}

\item {} 
\textbf{can be different}

\end{itemize}

\item[{Returns}] \leavevmode
\emph{(float, float)} --
T statistic, two-tailed p-value

\end{description}\end{quote}
\paragraph{References}
\begin{itemize}
\item {} 
modified from \href{https://github.com/scipy/scipy/pull/3659}{https://github.com/scipy/scipy/pull/3659}

\end{itemize}

\end{fulllineitems}



\paragraph{ClearMap.Analysis.Label module}
\label{api/ClearMap.Analysis:module-ClearMap.Analysis.Label}\label{api/ClearMap.Analysis:clearmap-analysis-label-module}\index{ClearMap.Analysis.Label (module)}
Label and annotation info from Allen Brain Atlas (v2)
\paragraph{Notes}
\begin{itemize}
\item {} 
The annotation file is assumed to be in `./Data/Annotation/annotation\_25\_right.tif'
but can be set in the constant {\hyperref[api/ClearMap.Analysis:ClearMap.Analysis.Label.DefaultLabeledImageFile]{\emph{\code{DefaultLabeledImageFile}}}}

\item {} 
The mapping between labels and brain area information is found in the
`./Data/ARA2\_annotation\_info.csv' file.
In the `./Data/ARA2\_annotation\_info\_collapse.csv' file a cross marks an area
to which all sub-areas will be collapsed.
The location of this file is set in {\hyperref[api/ClearMap.Analysis:ClearMap.Analysis.Label.DefaultAnnotationFile]{\emph{\code{DefaultAnnotationFile}}}}.

\item {} 
For consistentcy cretain labels of the Allen brain atlas without
annotation were assinged to their correct parent regions.

\item {} 
A collapse collumn in the mapping file was added to allow for a region
based collapse of statistics based on the inheritance structure of the
annotated regions. These might need to be adjusted to the particular
scientific question.

\end{itemize}
\paragraph{References}
\begin{itemize}
\item {} 
\href{http://mouse.brain-map.org/static/atlas}{Allen Brain Atlas}

\end{itemize}
\index{DefaultLabeledImageFile (in module ClearMap.Analysis.Label)}

\begin{fulllineitems}
\phantomsection\label{api/ClearMap.Analysis:ClearMap.Analysis.Label.DefaultLabeledImageFile}\pysigline{\bfcode{DefaultLabeledImageFile}\strong{ = `/Users/nicolasrenier/Documents/ClearMap/idisco/ClearMap/Test/Data/Annotation/annotation\_25\_right.tif'}}
str: default volumetric annotated image file

This file is by default the Allen brain annotated mouse atlas with 25um
isotropic reslution.

\end{fulllineitems}

\index{DefaultAnnotationFile (in module ClearMap.Analysis.Label)}

\begin{fulllineitems}
\phantomsection\label{api/ClearMap.Analysis:ClearMap.Analysis.Label.DefaultAnnotationFile}\pysigline{\bfcode{DefaultAnnotationFile}\strong{ = `/Users/nicolasrenier/Documents/ClearMap/idisco/ClearMap/Data/ARA2\_annotation\_info\_collapse.csv'}}
str: default list of labels in the annotatd image and names of annotated regoins

This file is by default the lables for the Allen brain annotated mouse
atlas with 25um isotropic reslution.

An extra column for collapse indicats how to automatically collapse data into
the different bain regions if the \code{collapse} option is given.

\end{fulllineitems}

\index{LabelRecord (class in ClearMap.Analysis.Label)}

\begin{fulllineitems}
\phantomsection\label{api/ClearMap.Analysis:ClearMap.Analysis.Label.LabelRecord}\pysiglinewithargsret{\strong{class }\bfcode{LabelRecord}}{\emph{id}, \emph{name}, \emph{acronym}, \emph{color}, \emph{parent}, \emph{collapse}}{}
Bases: \code{tuple}

Structure of a label for a annotated region
\index{\_\_getnewargs\_\_() (LabelRecord method)}

\begin{fulllineitems}
\phantomsection\label{api/ClearMap.Analysis:ClearMap.Analysis.Label.LabelRecord.__getnewargs__}\pysiglinewithargsret{\bfcode{\_\_getnewargs\_\_}}{}{}
Return self as a plain tuple.  Used by copy and pickle.

\end{fulllineitems}

\index{\_\_getstate\_\_() (LabelRecord method)}

\begin{fulllineitems}
\phantomsection\label{api/ClearMap.Analysis:ClearMap.Analysis.Label.LabelRecord.__getstate__}\pysiglinewithargsret{\bfcode{\_\_getstate\_\_}}{}{}
Exclude the OrderedDict from pickling

\end{fulllineitems}

\index{\_\_repr\_\_() (LabelRecord method)}

\begin{fulllineitems}
\phantomsection\label{api/ClearMap.Analysis:ClearMap.Analysis.Label.LabelRecord.__repr__}\pysiglinewithargsret{\bfcode{\_\_repr\_\_}}{}{}
Return a nicely formatted representation string

\end{fulllineitems}

\index{acronym (LabelRecord attribute)}

\begin{fulllineitems}
\phantomsection\label{api/ClearMap.Analysis:ClearMap.Analysis.Label.LabelRecord.acronym}\pysigline{\bfcode{acronym}}
Alias for field number 2

\end{fulllineitems}

\index{collapse (LabelRecord attribute)}

\begin{fulllineitems}
\phantomsection\label{api/ClearMap.Analysis:ClearMap.Analysis.Label.LabelRecord.collapse}\pysigline{\bfcode{collapse}}
Alias for field number 5

\end{fulllineitems}

\index{color (LabelRecord attribute)}

\begin{fulllineitems}
\phantomsection\label{api/ClearMap.Analysis:ClearMap.Analysis.Label.LabelRecord.color}\pysigline{\bfcode{color}}
Alias for field number 3

\end{fulllineitems}

\index{id (LabelRecord attribute)}

\begin{fulllineitems}
\phantomsection\label{api/ClearMap.Analysis:ClearMap.Analysis.Label.LabelRecord.id}\pysigline{\bfcode{id}}
Alias for field number 0

\end{fulllineitems}

\index{name (LabelRecord attribute)}

\begin{fulllineitems}
\phantomsection\label{api/ClearMap.Analysis:ClearMap.Analysis.Label.LabelRecord.name}\pysigline{\bfcode{name}}
Alias for field number 1

\end{fulllineitems}

\index{parent (LabelRecord attribute)}

\begin{fulllineitems}
\phantomsection\label{api/ClearMap.Analysis:ClearMap.Analysis.Label.LabelRecord.parent}\pysigline{\bfcode{parent}}
Alias for field number 4

\end{fulllineitems}


\end{fulllineitems}

\index{LabelInfo (class in ClearMap.Analysis.Label)}

\begin{fulllineitems}
\phantomsection\label{api/ClearMap.Analysis:ClearMap.Analysis.Label.LabelInfo}\pysiglinewithargsret{\strong{class }\bfcode{LabelInfo}}{\emph{slf}, \emph{annotationFile='/Users/nicolasrenier/Documents/ClearMap/idisco/ClearMap/Data/ARA2\_annotation\_info\_collapse.csv'}}{}
Bases: \code{object}

Class that holds infomration of the annotated regions
\index{ids (LabelInfo attribute)}

\begin{fulllineitems}
\phantomsection\label{api/ClearMap.Analysis:ClearMap.Analysis.Label.LabelInfo.ids}\pysigline{\bfcode{ids}\strong{ = None}}
\end{fulllineitems}

\index{names (LabelInfo attribute)}

\begin{fulllineitems}
\phantomsection\label{api/ClearMap.Analysis:ClearMap.Analysis.Label.LabelInfo.names}\pysigline{\bfcode{names}\strong{ = None}}
\end{fulllineitems}

\index{acronyms (LabelInfo attribute)}

\begin{fulllineitems}
\phantomsection\label{api/ClearMap.Analysis:ClearMap.Analysis.Label.LabelInfo.acronyms}\pysigline{\bfcode{acronyms}\strong{ = None}}
\end{fulllineitems}

\index{colors (LabelInfo attribute)}

\begin{fulllineitems}
\phantomsection\label{api/ClearMap.Analysis:ClearMap.Analysis.Label.LabelInfo.colors}\pysigline{\bfcode{colors}\strong{ = None}}
\end{fulllineitems}

\index{parents (LabelInfo attribute)}

\begin{fulllineitems}
\phantomsection\label{api/ClearMap.Analysis:ClearMap.Analysis.Label.LabelInfo.parents}\pysigline{\bfcode{parents}\strong{ = None}}
\end{fulllineitems}

\index{levels (LabelInfo attribute)}

\begin{fulllineitems}
\phantomsection\label{api/ClearMap.Analysis:ClearMap.Analysis.Label.LabelInfo.levels}\pysigline{\bfcode{levels}\strong{ = None}}
\end{fulllineitems}

\index{collapse (LabelInfo attribute)}

\begin{fulllineitems}
\phantomsection\label{api/ClearMap.Analysis:ClearMap.Analysis.Label.LabelInfo.collapse}\pysigline{\bfcode{collapse}\strong{ = None}}
\end{fulllineitems}

\index{collapseMap (LabelInfo attribute)}

\begin{fulllineitems}
\phantomsection\label{api/ClearMap.Analysis:ClearMap.Analysis.Label.LabelInfo.collapseMap}\pysigline{\bfcode{collapseMap}\strong{ = None}}
\end{fulllineitems}

\index{initialize() (LabelInfo method)}

\begin{fulllineitems}
\phantomsection\label{api/ClearMap.Analysis:ClearMap.Analysis.Label.LabelInfo.initialize}\pysiglinewithargsret{\bfcode{initialize}}{\emph{slf}, \emph{annotationFile='/Users/nicolasrenier/Documents/ClearMap/idisco/ClearMap/Data/ARA2\_annotation\_info\_collapse.csv'}}{}
\end{fulllineitems}

\index{name() (LabelInfo method)}

\begin{fulllineitems}
\phantomsection\label{api/ClearMap.Analysis:ClearMap.Analysis.Label.LabelInfo.name}\pysiglinewithargsret{\bfcode{name}}{\emph{slf}, \emph{iid}}{}
\end{fulllineitems}

\index{acronym() (LabelInfo method)}

\begin{fulllineitems}
\phantomsection\label{api/ClearMap.Analysis:ClearMap.Analysis.Label.LabelInfo.acronym}\pysiglinewithargsret{\bfcode{acronym}}{\emph{slf}, \emph{iid}}{}
\end{fulllineitems}

\index{color() (LabelInfo method)}

\begin{fulllineitems}
\phantomsection\label{api/ClearMap.Analysis:ClearMap.Analysis.Label.LabelInfo.color}\pysiglinewithargsret{\bfcode{color}}{\emph{slf}, \emph{iid}}{}
\end{fulllineitems}

\index{parent() (LabelInfo method)}

\begin{fulllineitems}
\phantomsection\label{api/ClearMap.Analysis:ClearMap.Analysis.Label.LabelInfo.parent}\pysiglinewithargsret{\bfcode{parent}}{\emph{slf}, \emph{iid}}{}
\end{fulllineitems}

\index{level() (LabelInfo method)}

\begin{fulllineitems}
\phantomsection\label{api/ClearMap.Analysis:ClearMap.Analysis.Label.LabelInfo.level}\pysiglinewithargsret{\bfcode{level}}{\emph{slf}, \emph{iid}}{}
\end{fulllineitems}

\index{toLabelAtLevel() (LabelInfo method)}

\begin{fulllineitems}
\phantomsection\label{api/ClearMap.Analysis:ClearMap.Analysis.Label.LabelInfo.toLabelAtLevel}\pysiglinewithargsret{\bfcode{toLabelAtLevel}}{\emph{slf}, \emph{iid}, \emph{level}}{}
\end{fulllineitems}

\index{toLabelAtCollapseMap() (LabelInfo method)}

\begin{fulllineitems}
\phantomsection\label{api/ClearMap.Analysis:ClearMap.Analysis.Label.LabelInfo.toLabelAtCollapseMap}\pysiglinewithargsret{\bfcode{toLabelAtCollapseMap}}{\emph{slf}, \emph{iid}}{}
\end{fulllineitems}

\index{toLabelAtCollapse() (LabelInfo method)}

\begin{fulllineitems}
\phantomsection\label{api/ClearMap.Analysis:ClearMap.Analysis.Label.LabelInfo.toLabelAtCollapse}\pysiglinewithargsret{\bfcode{toLabelAtCollapse}}{\emph{slf}, \emph{iid}}{}
\end{fulllineitems}


\end{fulllineitems}

\index{Label (in module ClearMap.Analysis.Label)}

\begin{fulllineitems}
\phantomsection\label{api/ClearMap.Analysis:ClearMap.Analysis.Label.Label}\pysigline{\bfcode{Label}\strong{ = \textless{}ClearMap.Analysis.Label.LabelInfo object\textgreater{}}}
Information on the annotated regions

\end{fulllineitems}

\index{initialize() (in module ClearMap.Analysis.Label)}

\begin{fulllineitems}
\phantomsection\label{api/ClearMap.Analysis:ClearMap.Analysis.Label.initialize}\pysiglinewithargsret{\bfcode{initialize}}{\emph{annotationFile='/Users/nicolasrenier/Documents/ClearMap/idisco/ClearMap/Data/ARA2\_annotation\_info\_collapse.csv'}}{}
\end{fulllineitems}

\index{labelAtLevel() (in module ClearMap.Analysis.Label)}

\begin{fulllineitems}
\phantomsection\label{api/ClearMap.Analysis:ClearMap.Analysis.Label.labelAtLevel}\pysiglinewithargsret{\bfcode{labelAtLevel}}{\emph{label}, \emph{level}}{}
\end{fulllineitems}

\index{labelAtCollapse() (in module ClearMap.Analysis.Label)}

\begin{fulllineitems}
\phantomsection\label{api/ClearMap.Analysis:ClearMap.Analysis.Label.labelAtCollapse}\pysiglinewithargsret{\bfcode{labelAtCollapse}}{\emph{label}}{}
\end{fulllineitems}

\index{labelPoints() (in module ClearMap.Analysis.Label)}

\begin{fulllineitems}
\phantomsection\label{api/ClearMap.Analysis:ClearMap.Analysis.Label.labelPoints}\pysiglinewithargsret{\bfcode{labelPoints}}{\emph{points}, \emph{labeledImage='/Users/nicolasrenier/Documents/ClearMap/idisco/ClearMap/Test/Data/Annotation/annotation\_25\_right.tif'}, \emph{level=None}, \emph{collapse=None}}{}
\end{fulllineitems}

\index{countPointsInRegions() (in module ClearMap.Analysis.Label)}

\begin{fulllineitems}
\phantomsection\label{api/ClearMap.Analysis:ClearMap.Analysis.Label.countPointsInRegions}\pysiglinewithargsret{\bfcode{countPointsInRegions}}{\emph{points}, \emph{labeledImage='/Users/nicolasrenier/Documents/ClearMap/idisco/ClearMap/Test/Data/Annotation/annotation\_25\_right.tif'}, \emph{intensities=None}, \emph{intensityRow=0}, \emph{level=None}, \emph{allIds=False}, \emph{sort=True}, \emph{returnIds=True}, \emph{returnCounts=False}, \emph{collapse=None}}{}
\end{fulllineitems}

\index{labelToName() (in module ClearMap.Analysis.Label)}

\begin{fulllineitems}
\phantomsection\label{api/ClearMap.Analysis:ClearMap.Analysis.Label.labelToName}\pysiglinewithargsret{\bfcode{labelToName}}{\emph{label}}{}
\end{fulllineitems}

\index{labelToAcronym() (in module ClearMap.Analysis.Label)}

\begin{fulllineitems}
\phantomsection\label{api/ClearMap.Analysis:ClearMap.Analysis.Label.labelToAcronym}\pysiglinewithargsret{\bfcode{labelToAcronym}}{\emph{label}}{}
\end{fulllineitems}

\index{labelToColor() (in module ClearMap.Analysis.Label)}

\begin{fulllineitems}
\phantomsection\label{api/ClearMap.Analysis:ClearMap.Analysis.Label.labelToColor}\pysiglinewithargsret{\bfcode{labelToColor}}{\emph{label}}{}
\end{fulllineitems}

\index{writePAL() (in module ClearMap.Analysis.Label)}

\begin{fulllineitems}
\phantomsection\label{api/ClearMap.Analysis:ClearMap.Analysis.Label.writePAL}\pysiglinewithargsret{\bfcode{writePAL}}{\emph{filename}, \emph{cols}}{}
\end{fulllineitems}

\index{writeLUT() (in module ClearMap.Analysis.Label)}

\begin{fulllineitems}
\phantomsection\label{api/ClearMap.Analysis:ClearMap.Analysis.Label.writeLUT}\pysiglinewithargsret{\bfcode{writeLUT}}{\emph{filename}, \emph{cols}}{}
\end{fulllineitems}

\index{makeColorPalette() (in module ClearMap.Analysis.Label)}

\begin{fulllineitems}
\phantomsection\label{api/ClearMap.Analysis:ClearMap.Analysis.Label.makeColorPalette}\pysiglinewithargsret{\bfcode{makeColorPalette}}{\emph{filename=None}}{}
Creates a pal file for imaris based on label colors

\end{fulllineitems}

\index{makeColorAnnotations() (in module ClearMap.Analysis.Label)}

\begin{fulllineitems}
\phantomsection\label{api/ClearMap.Analysis:ClearMap.Analysis.Label.makeColorAnnotations}\pysiglinewithargsret{\bfcode{makeColorAnnotations}}{\emph{filename}, \emph{labeledImage=None}}{}
\end{fulllineitems}

\index{test() (in module ClearMap.Analysis.Label)}

\begin{fulllineitems}
\phantomsection\label{api/ClearMap.Analysis:ClearMap.Analysis.Label.test}\pysiglinewithargsret{\bfcode{test}}{}{}
Test Label module

\end{fulllineitems}



\paragraph{ClearMap.Analysis.Statistics module}
\label{api/ClearMap.Analysis:clearmap-analysis-statistics-module}\label{api/ClearMap.Analysis:module-ClearMap.Analysis.Statistics}\index{ClearMap.Analysis.Statistics (module)}
Create some statistics to test significant changes
in voxelized and labeled data

TODO: cleanup / make generic
\index{readDataGroup() (in module ClearMap.Analysis.Statistics)}

\begin{fulllineitems}
\phantomsection\label{api/ClearMap.Analysis:ClearMap.Analysis.Statistics.readDataGroup}\pysiglinewithargsret{\bfcode{readDataGroup}}{\emph{filenames}, \emph{combine=True}, \emph{**args}}{}
Turn a list of filenames for data into a numpy stack

\end{fulllineitems}

\index{readPointsGroup() (in module ClearMap.Analysis.Statistics)}

\begin{fulllineitems}
\phantomsection\label{api/ClearMap.Analysis:ClearMap.Analysis.Statistics.readPointsGroup}\pysiglinewithargsret{\bfcode{readPointsGroup}}{\emph{filenames}, \emph{**args}}{}
Turn a list of filenames for points into a numpy stack

\end{fulllineitems}

\index{tTestVoxelization() (in module ClearMap.Analysis.Statistics)}

\begin{fulllineitems}
\phantomsection\label{api/ClearMap.Analysis:ClearMap.Analysis.Statistics.tTestVoxelization}\pysiglinewithargsret{\bfcode{tTestVoxelization}}{\emph{group1}, \emph{group2}, \emph{signed=False}, \emph{removeNaN=True}, \emph{pcutoff=None}}{}
t-Test on differences between the individual voxels in group1 and group2, group is a array of voxelizations

\end{fulllineitems}

\index{cutoffPValues() (in module ClearMap.Analysis.Statistics)}

\begin{fulllineitems}
\phantomsection\label{api/ClearMap.Analysis:ClearMap.Analysis.Statistics.cutoffPValues}\pysiglinewithargsret{\bfcode{cutoffPValues}}{\emph{pvals}, \emph{pcutoff=0.05}}{}
\end{fulllineitems}

\index{colorPValues() (in module ClearMap.Analysis.Statistics)}

\begin{fulllineitems}
\phantomsection\label{api/ClearMap.Analysis:ClearMap.Analysis.Statistics.colorPValues}\pysiglinewithargsret{\bfcode{colorPValues}}{\emph{pvals, psign, positive={[}1, 0{]}, negative={[}0, 1{]}, pcutoff=None, positivetrend={[}0, 0, 1, 0{]}, negativetrend={[}0, 0, 0, 1{]}, pmax=None}}{}
\end{fulllineitems}

\index{mean() (in module ClearMap.Analysis.Statistics)}

\begin{fulllineitems}
\phantomsection\label{api/ClearMap.Analysis:ClearMap.Analysis.Statistics.mean}\pysiglinewithargsret{\bfcode{mean}}{\emph{group}, \emph{**args}}{}
\end{fulllineitems}

\index{std() (in module ClearMap.Analysis.Statistics)}

\begin{fulllineitems}
\phantomsection\label{api/ClearMap.Analysis:ClearMap.Analysis.Statistics.std}\pysiglinewithargsret{\bfcode{std}}{\emph{group}, \emph{**args}}{}
\end{fulllineitems}

\index{var() (in module ClearMap.Analysis.Statistics)}

\begin{fulllineitems}
\phantomsection\label{api/ClearMap.Analysis:ClearMap.Analysis.Statistics.var}\pysiglinewithargsret{\bfcode{var}}{\emph{group}, \emph{**args}}{}
\end{fulllineitems}

\index{thresholdPoints() (in module ClearMap.Analysis.Statistics)}

\begin{fulllineitems}
\phantomsection\label{api/ClearMap.Analysis:ClearMap.Analysis.Statistics.thresholdPoints}\pysiglinewithargsret{\bfcode{thresholdPoints}}{\emph{points}, \emph{intensities}, \emph{threshold=0}, \emph{row=0}}{}
Threshold points by intensities

\end{fulllineitems}

\index{weightsFromPrecentiles() (in module ClearMap.Analysis.Statistics)}

\begin{fulllineitems}
\phantomsection\label{api/ClearMap.Analysis:ClearMap.Analysis.Statistics.weightsFromPrecentiles}\pysiglinewithargsret{\bfcode{weightsFromPrecentiles}}{\emph{intensities, percentiles={[}25, 50, 75, 100{]}}}{}
\end{fulllineitems}

\index{countPointsGroupInRegions() (in module ClearMap.Analysis.Statistics)}

\begin{fulllineitems}
\phantomsection\label{api/ClearMap.Analysis:ClearMap.Analysis.Statistics.countPointsGroupInRegions}\pysiglinewithargsret{\bfcode{countPointsGroupInRegions}}{\emph{pointGroup}, \emph{labeledImage='/Users/nicolasrenier/Documents/ClearMap/idisco/ClearMap/Test/Data/Annotation/annotation\_25\_right.tif'}, \emph{intensityGroup=None}, \emph{intensityRow=0}, \emph{returnIds=True}, \emph{returnCounts=False}, \emph{collapse=None}}{}
Generates a table of counts for the various point datasets in pointGroup

\end{fulllineitems}

\index{tTestPointsInRegions() (in module ClearMap.Analysis.Statistics)}

\begin{fulllineitems}
\phantomsection\label{api/ClearMap.Analysis:ClearMap.Analysis.Statistics.tTestPointsInRegions}\pysiglinewithargsret{\bfcode{tTestPointsInRegions}}{\emph{pointCounts1}, \emph{pointCounts2}, \emph{labeledImage='/Users/nicolasrenier/Documents/ClearMap/idisco/ClearMap/Test/Data/Annotation/annotation\_25\_right.tif'}, \emph{signed=False}, \emph{removeNaN=True}, \emph{pcutoff=None}, \emph{equal\_var=False}}{}
t-Test on differences in counts of points in labeled regions

\end{fulllineitems}

\index{testCompletedCumulatives() (in module ClearMap.Analysis.Statistics)}

\begin{fulllineitems}
\phantomsection\label{api/ClearMap.Analysis:ClearMap.Analysis.Statistics.testCompletedCumulatives}\pysiglinewithargsret{\bfcode{testCompletedCumulatives}}{\emph{data}, \emph{method='AndersonDarling'}, \emph{offset=None}, \emph{plot=False}}{}
Test if data sets have the same number / intensity distribution by adding max intensity counts to the smaller sized data sets and performing a distribution comparison test

\end{fulllineitems}

\index{testCompletedInvertedCumulatives() (in module ClearMap.Analysis.Statistics)}

\begin{fulllineitems}
\phantomsection\label{api/ClearMap.Analysis:ClearMap.Analysis.Statistics.testCompletedInvertedCumulatives}\pysiglinewithargsret{\bfcode{testCompletedInvertedCumulatives}}{\emph{data}, \emph{method='AndersonDarling'}, \emph{offset=None}, \emph{plot=False}}{}
Test if data sets have the same number / intensity distribution by adding zero intensity counts to the smaller sized data sets and performing a distribution comparison test on the reversed cumulative distribution

\end{fulllineitems}

\index{testCompletedCumulativesInSpheres() (in module ClearMap.Analysis.Statistics)}

\begin{fulllineitems}
\phantomsection\label{api/ClearMap.Analysis:ClearMap.Analysis.Statistics.testCompletedCumulativesInSpheres}\pysiglinewithargsret{\bfcode{testCompletedCumulativesInSpheres}}{\emph{points1}, \emph{intensities1}, \emph{points2}, \emph{intensities2}, \emph{dataSize='/Users/nicolasrenier/Documents/ClearMap/idisco/ClearMap/Test/Data/Annotation/annotation\_25\_right.tif'}, \emph{radius=100}, \emph{method='AndresonDarling'}}{}
Performs completed cumulative distribution tests for each pixel using points in a ball centered at that cooridnates, returns 4 arrays p value, statistic value, number in each group

\end{fulllineitems}

\index{test() (in module ClearMap.Analysis.Statistics)}

\begin{fulllineitems}
\phantomsection\label{api/ClearMap.Analysis:ClearMap.Analysis.Statistics.test}\pysiglinewithargsret{\bfcode{test}}{}{}
Test the statistics array

\end{fulllineitems}



\paragraph{ClearMap.Analysis.Voxelization module}
\label{api/ClearMap.Analysis:clearmap-analysis-voxelization-module}

\subsubsection{ClearMap.Visualization package}
\label{api/ClearMap.Visualization:module-ClearMap.Visualization}\label{api/ClearMap.Visualization::doc}\label{api/ClearMap.Visualization:clearmap-visualization-package}\index{ClearMap.Visualization (module)}
This sub-package provides tools for the visualization of the alignment and
analysis results

Author
\begin{quote}

Christoph Kirst, The Rockefeller University, New York City, 2015
\end{quote}


\paragraph{ClearMap.Visualization.Plot module}
\label{api/ClearMap.Visualization:clearmap-visualization-plot-module}

\subsubsection{ClearMap.Parameter module}
\label{api/ClearMap.Parameter:module-ClearMap.Parameter}\label{api/ClearMap.Parameter:clearmap-parameter-module}\label{api/ClearMap.Parameter::doc}\index{ClearMap.Parameter (module)}
ClearMap default parameter module.

This module defines default parameter used by various sub-packages.


\strong{See also:}


{\hyperref[api/ClearMap.Settings:module-ClearMap.Settings]{\emph{\code{Settings}}}}


\index{detectCellParameter (in module ClearMap.Parameter)}

\begin{fulllineitems}
\phantomsection\label{api/ClearMap.Parameter:ClearMap.Parameter.detectCellParameter}\pysigline{\bfcode{detectCellParameter}\strong{ = \{`findExtendedMaximaParameter': \{`threshold': 0, `save': None, `verbose': False, `hMax': 20, `size': 5\}, `correctIlluminationParameter': \{`scaling': `Mean', `flatfield': True, `save': None, `verbose': False, `background': None\}, `removeBackgroundParameter': \{`save': None, `verbose': False, `size': (15, 15)\}, `filterDoGParameter': \{`save': None, `sigma2': None, `sigma': None, `verbose': False, `size': (7, 7, 11)\}, `detectCellShapeParameter': \{`threshold': 700, `save': None, `verbose': False\}, `findIntensityParameter': \{`method': `Max', `size': (3, 3, 3)\}\}}}
dict: Paramters for cell detection using the spot detection algorithm


\strong{See also:}


{\hyperref[api/ClearMap.Parameter:ClearMap.Parameter.IlastikParameter]{\emph{\code{IlastikParameter}}}}, \code{StackProcessingParameter}



\end{fulllineitems}

\index{IlastikParameter (in module ClearMap.Parameter)}

\begin{fulllineitems}
\phantomsection\label{api/ClearMap.Parameter:ClearMap.Parameter.IlastikParameter}\pysigline{\bfcode{IlastikParameter}\strong{ = \{`rescale': None, `backgroundSize': (15, 15), `classifier': `/Test/Ilastik/classifier.h5'\}}}
dict: Paramters for cell detection using Ilastik classification
\begin{itemize}
\item {} 
``classifier'': ilastic classifier to use

\item {} 
``rescale'': rescale images before classification

\item {} 
``backgroundSize'': Background correctoin: None or (y,x) which is size of disk for gray scale opening

\end{itemize}


\strong{See also:}


\code{SpotDetectionParameter}, \code{StackProcessingParameter}



\end{fulllineitems}

\index{processStackParameter (in module ClearMap.Parameter)}

\begin{fulllineitems}
\phantomsection\label{api/ClearMap.Parameter:ClearMap.Parameter.processStackParameter}\pysigline{\bfcode{processStackParameter}\strong{ = \{`chunkOptimizationSize': \textless{}built-in function all\textgreater{}, `processes': 2, `chunkSizeMin': 30, `chunkOptimization': True, `cChunkOverlap': 15, `chunkSizeMax': 100\}}}~\begin{description}
\item[{dict: Parameter for processing an image stack in parallel}] \leavevmode\begin{itemize}
\item {} 
``processes'': max number of parallel processes

\item {} 
``chunkSizeMax'' : maximal chunk size in z

\item {} 
``chunkSizeMin'' : minimal chunk size in z,

\item {} 
``chunkOverlap'' : overlap between two chunks,

\item {} 
``chunkOptimization'': optimize chunk size and number to number of processes

\item {} 
``chunkOptimizationSize'': increase chunk size for optimizaition (True, False or all = automatic)

\end{itemize}

\end{description}


\strong{See also:}


\code{SpotDetectionParameter}, {\hyperref[api/ClearMap.Parameter:ClearMap.Parameter.IlastikParameter]{\emph{\code{IlastikParameter}}}}



\end{fulllineitems}

\index{AlignmentParameter (in module ClearMap.Parameter)}

\begin{fulllineitems}
\phantomsection\label{api/ClearMap.Parameter:ClearMap.Parameter.AlignmentParameter}\pysigline{\bfcode{AlignmentParameter}\strong{ = \{`fixedImageMask': None, `alignmentDirectory': None, `movingImage': `/Test/Data/Elastix/150524\_0\_8X-s3-20HFautofluor\_18-51-1-warpable.tif', `affineParameterFile': `/Test/Elastix/ElastixParameterAffine.txt', `bSplineParameterFile': `/Test/Elastix/ElastixParameterBSpline.txt', `fixedImage': `/Test/Data/Elastix/OstenRefARA\_v2\_lowerHalf.tif'\}}}
dict: Parameter for Elastix alignment
\begin{itemize}
\item {} 
``alignmentDirectory'' : directory to save the alignment result

\item {} 
``movingImage'': image to be aligned

\item {} 
``fixedImage'':  reference image

\item {} 
``affineParameterFile'': elastix parameter files for affine alignment

\item {} 
``bSplineParameterFile'' : elastix parameter files for non-linear alignment

\end{itemize}


\strong{See also:}


\code{Elastix}



\end{fulllineitems}

\index{ResamplingParameter (in module ClearMap.Parameter)}

\begin{fulllineitems}
\phantomsection\label{api/ClearMap.Parameter:ClearMap.Parameter.ResamplingParameter}\pysigline{\bfcode{ResamplingParameter}\strong{ = \{`orientation': None, `source': None, `resolutionSink': (25, 25, 25), `sink': None, `resolutionSource': (4.0625, 4.0625, 3)\}}}
dict: Parameter for resampling data
\begin{quote}
\begin{itemize}
\item {} 
``source'' : data source file

\item {} 
``sink''   : data output file

\end{itemize}

``resolutionSource'': resolution of the raw data (in um / pixel) as (x,y,z)

``resolutionSink'' : resolution of the reference / atlas image (in um/ pixel) as (x,y,z)
\begin{description}
\item[{``orientation''}] \leavevmode{[}Orientation of the data set wrt reference as (x=1,y=2,z=3){]}
(-axis will invert the orientation, for other hemisphere use (-1, 2, 3), to exchnge x,y use (2,1,3) etc)

\end{description}
\end{quote}


\strong{See also:}


\code{Resampling}



\end{fulllineitems}

\index{VoxelizationParameter (in module ClearMap.Parameter)}

\begin{fulllineitems}
\phantomsection\label{api/ClearMap.Parameter:ClearMap.Parameter.VoxelizationParameter}\pysigline{\bfcode{VoxelizationParameter}\strong{ = \{`method': `Spherical', `voxelizationSize': (1, 1, 1)\}}}
dict: Parameter to calculate density voxelization
\begin{itemize}
\item {} 
``method'': Method to voxelize: `Spherical','Rectangular, `Gaussian'

\item {} 
``voxelizationSize'': max size of the volume to be voxelized

\end{itemize}


\strong{See also:}


\code{voxelization}



\end{fulllineitems}



\subsubsection{ClearMap.Settings module}
\label{api/ClearMap.Settings:clearmap-settings-module}\label{api/ClearMap.Settings::doc}\label{api/ClearMap.Settings:module-ClearMap.Settings}\index{ClearMap.Settings (module)}
Module to set \emph{ClearMap's} internal parameter and paths to external programs.
\paragraph{Notes}

Edit the {\hyperref[api/ClearMap.Settings:ClearMap.Settings.setup]{\emph{\code{setup()}}}} routine to point to the ilastik and elastix paths
for specific hosts


\strong{See also:}

\begin{itemize}
\item {} 
{\hyperref[api/ClearMap.Settings:ClearMap.Settings.IlastikPath]{\emph{\code{IlastikPath}}}}

\item {} 
{\hyperref[api/ClearMap.Settings:ClearMap.Settings.ElastixPath]{\emph{\code{ElastixPath}}}}

\item {} 
{\hyperref[api/ClearMap.Parameter:module-ClearMap.Parameter]{\emph{\code{Parameter}}}}

\end{itemize}


\index{IlastikPath (in module ClearMap.Settings)}

\begin{fulllineitems}
\phantomsection\label{api/ClearMap.Settings:ClearMap.Settings.IlastikPath}\pysigline{\bfcode{IlastikPath}\strong{ = None}}
str: Absolute path to the Ilastik 0.5 installation
\paragraph{Notes}

\href{http://ilastik.org/}{Ilastik Webpage}

\href{http://old.ilastik.org/}{Ilastik 0.5 Download}

\end{fulllineitems}

\index{ElastixPath (in module ClearMap.Settings)}

\begin{fulllineitems}
\phantomsection\label{api/ClearMap.Settings:ClearMap.Settings.ElastixPath}\pysigline{\bfcode{ElastixPath}\strong{ = None}}
str: Absolue path to the elastix installation
\paragraph{Notes}

\href{http://elastix.isi.uu.nl/}{Elastix Webpage}

\end{fulllineitems}

\index{setup() (in module ClearMap.Settings)}

\begin{fulllineitems}
\phantomsection\label{api/ClearMap.Settings:ClearMap.Settings.setup}\pysiglinewithargsret{\bfcode{setup}}{}{}
Setup ClearMap for specific hosts
\paragraph{Notes}

Edit this routine to include special setttings for specific hosts


\strong{See also:}


{\hyperref[api/ClearMap.Settings:ClearMap.Settings.IlastikPath]{\emph{\code{IlastikPath}}}}, {\hyperref[api/ClearMap.Settings:ClearMap.Settings.ElastixPath]{\emph{\code{ElastixPath}}}}



\end{fulllineitems}

\index{clearMapPath() (in module ClearMap.Settings)}

\begin{fulllineitems}
\phantomsection\label{api/ClearMap.Settings:ClearMap.Settings.clearMapPath}\pysiglinewithargsret{\bfcode{clearMapPath}}{}{}
Returns root path to the ClearMap software
\begin{quote}\begin{description}
\item[{Returns}] \leavevmode
\emph{str} --
root path to ClearMap

\end{description}\end{quote}

\end{fulllineitems}

\index{ClearMapPath (in module ClearMap.Settings)}

\begin{fulllineitems}
\phantomsection\label{api/ClearMap.Settings:ClearMap.Settings.ClearMapPath}\pysigline{\bfcode{ClearMapPath}\strong{ = `/Users/nicolasrenier/Documents/ClearMap/idisco/ClearMap'}}
str: Absolute path to the ClearMap root folder

\end{fulllineitems}



\subsubsection{ClearMap.Utils package}
\label{api/ClearMap.Utils:module-ClearMap.Utils}\label{api/ClearMap.Utils::doc}\label{api/ClearMap.Utils:clearmap-utils-package}\index{ClearMap.Utils (module)}
This sub-package provides utility functions used throughout the package

Author
\begin{quote}

Christoph Kirst, The Rockefeller University, New York City, 2015
\end{quote}


\paragraph{ClearMap.Utils.ParameterTools module}
\label{api/ClearMap.Utils:module-ClearMap.Utils.ParameterTools}\label{api/ClearMap.Utils:clearmap-utils-parametertools-module}\index{ClearMap.Utils.ParameterTools (module)}
ParameterTools

Provides simple formatting tools to handle / print parameter dictionaries
organized as key:value pairs.

Author
\begin{quote}

Christoph Kirst, The Rockefeller University, New York City, 2015
\end{quote}
\index{getParameter() (in module ClearMap.Utils.ParameterTools)}

\begin{fulllineitems}
\phantomsection\label{api/ClearMap.Utils:ClearMap.Utils.ParameterTools.getParameter}\pysiglinewithargsret{\bfcode{getParameter}}{\emph{parameter}, \emph{key}, \emph{default=None}}{}
Gets a parameter from a dict, returns default value if not defined
\begin{quote}\begin{description}
\item[{Parameters}] \leavevmode\begin{itemize}
\item {} 
\textbf{parameter} (\emph{dict}) --
parameter dictionary

\item {} 
\textbf{key} (\emph{object}) --
key

\item {} 
\textbf{default} (\emph{object}) --
deault return value if parameter not defined

\end{itemize}

\item[{Returns}] \leavevmode
\emph{object} --
parameter value for key

\end{description}\end{quote}

\end{fulllineitems}

\index{writeParameter() (in module ClearMap.Utils.ParameterTools)}

\begin{fulllineitems}
\phantomsection\label{api/ClearMap.Utils:ClearMap.Utils.ParameterTools.writeParameter}\pysiglinewithargsret{\bfcode{writeParameter}}{\emph{head=None}, \emph{out=None}, \emph{**args}}{}
Writes parameter settings in a formatted way
\begin{quote}\begin{description}
\item[{Parameters}] \leavevmode\begin{itemize}
\item {} 
\textbf{head} (\emph{str or None}) --
prefix of each line

\item {} 
\textbf{out} (\emph{object or None}) --
write to a specific output, if None return string

\item {} 
\textbf{**args} --
the parameter values as key=value arguments

\end{itemize}

\item[{Returns}] \leavevmode
\emph{str or None} --
a formated string with parameter info

\end{description}\end{quote}

\end{fulllineitems}

\index{joinParameter() (in module ClearMap.Utils.ParameterTools)}

\begin{fulllineitems}
\phantomsection\label{api/ClearMap.Utils:ClearMap.Utils.ParameterTools.joinParameter}\pysiglinewithargsret{\bfcode{joinParameter}}{\emph{*args}}{}
Joins dictionaries in a consitent way

For multiple occurences of a key the  value is defined by the first key : value pair.
\begin{quote}\begin{description}
\item[{Parameters}] \leavevmode
\textbf{*args} --
list of parameter dictonaries

\item[{Returns}] \leavevmode
\emph{dict} --
the joined dictionary

\end{description}\end{quote}

\end{fulllineitems}



\paragraph{ClearMap.Utils.ProcessWriter module}
\label{api/ClearMap.Utils:module-ClearMap.Utils.ProcessWriter}\label{api/ClearMap.Utils:clearmap-utils-processwriter-module}\index{ClearMap.Utils.ProcessWriter (module)}
Provides simple formatting tools to print text with parallel process header

Author
\begin{quote}

Christoph Kirst, The Rockefeller University, New York City, 2015
\end{quote}
\index{ProcessWriter (class in ClearMap.Utils.ProcessWriter)}

\begin{fulllineitems}
\phantomsection\label{api/ClearMap.Utils:ClearMap.Utils.ProcessWriter.ProcessWriter}\pysiglinewithargsret{\strong{class }\bfcode{ProcessWriter}}{\emph{process=0}}{}
Bases: \code{object}

Class to handle writing from parallel processes
\index{process (ProcessWriter attribute)}

\begin{fulllineitems}
\phantomsection\label{api/ClearMap.Utils:ClearMap.Utils.ProcessWriter.ProcessWriter.process}\pysigline{\bfcode{process}}
\emph{int}

the process number

\end{fulllineitems}

\index{writeString() (ProcessWriter method)}

\begin{fulllineitems}
\phantomsection\label{api/ClearMap.Utils:ClearMap.Utils.ProcessWriter.ProcessWriter.writeString}\pysiglinewithargsret{\bfcode{writeString}}{\emph{text}}{}
Generate string with process prefix
\begin{quote}\begin{description}
\item[{Parameters}] \leavevmode
\textbf{text} (\emph{str}) --
the text input

\item[{Returns}] \leavevmode
\emph{str} --
text with {[}process prefix

\end{description}\end{quote}

\end{fulllineitems}

\index{write() (ProcessWriter method)}

\begin{fulllineitems}
\phantomsection\label{api/ClearMap.Utils:ClearMap.Utils.ProcessWriter.ProcessWriter.write}\pysiglinewithargsret{\bfcode{write}}{\emph{text}}{}
Write string with process prefix to sys.stdout
\begin{quote}\begin{description}
\item[{Parameters}] \leavevmode
\textbf{text} (\emph{str}) --
the text input

\end{description}\end{quote}

\end{fulllineitems}


\end{fulllineitems}



\paragraph{ClearMap.Utils.Timer module}
\label{api/ClearMap.Utils:module-ClearMap.Utils.Timer}\label{api/ClearMap.Utils:clearmap-utils-timer-module}\index{ClearMap.Utils.Timer (module)}
Provides tools for timing

Author
\begin{quote}

Christoph Kirst, The Rockefeller University, New York City, 2015
\end{quote}
\index{Timer (class in ClearMap.Utils.Timer)}

\begin{fulllineitems}
\phantomsection\label{api/ClearMap.Utils:ClearMap.Utils.Timer.Timer}\pysiglinewithargsret{\strong{class }\bfcode{Timer}}{\emph{verbose=False}}{}
Bases: \code{object}

Class to stop time and print results in formatted way
\index{time (Timer attribute)}

\begin{fulllineitems}
\phantomsection\label{api/ClearMap.Utils:ClearMap.Utils.Timer.Timer.time}\pysigline{\bfcode{time}}
\emph{float}

the time since the timer was started

\end{fulllineitems}

\index{start() (Timer method)}

\begin{fulllineitems}
\phantomsection\label{api/ClearMap.Utils:ClearMap.Utils.Timer.Timer.start}\pysiglinewithargsret{\bfcode{start}}{}{}
Start the timer

\end{fulllineitems}

\index{reset() (Timer method)}

\begin{fulllineitems}
\phantomsection\label{api/ClearMap.Utils:ClearMap.Utils.Timer.Timer.reset}\pysiglinewithargsret{\bfcode{reset}}{}{}
Reset the timer

\end{fulllineitems}

\index{elapsedTime() (Timer method)}

\begin{fulllineitems}
\phantomsection\label{api/ClearMap.Utils:ClearMap.Utils.Timer.Timer.elapsedTime}\pysiglinewithargsret{\bfcode{elapsedTime}}{\emph{head=None}, \emph{asstring=True}}{}
Calculate elapsed time and return as formated string
\begin{quote}\begin{description}
\item[{Parameters}] \leavevmode\begin{itemize}
\item {} 
\textbf{head} (\emph{str or None}) --
prefix to the string

\item {} 
\textbf{asstring} (\emph{bool}) --
return as string or float

\end{itemize}

\item[{Returns}] \leavevmode
\emph{str or float} --
elapsed time

\end{description}\end{quote}

\end{fulllineitems}

\index{printElapsedTime() (Timer method)}

\begin{fulllineitems}
\phantomsection\label{api/ClearMap.Utils:ClearMap.Utils.Timer.Timer.printElapsedTime}\pysiglinewithargsret{\bfcode{printElapsedTime}}{\emph{head=None}}{}
Print elapsed time as formated string
\begin{quote}\begin{description}
\item[{Parameters}] \leavevmode
\textbf{head} (\emph{str or None}) --
prefix to the string

\end{description}\end{quote}

\end{fulllineitems}

\index{formatElapsedTime() (Timer method)}

\begin{fulllineitems}
\phantomsection\label{api/ClearMap.Utils:ClearMap.Utils.Timer.Timer.formatElapsedTime}\pysiglinewithargsret{\bfcode{formatElapsedTime}}{\emph{t}}{}
Format time to string
\begin{quote}\begin{description}
\item[{Parameters}] \leavevmode
\textbf{t} (\emph{float}) --
time in seconds prefix

\item[{Returns}] \leavevmode
\emph{str} --
time as hours:minutes:seconds

\end{description}\end{quote}

\end{fulllineitems}


\end{fulllineitems}



\chapter{Indices}
\label{index:indices}\begin{itemize}
\item {} 
\DUspan{xref,std,std-ref}{genindex}

\item {} 
\DUspan{xref,std,std-ref}{modindex}

\item {} 
\DUspan{xref,std,std-ref}{search}

\end{itemize}


\chapter{Author and License}
\label{index:author-and-license}

\section{ClearMap lead programming and design:}
\label{index:clearmap-lead-programming-and-design}
Christoph Kirst,
\emph{The Rockefeller University}


\section{Scripts and specific applications:}
\label{index:scripts-and-specific-applications}
Nicolas Renier and Christoph Kirst
\emph{The Rockefeller University}


\section{Documentation:}
\label{index:documentation}
Christoph Kirst and Nicolas Renier
\emph{The Rockefeller University}


\section{Additional programming and consulting:}
\label{index:additional-programming-and-consulting}
Kannan Umadevi Venkataraju
\emph{Cold Spring Harbor Laboratories}


\section{License}
\label{index:license}
GNU GENERAL PUBLIC LICENSE Version 3

See \code{LICENSE} or \href{http://www.gnu.org/licenses/gpl-3.0.en.html}{gnu.org} for details.

\begin{thebibliography}{Erturk2012}
\bibitem[Erturk2012]{Erturk2012}{\phantomsection\label{introduction:erturk2012} 
\href{http://dx.doi.org/10.1038/nprot.2012.119}{Three-dimensional imaging of solvent-cleared organs using 3DISCO.iDISCO: A Simple, Rapid Method to Immunolabel Large Tissue Samples, A. Erturk, et al. Nat. Protocol 2012}
}
\bibitem[Renier2014]{Renier2014}{\phantomsection\label{introduction:renier2014} 
\href{http://dx.doi.org/10.1016/j.cell.2014.10.010}{iDISCO: A Simple, Rapid Method to Immunolabel Large Tissue Samples
for Volume Imaging, N. Renier, et al. Cell 2014}
}
\bibitem[Renier2015]{Renier2015}{\phantomsection\label{introduction:renier2015} 
`Mapping brain activity in the mouse at cellular resolution
with volume imaging using immediate early genes, N. Renier, et al. in prep.
}
\bibitem[iDISCO]{iDISCO}{\phantomsection\label{introduction:idisco} 
{\color{red}\bfseries{}{}`iDISCO webpage, http://idisco.info/ \textless{} http://idisco.info/\textgreater{}{}`\_}
}
\bibitem[ABA]{ABA}{\phantomsection\label{introduction:aba} 
\href{http://www.brain-map.org/}{Allen Brain Atlas, http://www.brain-map.org/.}
}
\bibitem[Elastix]{Elastix}{\phantomsection\label{introduction:elastix} 
\href{http://elastix.isi.uu.nl}{Elastix toolbox for rigid and nonrigid registration of
images, http://elastix.isi.uu.nl}
}
\bibitem[Ilastik]{Ilastik}{\phantomsection\label{introduction:ilastik} 
\href{http://ilastik.org/}{Ilastik the interactive learning and segmentation toolkit,
http://ilastik.org/}
}
\bibitem[ImageJ]{ImageJ}{\phantomsection\label{introduction:imagej} 
\href{http://imagej.net/Welcome}{ImageJ}
}
\end{thebibliography}


\renewcommand{\indexname}{Python Module Index}
\begin{theindex}
\def\bigletter#1{{\Large\sffamily#1}\nopagebreak\vspace{1mm}}
\bigletter{c}
\item {\texttt{ClearMap}}, \pageref{api/ClearMap:module-ClearMap}
\item {\texttt{ClearMap.Alignment}}, \pageref{api/ClearMap.Alignment:module-ClearMap.Alignment}
\item {\texttt{ClearMap.Analysis}}, \pageref{api/ClearMap.Analysis:module-ClearMap.Analysis}
\item {\texttt{ClearMap.Analysis.Label}}, \pageref{api/ClearMap.Analysis:module-ClearMap.Analysis.Label}
\item {\texttt{ClearMap.Analysis.Statistics}}, \pageref{api/ClearMap.Analysis:module-ClearMap.Analysis.Statistics}
\item {\texttt{ClearMap.Analysis.Tools}}, \pageref{api/ClearMap.Analysis.Tools:module-ClearMap.Analysis.Tools}
\item {\texttt{ClearMap.Analysis.Tools.Extrapolate}}, \pageref{api/ClearMap.Analysis.Tools:module-ClearMap.Analysis.Tools.Extrapolate}
\item {\texttt{ClearMap.Analysis.Tools.MultipleComparisonCorrection}}, \pageref{api/ClearMap.Analysis.Tools:module-ClearMap.Analysis.Tools.MultipleComparisonCorrection}
\item {\texttt{ClearMap.Analysis.Tools.StatisticalTests}}, \pageref{api/ClearMap.Analysis.Tools:module-ClearMap.Analysis.Tools.StatisticalTests}
\item {\texttt{ClearMap.ImageProcessing}}, \pageref{api/ClearMap.ImageProcessing:module-ClearMap.ImageProcessing}
\item {\texttt{ClearMap.ImageProcessing.Filter}}, \pageref{api/ClearMap.ImageProcessing.Filter:module-ClearMap.ImageProcessing.Filter}
\item {\texttt{ClearMap.ImageProcessing.Filter.Convolution}}, \pageref{api/ClearMap.ImageProcessing.Filter:module-ClearMap.ImageProcessing.Filter.Convolution}
\item {\texttt{ClearMap.ImageProcessing.Filter.FilterKernel}}, \pageref{api/ClearMap.ImageProcessing.Filter:module-ClearMap.ImageProcessing.Filter.FilterKernel}
\item {\texttt{ClearMap.ImageProcessing.Filter.StructureElement}}, \pageref{api/ClearMap.ImageProcessing.Filter:module-ClearMap.ImageProcessing.Filter.StructureElement}
\item {\texttt{ClearMap.ImageProcessing.ImageStatistics}}, \pageref{api/ClearMap.ImageProcessing:module-ClearMap.ImageProcessing.ImageStatistics}
\item {\texttt{ClearMap.ImageProcessing.StackProcessing}}, \pageref{api/ClearMap.ImageProcessing:module-ClearMap.ImageProcessing.StackProcessing}
\item {\texttt{ClearMap.IO}}, \pageref{api/ClearMap.IO:module-ClearMap.IO}
\item {\texttt{ClearMap.IO.CSV}}, \pageref{api/ClearMap.IO:module-ClearMap.IO.CSV}
\item {\texttt{ClearMap.IO.FileList}}, \pageref{api/ClearMap.IO:module-ClearMap.IO.FileList}
\item {\texttt{ClearMap.IO.IO}}, \pageref{api/ClearMap.IO:module-ClearMap.IO.IO}
\item {\texttt{ClearMap.IO.NPY}}, \pageref{api/ClearMap.IO:module-ClearMap.IO.NPY}
\item {\texttt{ClearMap.IO.NRRD}}, \pageref{api/ClearMap.IO:module-ClearMap.IO.NRRD}
\item {\texttt{ClearMap.IO.VTK}}, \pageref{api/ClearMap.IO:module-ClearMap.IO.VTK}
\item {\texttt{ClearMap.Parameter}}, \pageref{api/ClearMap.Parameter:module-ClearMap.Parameter}
\item {\texttt{ClearMap.Settings}}, \pageref{api/ClearMap.Settings:module-ClearMap.Settings}
\item {\texttt{ClearMap.Utils}}, \pageref{api/ClearMap.Utils:module-ClearMap.Utils}
\item {\texttt{ClearMap.Utils.ParameterTools}}, \pageref{api/ClearMap.Utils:module-ClearMap.Utils.ParameterTools}
\item {\texttt{ClearMap.Utils.ProcessWriter}}, \pageref{api/ClearMap.Utils:module-ClearMap.Utils.ProcessWriter}
\item {\texttt{ClearMap.Utils.Timer}}, \pageref{api/ClearMap.Utils:module-ClearMap.Utils.Timer}
\item {\texttt{ClearMap.Visualization}}, \pageref{api/ClearMap.Visualization:module-ClearMap.Visualization}
\end{theindex}

\renewcommand{\indexname}{Index}
\printindex
\end{document}
